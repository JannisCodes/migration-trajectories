% define document type (i.e., template. Here: A4 APA manuscript with 12pt font)
\documentclass[man, 12pt, a4paper]{apa7}

% change margins (e.g., for margin comments):
%\usepackage{geometry}
% \geometry{
% a4paper,
% marginparwidth=30mm,
% right=50mm,
%}

% add packages
\usepackage[american]{babel}
\usepackage[utf8]{inputenc}
\usepackage{csquotes}
\usepackage{hyperref}
\usepackage[style=apa, sortcites=true, sorting=nyt, backend=biber, natbib=true, uniquename=false, uniquelist=false, useprefix=true]{biblatex}
\usepackage{authblk}
\usepackage{graphicx}
\usepackage{setspace,caption}
\usepackage{subcaption}
\usepackage{enumitem}
\usepackage{lipsum}
\usepackage{soul}
\usepackage{xcolor}
\usepackage{fourier}
\usepackage{stackengine}
\usepackage{scalerel}
\usepackage{fontawesome5}
\usepackage[normalem]{ulem}
% \usepackage{longtable}
\usepackage{amsmath, nccmath}
\usepackage{mdframed}
\usepackage{ntheorem}
\usepackage{afterpage}
\usepackage{float}
\usepackage{array}
\usepackage{censor}
\usepackage{pdflscape}
\usepackage{lscape}
\usepackage{pdfpages}
\usepackage{enumitem}
\usepackage{caption}
\usepackage{adjustbox}
\usepackage{makecell}
\usepackage{tabu}

% make warning with red triangle
\newcommand\Warning[1][2ex]{%
  \renewcommand\stacktype{L}%
  \scaleto{\stackon[1.3pt]{\color{red}$\triangle$}{\tiny\bfseries !}}{#1}}%

% make question with red triangle
\newcommand\Question[1][2ex]{%
  \renewcommand\stacktype{L}%
  \scaleto{\stackon[1.3pt]{\color{red}$\triangle$}{\tiny\bfseries ?}}{#1}}%
  
% add definition sections
\theoremstyle{break}
\newtheorem{definition}{Definition}

% add hypothesis sections
\theoremstyle{plain}
\theoremseparator{:}
\newtheorem{hyp}{Hypothesis}

\newtheorem{subhyp}{Hypothesis}
   \renewcommand\thesubhyp{\thehyp\alph{subhyp}}

% add quote section
\usepackage{csquotes}

% framed box section
\usepackage{framed}
\emergencystretch=1em

% formatting links in the PDF file
\hypersetup{
pdfpagemode={UseOutlines},
bookmarksopen=true,
bookmarksopenlevel=0,
hypertexnames=false,
colorlinks   = true, %Colours links instead of ugly boxes
urlcolor     = blue, %Colour for external hyperlinks
linkcolor    = blue, %Colour of internal links
citecolor   = cyan, %Colour of citations
pdfstartview={FitV},
unicode,
breaklinks=true,
}

% ref labels
\newcommand{\fgrref}[2][]{\hyperref[#2]{Figure \ref*{#2}#1}}
\newcommand{\tblref}[2][]{\hyperref[#2]{Table \ref*{#2}#1}}
\newcommand{\appref}[2][]{\hyperref[#2]{Appendix \ref*{#2}#1}}

% custom open science badge height
\newlength{\badgeheight}
\setlength{\badgeheight}{1em}

% language settings
\DeclareLanguageMapping{american}{american-apa}

% add reference library file
\addbibresource{referencesZotero.bib}

% Title and header
\title{Describe and Explore: Using feature-based time-series clustering to understand modern intensive longitudinal data (Working Title)}
\shorttitle{Migration Experience Trajectories}

% Authors
\author[*,1,3]{Jannis Kreienkamp}
\author[1,3]{Laura F. Bringmann}
\author[1,3]{Kai Epstude}
\author[1,3]{Maximilian Agostini}
\author[1,3]{Peter de Jonge}
\author[2,3]{Rei Monden}
\affiliation{\hfill}

\affil[1]{University of Groningen, Department of Psychology}
\affil[2]{Hiroshima University, Graduate School of Advanced Science and Engineering}
\affil[3]{Author order TBD (currently in first name alphabetical order)}

\authornote{   
   \addORCIDlink{* Jannis Kreienkamp}{0000-0002-1831-5604}\\
   \addORCIDlink{Laura F. Bringmann}{0000-0002-8091-9935}\\
   \addORCIDlink{Kai Epstude}{0000-0001-9817-3847}\\
   \addORCIDlink{Maximilian Agostini}{0000-0001-6435-7621}\\
   \addORCIDlink{Peter de Jonge}{0000-0002-0866-6929}\\
   \addORCIDlink{Rei Monden}{0000-0003-1744-5447}

We have no known conflict of interest to declare. The authors received no specific funding for this work. Materials and  software is available at \url{https://janniscodes.github.io/migration-trajectories/}  \citep{KreienkampTBD}. Protocols, materials, data, and code are available at \url{https://osf.io/TBA} \citep{KreienkampTBD}. The preregistration of our analysis can be accessed as part of our Open Science Framework repository \citep{KreienkampTBD}.

Correspondence concerning this article should be addressed to Jannis Kreienkamp, Department of Psychology, University of Groningen, Grote Kruisstraat 2/1, 9712 TS Groningen (The Netherlands).  E-mail: \href{mailto:j.kreienkamp@rug.nl}{j.kreienkamp@rug.nl}
}

\leftheader{Kreienkamp}

% Abstract
\abstract{
Abstract to be written.

\noindent\textbf{Public significance statement}: To be written.

}

\keywords{
    TBD\\
    \vspace{1em}
    \textit{Open Science Practices [?]:}
    \noindent \href{https://osf.io}{\includegraphics[height=\badgeheight]{assets/open-badges-small/registration-color.png}} Preregistration, 
    \href{https://osf.io}{\includegraphics[height=\badgeheight]{assets/open-badges-small/material-color.png}} Open Materials, 
    \href{https://osf.io}{\includegraphics[height=\badgeheight]{assets/open-badges-small/data-color.png}} Open Data, \break
    \href{https://osf.io}{\includegraphics[height=\badgeheight]{assets/open-badges-small/code-color.png}} Open Code, 
    \href{https://osf.io}{\includegraphics[height=\badgeheight]{assets/open-badges-small/supplements-color.png}} Open Supplements
}


% set indentation size
\setlength\parindent{1.27cm}

% Start of the main document:
\begin{document}

% add title information (incl. title page and abstract)
\maketitle

% **CHEAT SHEET / LEGEND**
%
% Comments:
% '%' starts a comment in LaTeX (not printed)
% '\todo[inline]{} makes orange boxes in PDF
% '\marginpar{}' notes in margins
% '\footnote{}' footnote
% '\Warning' important note indicator in PDF (triangle with exclamation mark)
% '\Question' question note indicator in PDF (triangle with question mark)
%
% Citation (with Natbib citation style):
% '\citep[e.g.][p. 15]{CitationKey}' citation in parentheses "(e.g., Berry, 2003, p. 15)"
% '\citet{CitationKey}' citation in text "Berry (2003)"
% '\citealt' and '\citealp' alternate citation without parentheses
% '\citeauthor' and '\citeyear' only year or author
% 
% Headings:
% '\part{}' and '\chapter{}' only relevant for multi-part or multi-chapter documents
% '\section{}' heading level 1
% '\subsection{}' heading level 2
% '\subsubsection{}' heading level 3
% '\paragraph{}' heading level 4
% '\subparagraph{}' heading level 5
%
% formatting:
% '\textbf{}' text bold font
% '\textit{}' text italic font
% '\underline{}' text underline
% '\sout{}' text strike out
% '\textsc{}' text small caps
% '\vspace{1em}' add vertical space
% '\hspace{1em}' add horizontal space
% '\\' new line (i.e., line break)
% '\pagebreak' start new page (i.e., page break)
% '\noindent' do not indent current line (e.g., current paragraph)
% 'begin{center}...end{center}' center text or object
%
% Math mode:
% '$\alpha = .8$' mathematical equation inline
% '$$\hat{y} = b_0 + b_1x$$' mathematical equation in its own line
% '\begin{equation}...\end{equation}' multi-line equation
% '\approx' approximate symbol
% '\neq' not equal
% '\bar' mean bar over letter
% '\pm' plus minus sign 
% '^{}' superscript
% '_{}' subscript
% '\fraq{numerator}{denominator}' fraction
% '\sqrt[n]{}' square root
% '\sum_{k=1}^n' sum for 1 through n
%
% Insert things from elsewhere:
% '\input{filename}' inputs the raw (tex) file as a command (e.g., tables and R-Markdown imports)
% '\include{filename}' includes section on new page (incl. possible auxiliary info)
% '\includegraphics[settings]{filename}' add a figure or graph
% '\caption{}' adds a caption to a table or figure
% '\label{}' labels sections, tables, figures, etc. so that they can be referred to.
% '\ref{}' refer to a labelled sections, tables, figures, etc.
% '\begin{enumerate}...\end{enumerate}' numbered list
% '\begin{itemize}...\end{itemize}' bullet-ed list
% '\item' item in list section 
%
% Symbols:
% '\&' and sign
% '\%' percent sign
% '\_' three dotes
% '\#' hash symbol
% ------------------------------------------------------------------

\newlength{\mdfmar}
\setlength{\mdfmar}{1.5em}
\mdfdefinestyle{mdfbox}{
    innerleftmargin = +\mdfmar, %
    innerrightmargin = +2\mdfmar, 
    innertopmargin = +0\mdfmar, 
    innerbottommargin = +1.5\mdfmar, 
    skipabove = -12pt
}

\begin{mdframed}[style=mdfbox]
\noindent\center\textit{Argument Structure Introduction}:
\begin{itemize}[noitemsep]
    \item\ [Relevance]: increase in number and variety of ESM data
    \item\ [Problem]: no descriptive / understand structure analyses
    \item\ [Proposal]: feature based clusterting = flexible +  transparent
    \item\ [Section]: new type of ESM data and issues for analysis:
    \begin{itemize}[noitemsep]
        \item incomplete data (complete data models vs. structural missingness)
        \item non-equidistant
        \item non-stationary
        \item changes/trends on different time scales
        \item interest in multivariate developments
        \item dimensionality curse
    \end{itemize}
    \item\ [Section]: time-series clustering
    \begin{itemize}[noitemsep]
        \item shape-based
        \item model-based
        \item \textbf{feature-based}
    \end{itemize}
    \item\ [Section]: The present Research
\end{itemize}
\end{mdframed}

% Relevance Pragraph and issue
Recent years have seen a striking increase in the number and variety of research studies using experience sampling data \citep[][; also see \fgrref{fig:ScopusEsm}]{hamaker2017}. 
With the increased availability of technologies to easily collect large amounts of experience sampling data using mobile devices \citep[e.g.,][]{Keil2020} and web-based applications \citep[e.g.,][]{Arslan2020}, the experience-sampling method (ESM) has been applied to a host of new types of psychological data. One example of this phenomenon has been the cultural adaptation literature, where researchers have started using ESM data to follow the daily interactions of migrants with the local majority \citep[e.g.,][]{Kreienkamp2022b}. 
This move toward a broader application of the experience-sampling method, however, stands in stark contrast with recent developments within the psychometric literature on ESM analyses, which have become more specialized and model-focused. 
While generally speaking analytical methods for ESM data have become more readily available \citep[e.g.,][]{ODonnell2021}, flexible descriptive and exploratory analyses have largely been neglected and remain crucially understudied within the experience sampling literature. 

% Problem Illustration Paragraph
Importantly, the understudied `describe and explore' analyses are not only important for contextualizing inferential model tests, but are powerful methods for theoretical and applied users in their own regard. Extensive longitudinal data sets come with new forms of heterogeneity, where researchers have to consider differences across large numbers of participants, time points, and variables. Data-driven models not only describe and summarize this complex data but can help us understand the data by uncovering patterns that might not be detectable otherwise. Returning to the case of migration experiences, understanding how people differ in their migration trajectories can be crucial in understanding adaptive and maladaptive patterns. By identifying and contextualizing distinct trajectories, we might for example determine which situation- or person-specific differences discern positive from negative developments. Outcomes from such an approach might inform both practical early-warning signals as well as academic theory-building efforts. There is, thus, a clear need to assess the utility of data-driven/bottom-up analysis procedures that are flexible enough to address the growing variety of intensive longitudinal psychological data. 

% Aim / Solution / Proposal Paragraph:
In this article, we propose to look at recent developments within the wider time-series clustering literature to identify an exploratory data analysis that is flexible enough to apply to the growing variety of ESM data. In particular, we propose that feature-based clustering is ideally suited because it offers a flexible variety of validated and intuitive components while also allowing for adequate control by the researcher. To showcase the utility of this technique, we apply feature-based clustering to a set of three ESM studies following recent migrants in their daily interactions and relations with the cultural majority group. 

\section{Modern ESM Data}
Two important developments within the psychological sciences have been (1) a shift towards more comprehensive assessments of psychological phenomena and (2) a growing focus on intensive longitudinal real-life data. As an example, within the cultural adaptation literature prominent reviews have called for more longitudinal \citep[e.g.,][]{Ward2019} and real-world data \citep[e.g.,][]{McKeown2017}, while at the same time conceptual works have pushed for more comprehensive assessments of human experiences, including motivational, affective, cognitive, and behavioral aspects \citep[e.g.,][]{Kreienkamp2022d}. Such a multivariate, experience-sampling approach allows researchers to test theories more exhaustively and helps build a more embedded understanding of migration processes. However, such data also deviates from past intensive longitudinal data sets that have focused on psychological states that are more homogeneous in their developmental characteristics. In particular, the new applications of ESM tend to bring new challenges for data dimensionality, -missingness, and time scales. 

Concerning dimensionality issues, especially more abstract psychological experiences often need a wider variety of measurements to be captured adequately. Today, few clinical conditions are captured with a single symptom measure, emotions are rarely assessed in isolation, and as introduced earlier cultural adaptation has at least four psychological aspects. This also means that modern analysis techniques increasingly need be able to accommodate this increased focus on multivariate developments. The recent rise in psychological network models has highlighted this need for more multivariate outcome measurements (REFERENCE). Additionally, the increased number of variables also heightens the computational load for model estimations and especially when the number of variables and measurements compounds models may not converge --- an issue sometimes referred to as the dimensionality curse. A modern descriptive and exploratory analysis technique should consequently be able to summarize and structure the multivariate phenomena without running into computational load issues.

The push for more contextualized models also presents new issues with missing data. More event-based assessments tend to encounter issues when models require time intervals between measurements to be identical (i.e., equidistant measurements) and an increased focus on context-specific phenomena has lead to issues when data is only available under certain conditions, including follow-up questions (i.e., structurally missing data). Smaller issues of non-equidistant data can be avoided with transformations (e.g., dynamic time warping, REFERENCE) or newer modeling procedures (e.g., continuous-time models, REFERENCE) but it's prevalence is indicative of the structure of real-world issues researchers seek to investigate. Structural missingness remains a much more strenuous challenge. Many of the most commonly used models need complete data (including, ...) and structurally missing data cannot be imputed as it logically does not exist (REFERENCE). This type of missingness becomes a colossal issue when researchers are interested in how two things co-develop and one of the concepts has structural missingness under certain conditions. In the example of the migration experiences, researchers might for example be interested in how well-being develops in relations the quality of interactions with the majority group. Because rating of interaction quality are only possible when an interaction actually happened, one cannot impute interaction quality ratings for measurements that did not include an interaction. The most common practice for structurally missing data is to either exclude the variable or any measurement that has no structurally missing data (REFERENCE) --- neither option suits a research question that includes a developmental relationship between the two variables. In short, adequate descriptive approaches should be able to deal with non-equidistant and structurally missing data in order to address modern ESM data.

Finally, both the push for more multivariate conceptualizations and contextualized measurements have highlighted issues with how temporal developments are conceptualized. Firstly, researchers are increasingly interested in how mean levels of variables change over time (i.e., non-stationary), such trend data is often assumed to be non-linear, and with the increased focus on more diverse conceptualizations researchers are interested in relationships between variables that change on fundamentally different time scales. Many of the commonly used analyses for ESM data are stationary lagged regression models that assume stable means and variances over time (incl., vector autoregressive models, dynamic structural equation models, autoregressive integrated moving average models, and cross-lagged panel analyses) or basic trajectory models (e.g., mixed effects models, spline regression models, and latent growth curve modeling). And even fewer analysis techniques allow for substantially different time scales between variables. A modern descriptive and exploratory methodology should thus be able to flexibly capture and comparable non-stationary and non-linear trends, also across different time scales.

\section{Clustering}
One promising approach to describing and exploring time-series data has been the idea of structuring the data using clustering analyses. Clustering seeks to organize data into groups that are maximally different from one another while minimizing differences within the groups. Time-series clustering has been extremely common in other fields, including analyses of astronomical, meteorological, and aviation pathways, biological and medical developments, or energy and finance patterns \citep{Aghabozorgi2015}. But also for psychological data time-series clustering is not entirely novel \citep[e.g.,][]{ernst2021}. However, most clustering introduced for psychological data has faced the dimensionality, -missingness, and time scale issues described above.

Many of the common time-series clustering algorithms used today have been developed within the machine learning literature and have commonly been separated into three different approaches: (1) shape-based, (2) model-based, and (3) feature-based clustering \citep{hautamaki2008, liao2005}. Shape-based clustering analyses use the raw time series and seeks to match time series by non-linearly stretching and contracting the time axes. The other two approaches do not use the the raw time series directly but convert the data first. The model-based clustering has been the most-commonly used method within the psychological literature and clusters the parameters of a parametric model fitted for each participant \citep[e.g.,][]{ernst2021}. The flexibility of such clustering is restricted to the assumptions of the model used, so that stationary models, for example, do not capture non-stationary data well. Finally, feature-based clustering analyses are the most flexible, as they allow users to summarize the time series data with a wide variety of features that capture the data characteristics of interest. The participants are then clustered on basis of the extracted features. To showcase the flexibility of this approach, a commonly used software package for this approach allows users to extract a total of 794 time series features \citep[][]{christ2018}. Time series features can be chosen by the user based on their research question and can, for example, include the mean and standard deviation of a time-series, but also linear and non-linear trends, inertia related features such as auto-correlations, or the stability of the time series (e.g., mean absolute change). Given the flexibility of feature-based approach, users should be able to apply the analysis across multivariate data, feature extraction should mitigate issues of non-equidistant and structurally missing data, and the features can capture and compare across non-linear trends at different time scales.

\section{Feature-based Time Series Clustering}

\subsection{Variable selection}
field- and concept specific selection of input variables. Diverse 

\subsection{Data preparation?}
data cleaning + importantly, comparable time series:
\begin{itemize}[noitemsep]
    \item time-frame
    \item response scale (e.g., standardize within clusters)
\end{itemize}

\subsection{Feature extraction}
aim: 
hundreds of possible features to describe psychological time series. One approach to choosing relevant features would be to extract a large number of features and then assess which features are most effective at capturing differences in the time series. However, such an approach is not always advisable for psychological time series. For one, features should reduce the data dimensionality --- it would thus not necessarily be advisable to describe 60 ESM measurements with 180 time series features. More importantly, however, careful feature extraction can be crucial in the interpretability and explainability of the results. This is particularly the case when features have a conceptual/psychological meaning. Taking the concept of well-being as an example, psychologists might be interested in whether certain participants tend to have higher well-being in general (i.e., mean) or whether some participants fluctuate between extremely high and low well-being (i.e., variance). But psychologists might not necessarily be interested in the exact time point after which 50\% of the summed well-being values lie (i.e., relative mass quantile index) or how much different sine wave patterns within the well-being data correlate with one another (i.e., cross power spectral density). We would, thus, strongly advocate for a careful selection of time series features that are meaningful to the field and concepts.

Fortunately, past conceptual and empirical efforts offer valuable discussions of common time series features in psychological research. While the final selection of features should always be driven by the research questions and field-specific conventions, we can build a practical toolbox of meaningful time series features for psychological data. In particular, we propose to focus on six features based on common psychological research questions and recent works on affect dynamics \citep[e.g.,][]{dejonckheere2019, kuppens2017}. An overview of the features, their substantive interpretations and mathematical operationalizations is available in \tblref{tab:esmFeatures}.

The first two features are a person's \textit{central tendency} and \textit{variability}. Familiar statistics from the probability theory, the two features sit at the heart of many fundamental psychological questions. Are some people happier than others (i.e., difference in central tendency), and are people settled in their attitudes towards migrants or do they fluctuate over time (i.e., difference in variability)? Mathematically, both features have parametric and robust options to choose from (e.g., parametric: mean and standard deviation; robust: median and median absolute deviation). 

The third and fourth feature describe the structure of the variability within the time series. In particular, \textit{(in)stability} captures the average change between two consecutive measurements. Do changes tend to be slow and gradual or fast and abrupt? Mathematically, (in)stability measurements often look at the average change between two consecutive measurement. \textit{Inertia} describes how much a measurement carries over to future measurements. There are two main ways in which experiences tend to be connected to future experiences --- resistance to change and seasonality. Both forms of inertia relate to different types of research questions. Do patients stay in a depressed mood for several measurements (i.e., stability-inertia) and do participants drink more alcohol on Fridays (i.e., periodicity-inertia)? Both forms of inertia are often measured using autocorrelations with the either a lag-1 (stability-inertia) or the lag of the seasonality (e.g., seven days). Periodicity-inertia can additionally be captured using wave patterns (e.g., fourier or wavelet transformation).

The final two features are linear and nonlinear trends. \textit{Linear trends} are a common research interest for longitudinal data. Do patient symptoms improve or does worker productivity decline, are familiar types of research questions. Mathematically, such linear trends are commonly captured using (piecewise) linear regression coefficients. \textit{Nonlinear trends} are important in two regards. Firstly nonlinearity as a deviation from linear trends and secondly, the shape and structure of the nonlinear trend. Questions like ``Is the development of anxiety a nonlinear process?'' are mathematically captured using nonlinearity parameters such the bicoherence metrics \citep{cuddy2009}. The structure of nonlinear trends is often mathematically captured using polinomial coefficients or more broadly by how wiggly the line is (e.g., estimated degrees of freedom of GAM spline models; similar to number of spikes \citealp[]{caro-martin2018}). 

This selection of the proposed six time series features is in no way exhaustive or imperative. Both using a purely data-driven approach or selecting other aspects to summarize the time series are legitimate options. The list seeks to offer a practical toolbox of features that are common and meaningful to psychological research questions and -practice.

%\begin{table}%[hbt]
\begin{sidewaystable}
    \centering
    \caption{Examples of Features for Psychological Time Series.}
    \label{tab:esmFeatures} 
    \begin{tabular}{
    >{\raggedright\arraybackslash}p{0.15\linewidth} 
    >{\raggedright\arraybackslash}p{0.35\linewidth} 
    >{\raggedright\arraybackslash}p{0.25\linewidth} 
    >{\raggedright\arraybackslash}p{0.20\linewidth}
    }
        \hline 
        Time Series Feature & Substantive Interpretation & Example Operationalizations & Formulas, refs, ...? \\ 
        \hline \\ [-0.5em]
        Diversity \newline \hl{(drop this \& keep in var. selection only?)} & 
        Multivariate measurement of concepts (e.g., affect-behavior-cognition-desire, or bio-psycho-social) \linebreak & 
        -----\linebreak  & 
        {\centering --- ? ---\par} \\
        
        Central Tendency \linebreak & 
        Average level of the experience across the entire measurement period. \linebreak & 
        \vspace{-1em}
        \begin{itemize}[nosep,leftmargin=*,label={--}]
            \item mean
            \item median
            \item mode
        \end{itemize} \linebreak  & 
        {\centering --- ? ---\par} \\ 
        
        Variability & 
        Describes the average deviation from the central tendency across the entire measurement period. \linebreak & 
        \vspace{-1em}
        \begin{itemize}[nosep,leftmargin=*,label={--}]
            \item standard deviation
            \item variation coefficient
            \item median absolute deviation
        \end{itemize} \linebreak & 
        {\centering --- ? ---\par} \\ 
        
        (In)stability & 
        Describes the average change between two consecutive measurements of the experience. \linebreak & 
        \vspace{-1em}
        \begin{itemize}[nosep,leftmargin=*,label={--}]
            \item mean sum squared differences
            \item mean absolute change
            \item Ix instability index
        \end{itemize} \linebreak & 
        {\centering --- ? ---\par} \\ 
        
        Inertia & 
        Describes how much experiences carry over to the future measurements. This includes resistance to change (i.e., carries over to the next measurement) and periodic or seasonal returns (e.g., self-predictive on a daily or weekly basis). \linebreak &
        \vspace{-1em}
        \begin{itemize}[nosep,leftmargin=*,label={--}]
            \item autocorrelation (e.g., lag–1)
            \item fourier coefficients
            \item continuous wavelet transform
        \end{itemize} \linebreak & 
        {\centering --- ? ---\par} \\ 

        Linear Trend & 
        Describes upwards or downwards linear trend of the experience reports. \linebreak & 
        \vspace{-1em}
        \begin{itemize}[nosep,leftmargin=*,label={--}]
            \item OLS regression slope
            \item avg. piecewise linear reg. slope
        \end{itemize} \linebreak & 
        {\centering --- ? ---\par} \\ 
        
        Nonlinearity & 
        Describes the nonlinear structure of the time series. This includes measures that indicate the deviation from the a linear trend as well as nonlinear model parameters. \linebreak & 
        \vspace{-1em}
        \begin{itemize}[nosep,leftmargin=*,label={--}]
            \item GAM spline edf
            \item bicoherence metrics
            \item Langevin polinomial coefficient
        \end{itemize} \linebreak & 
        {\centering --- ? ---\par} \\ 
        
        \hline \\ [-0.75em]
        \multicolumn{4}{p{\linewidth}}{\footnotesize \textit{Note.} The presented features and operationalizations are neither exhaustive nor necessary for feature-based clustering.}
    \end{tabular}
\end{sidewaystable}


\subsection{Feature reduction}


\subsection{clustering}


\section{Empirical Implementation/Illustration}
% Methods and Results from RMarkdown render
Something about the migration experience literature should go here as
part of the illustration intro \ldots{}

\subsection{Data}

We used a data set following migration experiences collected by
\citet[][]{Kreienkamp2022b}. The data set consisted of three individual
studies that followed migrants who had recently arrived in the
Netherlands in their daily interactions with the Dutch majority group.
After a general migration-focused pre-questionnaire, participants were
invited twice per day to report on their (potential) interactions with
majority group members for at least 30 days. The short ESM surveys were
sent out at around lunch (12pm) and dinner time (7pm). After the 30 day
study period participants filled in a post-questionnaire that mirrored
the pre-questionnaire. Participants received either monetary
compensation or partial course credits based on the number of surveys
they completed. For our empirical example we focused on the variables
that were collected during the ESM surveys and were available for all
three studies. Full methodological details are available in the article
by \citet[][]{Kreienkamp2022b}.

\subsubsection{Sample and time frame (?)}

filter criteria
\citep[proportion of missing data and imputations:][]{Madley-Dowd2019}

Total N t. Total N ppt.

Study specific exclusions see Online Supplementary Material A.

\subsubsection{Variables}

We focus on \ldots{}

full methodological details are available in Online Supplemental
Material A, but basic item information, descriptives, and correlations
are available in \tblref{tab:??}

\subsection{Analysis and Results}

\subsubsection{Feature extraction}

\subsubsection{Feature reduction}

\subsubsection{Feature clustering}

feature engineering: feature extraction + feature selection

use domain knowledge to extract new variables from raw data (summarize)

options: tsfresh \citep[][]{christ2018}

\textit{k} features means \(2^k – 1\) possible models

maximize relevance and minimize redundancy



\section{Discussion}
% aims re-iterated
To be written.

\subsection{Limitations}
To be written.

\subsection{Implications}
To be written.


\subsection{Conclusion}
To be written.




\section{Notes}

\citep[][]{Madley-Dowd2019} [proportion of missing data and imputations]

\citep[e.g., see][]{Ram2014} [Processes Across Multiple Time-Scales]

"The goal of clustering is to identify structure in an unlabeled data set by objectively organizing data into homogeneous groups where the within-group-object dissimilarity is minimized and the between-group-object dissimilarity is maximized." --- \citep[][p.1857]{liao2005}

\section{Draft Graveyard (for scrapping only):}

Extensive longitudinal data sets come with new forms of heterogeneity, where we have to consider differences between people, over time, and across variables. Yet, past analytical advances have almost exclusively pushed for inferential modeling procedures\footnote{For example, stationary lagged regression models that assume stable means and variances over time (incl., vector autoregressive models, dynamic structural equation models, autoregressive integrated moving average models, and cross-lagged panel analyses) or basic trajectory models (e.g., mixed effects models, spline regression models, and latent growth curve modeling).}. And while inferential model testing is certainly important, we still miss discussions of methods for the more fundamental task of describing, summarizing, and understanding the raw/complex developmental data patterns.

As an example, a recent review of migration experiences has pointed out that despite many complex and dynamic theories, investigations of migrant adaptation have undervalued developmental data --- especially when it comes to the more internal experiences of motivations and emotions \citep[e.g.,][]{Kreienkamp2022d}. Yet, understanding how people differ in their migration trajectories, can be crucial in understanding adaptive and maladaptive patterns. To identify which variables are most important in the adaptation of migrants over time, we need methods to break down the data heterogeneity into its core components (in terms of important variables and developments) and we need ways to identify how these core components relate to key adaptation markers (including, well-being, intergroup anxiety, outgroup trust, or societal participation). There is, thus, a clear need to assess the utility of analysis procedures focused on the description and understanding of complex dynamical data. 

We are among the first to apply this analysis to experience sampling data and, to the best of our knowledge, we are the first to cluster social psychological trajectories. This stands in stark contrast, to a renewed recognition that social psychological phenomena unfold over time, a rapid increase of extensive longitudinal data collections, and a growing interest in understanding the co-development of multiple experience aspects \citep[e.g.,][]{Kreienkamp2022d, MacInnis2015, McKeown2017, Pettigrew2011, Ward2019}. 

To untangle the heterogeneity in real-life migrant adaptations, we will analyze data from three experience sampling studies, which followed the migration experiences of recent migrants to the Netherlands. All three studies focus on the psychological adaptation of migrants (i.e., psychological acculturation). However, given that past investigations of psychological acculturation have under-explored the crucial aspect of motivational experiences \citep[e.g.,][]{Kreienkamp2022d}, the three studies have placed a particular emphasis on the needs, goals, and motives of young migrants. To make full use of all three data sets and to guide future use of clustering procedures with extensive longitudinal psychological data, we will analyze the data in a descriptive and step-wise manner, where we consider the data jointly (and will then assess potential differences between data- and variable sets). 


ALTERNATIVE:

In the past, such data collections were often unfeasible because they were either physically impractical or too expensive. However, recent technological developments now allow us to easily collect large amounts of experience sampling data on mobile devices \citep[e.g.,][]{Keil2020} or using web-based applications \citep[e.g.,][]{Arslan2020}. And while generally speaking analytical methods for such, more complex, data have become more readily available \citep[e.g.,][]{ODonnell2021}, it remains unclear how we should identify key developmental patterns --- especially across multiple variables at the same time (i.e., multiple aspects). In essence, the new extensive longitudinal data comes with new forms of heterogeneity when jointly considering differences between people, over time, and across variables. Yet, past analytical advances have almost exclusively pushed for stationary lagged regression models\footnote{meaning models that assume stable means and variances over time; e.g., vector autoregression models, dynamic structural equation models, autoregressive integrated moving average models, and cross-lagged panel analyses} or basic trajectory models\footnote{e.g., mixed effects models, spline regression models, and latent growth curve modeling.}. And while these two approaches are important for inferential model testing, we still miss methods for the more fundamental task of describing and unraveling the developmental data patterns. 

Such descriptive methods are not merely important for hypothesis generation, but also for identifying and understanding adaptive and maladaptive patterns. We need to identify which variables are most important in the adaptation of migrants, we need methods to identify clusters of similar individuals and developments, and we need to assess whether such clusters differ in key adaptation markers (including, well-being, intergroup anxiety, outgroup trust, or societal participation). 


Problem: HOWEVER, ...

\begin{itemize}
    \item unclear how do deal with this more complex data (many variables, persons, and time points).
    \begin{itemize}
        \item many different variables could be important.
        \item how to consider them jointly (e.g., VAR, DSEM, ARIMA, cross-lagged panel analysis for testing model predictions, limited to specified lag and mean stationarity; Trajectory focused: mixed effects model, spline regressions, latent growth curve modeling, limited: often univariate outcome). 
    \end{itemize}
    \item \textbf{unclear how to identify core/important developments.} Especially across multiple variables at he same time (i.e., multiple aspects).
    \item unclear which variables, time scales, and methods are useful in practice.
    \item curse of dimensionality: clustering algorithms fail 
\end{itemize}


Solution:   

feature-based clustering

little data has thus far investigated the development of migration experiences and no research has assessed the co-development of multiple experience aspects. Clustering participants across variables and time-series features may allow to identify groups of similar developments and whether these clusters differ in key adaptation markers (including, well-being, anxiety, trust, or societal inclusion).




% Tables
% Example
%\input{Tables/descrFullWide}


% Figures

\begin{figure}
  \caption{Scopus ESM Development}
  \label{fig:ScopusEsm}
  \centering\includegraphics[width=\textwidth]{figures/Scopus-ESM-Development.png}
  \caption*{Note: \\
  Search Terms: `( "ecological momentary assessment"  OR  "experience sampling"  OR  "intensive longitudinal")' in Title, Abstract, or Keyword.}
\end{figure}

\begin{figure}
  \caption{Flowchart Feature-Based Time Series Clustering in Psychology}
  \label{fig:TSCFlow}
  \centering\includegraphics[width=\textwidth]{figures/TS Cluster Flow/TimeSeriesClusterFlowSelection.pdf}
  \caption*{Note: \\
  Choices selected for illustration in this manuscript are marked in bold.}
\end{figure}

\printbibliography

\appendix

\section{Placeholder Title}
\label{app:AppendixTitle}
To be written, if necessary.

\end{document}
