% define document type (i.e., template. Here: A4 APA manuscript with 12pt font)
\documentclass[man, 12pt, a4paper, mask, floatsintext]{apa7}

% change margins (e.g., for margin comments):
%\usepackage{geometry}
% \geometry{
% a4paper,
% marginparwidth=30mm,
% right=50mm,
%}

% add packages
\usepackage[american]{babel}
\usepackage[utf8]{inputenc}
\usepackage{csquotes}
\usepackage{hyperref}
\usepackage[style=apa, sortcites=true, sorting=nyt, backend=biber, natbib=true, uniquename=false, uniquelist=false, useprefix=true]{biblatex}
\usepackage{authblk}
\usepackage{graphicx}
\usepackage{setspace,caption}
\usepackage{subcaption}
\usepackage{enumitem}
\usepackage{lipsum}
\usepackage{soul}
\usepackage{xcolor}
\usepackage{fourier}
\usepackage{stackengine}
\usepackage{scalerel}
\usepackage{fontawesome5}
\usepackage[normalem]{ulem}
% \usepackage{longtable}
\usepackage{amsmath, nccmath}
\usepackage{mdframed}
\usepackage{ntheorem}
\usepackage{afterpage}
\usepackage{float}
\usepackage{array}
\usepackage{censor}
\usepackage{pdflscape}
\usepackage{lscape}
\usepackage{pdfpages}
\usepackage{enumitem}
\usepackage{caption}
\usepackage{adjustbox}
\usepackage{makecell}
\usepackage{tabu}


% make warning with red triangle
\newcommand\Warning[1][2ex]{%
  \renewcommand\stacktype{L}%
  \scaleto{\stackon[1.3pt]{\color{red}$\triangle$}{\tiny\bfseries !}}{#1}}%

% make question with red triangle
\newcommand\Question[1][2ex]{%
  \renewcommand\stacktype{L}%
  \scaleto{\stackon[1.3pt]{\color{red}$\triangle$}{\tiny\bfseries ?}}{#1}}%
  
% add definition sections
\theoremstyle{break}
\newtheorem{definition}{Definition}

% add hypothesis sections
\theoremstyle{plain}
\theoremseparator{:}
\newtheorem{hyp}{Hypothesis}

\newtheorem{subhyp}{Hypothesis}
   \renewcommand\thesubhyp{\thehyp\alph{subhyp}}

% add quote section
\usepackage{csquotes}

% framed box section
\usepackage{framed}
\emergencystretch=1em

% formatting links in the PDF file
\definecolor{myurlcolor}{HTML}{1a0dab}
\hypersetup{
pdfpagemode={UseOutlines},
bookmarksopen=true,
bookmarksopenlevel=0,
hypertexnames=false,
colorlinks   = true, %Colours links instead of ugly boxes
urlcolor     = myurlcolor, %blue,Colour for external hyperlinks
linkcolor    = black, %blue, Colour of internal links
citecolor   = black, % cyan, Colour of citations
pdfstartview={FitV},
unicode,
breaklinks=true,
}

% ref labels
\newcommand{\fgrref}[2][]{\hyperref[#2]{Figure \ref*{#2}#1}}
\newcommand{\tblref}[2][]{\hyperref[#2]{Table \ref*{#2}#1}}
\newcommand{\appref}[2][]{\hyperref[#2]{Appendix \ref*{#2}#1}}
\newcommand{\equatref}[2][]{\hyperref[#2]{Equation \ref*{#2}#1}}

% custom open science badge height
\newlength{\badgeheight}
\setlength{\badgeheight}{1em}

% prevent multipage footnotes
\interfootnotelinepenalty=10000

% language settings
\DeclareLanguageMapping{american}{american-apa}

% add reference library file
\addbibresource{referencesZotero.bib}

% Title and header
%\title{Describe and Explore: Using feature-based time series clustering to identify meaningful structures in intensive longitudinal data}
\title{A Gentle Introduction and Application of Feature-Based Clustering with Psychological Time Series}

\shorttitle{feature-based time series clustering}

% Authors
\author[*,1,3]{Jannis Kreienkamp}
\author[1,3]{Maximilian Agostini}
\author[1,3]{Kai Epstude}
\author[2,3]{Rei Monden}
\author[1,3]{Peter de Jonge}
\author[1,3]{Laura F. Bringmann}
\affiliation{\hfill}

\affil[1]{University of Groningen, Department of Psychology}
\affil[2]{Hiroshima University, Graduate School of Advanced Science and Engineering}
\affil[3]{Author order TBD}

\authornote{   
   \addORCIDlink{* Jannis Kreienkamp}{0000-0002-1831-5604}\\
   \addORCIDlink{Laura F. Bringmann}{0000-0002-8091-9935}\\
   \addORCIDlink{Kai Epstude}{0000-0001-9817-3847}\\
   \addORCIDlink{Maximilian Agostini}{0000-0001-6435-7621}\\
   \addORCIDlink{Peter de Jonge}{0000-0002-0866-6929}\\
   \addORCIDlink{Rei Monden}{0000-0003-1744-5447}

We have no known conflict of interest to declare. The authors received no specific funding for this work. Materials and  software is available at \url{https://janniscodes.github.io/migration-trajectories/}  \citep{KreienkampTBD}. Protocols, materials, data, and code are available at \url{https://osf.io/TBA} \citep{KreienkampTBD}. The preregistration of our analysis can be accessed as part of our Open Science Framework repository \citep{KreienkampTBD}.

Correspondence concerning this article should be addressed to Jannis Kreienkamp, Department of Psychology, University of Groningen, Grote Kruisstraat 2/1, 9712 TS Groningen (The Netherlands).  E-mail: \href{mailto:j.kreienkamp@rug.nl}{j.kreienkamp@rug.nl}
}

\leftheader{Kreienkamp}

% Abstract
\abstract{
Psychological time series data has not only become more common but also more diverse and complex. Researchers and practitioners are increasingly interested in identifying important differences in the developments of their patients, or participants. Past research has proposed a number of `dynamic measures' that describe meaningful developmental patterns for psychological data (e.g., instability, inertia, linear trend). Yet, commonly used clustering approaches are often not able to include these meaningful measures (e.g., due to model assumptions). We propose feature-based time series clustering as a flexible, transparent, and well-grounded approach that clusters participants based on the dynamic measures directly using common clustering algorithms. We introduce the approach and illustrate the utility of the method with real-world empirical data that highlight common ESM challenges of multivariate conceptualizations, structural missingness, and nonlinear trends. We use the data to showcase the main steps of input selection, feature extraction, feature reduction, feature clustering, and cluster evaluation. We also provide practical algorithm overviews and readily available code for data preparation, analysis, and interpretation.

}

\keywords{
    time series analysis, feature-based clustering, intensive longitudinal data, ESM\\
    \vspace{1em}
    \textit{Open Science Practices:}
    \href{https://osf.io}{\includegraphics[height=\badgeheight]{assets/open-badges-small/material-color.png}} Open Materials, 
    \href{https://osf.io}{\includegraphics[height=\badgeheight]{assets/open-badges-small/data-color.png}} Open Data,
    \href{https://osf.io}{\includegraphics[height=\badgeheight]{assets/open-badges-small/code-color.png}} Open Code, \break
    \href{https://osf.io}{\includegraphics[height=\badgeheight]{assets/open-badges-small/supplements-color.png}} Open Supplements
}


% set indentation size
\setlength\parindent{1.27cm}

% Start of the main document:
\begin{document}

% add title information (incl. title page and abstract)
\maketitle

% **CHEAT SHEET / LEGEND**

% Comments:
% '%' starts a comment in LaTeX (not printed)
% '\todo[inline]{} makes orange boxes in PDF
% '\marginpar{}' notes in margins
% '\footnote{}' footnote
% '\Warning' important note indicator in PDF (triangle with exclamation mark)
% '\Question' question note indicator in PDF (triangle with question mark)

% Citation (with Natbib citation style):
% '\citep[e.g.][p. 15]{CitationKey}' citation in parentheses "(e.g., Berry, 2003, p. 15)"
% '\citet{CitationKey}' citation in text "Berry (2003)"
% '\citealt' and '\citealp' alternate citation without parentheses
% '\citeauthor' and '\citeyear' only year or author

% Headings:
% '\part{}' and '\chapter{}' only relevant for multi-part or multi-chapter documents
% '\section{}' heading level 1
% '\subsection{}' heading level 2
% '\subsubsection{}' heading level 3
% '\paragraph{}' heading level 4
% '\subparagraph{}' heading level 5

% formatting:
% '\textbf{}' text bold font
% '\textit{}' text italic font
% '\underline{}' text underline
% '\sout{}' text strike out
% '\textsc{}' text small caps
% '\vspace{1em}' add vertical space
% '\hspace{1em}' add horizontal space
% '\\' new line (i.e., line break)
% '\pagebreak' start new page (i.e., page break)
% '\noindent' do not indent current line (e.g., current paragraph)
% 'begin{center}...end{center}' center text or object

% Math mode:
% '$\alpha = .8$' mathematical equation inline
% '$$\hat{y} = b_0 + b_1x$$' mathematical equation in its own line
% '\begin{equation}...\end{equation}' multi-line equation
% '\approx' approximate symbol
% '\neq' not equal
% '\bar' mean bar over letter
% '\pm' plus minus sign 
% '^{}' superscript
% '_{}' subscript
% '\fraq{numerator}{denominator}' fraction
% '\sqrt[n]{}' square root
% '\sum_{k=1}^n' sum for 1 through n

% Insert things from elsewhere:
% '\input{filename}' inputs the raw (tex) file as a command (e.g., tables and R-Markdown imports)
% '\include{filename}' includes section on new page (incl. possible auxiliary info)
% '\includegraphics[settings]{filename}' add a figure or graph
% '\caption{}' adds a caption to a table or figure
% '\label{}' labels sections, tables, figures, etc. so that they can be referred to.
% '\ref{}' refer to a labelled sections, tables, figures, etc.
% '\begin{enumerate}...\end{enumerate}' numbered list
% '\begin{itemize}...\end{itemize}' bullet-ed list
% '\item' item in list section 

% Symbols:
% '\&' and sign
% '\%' percent sign
% '\_' three dotes
% '\#' hash symbol
% ------------------------------------------------------------------

\newlength{\mdfmar}
\setlength{\mdfmar}{1.5em}
\mdfdefinestyle{mdfbox}{
    innerleftmargin = +\mdfmar, %
    innerrightmargin = +2\mdfmar, 
    innertopmargin = +0\mdfmar, 
    innerbottommargin = +1.5\mdfmar, 
    skipabove = -12pt
}

% \begin{mdframed}[style=mdfbox]
% \noindent\center\textit{Structure}:
% \begin{itemize}[nosep]
%     \item introduction
%         \begin{itemize}[nosep]
%             \item increase in number and variety of ESM data + ts clustering to understand complex developmental differences.
%             \item ESM literature: features to capture meaningful aspects
%             \item current clustering based on common model parameters, which only include a small selection of features and often have strong assumptions
%             \item Instead of building more complicated models (e.g., address violated assumptions or add different types of parameters), we propose to cluster based on important and relevant features directly.
%             \item Called feature-based clustering. Common procedure in digital phenotyping and the broader ml literature. Used in many fields
%             \item in this paper we provide a practical introduction to the method and illustrate its utility with real-world ESM data.
%         \end{itemize}
%     \item Data Used for Illustration
%     \item Analysis Steps + Application
%         \begin{itemize}[nosep]
%             \item input variables
%             \item feature extraction
%             \item (feature reduction)
%             \item feature clustering
%             \item cluster evaluation
%         \end{itemize}
%     \item Discussion (summary, limitation, implications, conclusion, etc.)
%     \newpage
%     \item\ [...]
%     \item Feature-based Time Series Clustering
%     \begin{itemize}[nosep]
%         \item why features: flexible similarity (b/c more variety), less assumptions, easy to calculate/extract, theory-based selection (as per RQ), psychological interpretability, ...
%         \item Features in Psychological Time Series
%         \begin{itemize}[nosep]
%             \item central tendency
%             \item variability
%             \item (in)stability
%             \item self-similarity
%             \item linear trend
%             \item nonlinearity
%         \end{itemize}
%         \item Analysis Steps
%         \begin{itemize}[nosep]
%             \item input variables
%             \item feature extraction
%             \item (feature reduction)
%             \item feature clustering
%             \item cluster evaluation
%         \end{itemize}
%     \end{itemize}
%     \item Empirical Illustration
%     \item Discussion (summary, limitation, implications, conclusion, etc.)
% \end{itemize}
% \vspace{1em}
% \end{mdframed}
% \newpage

Recent years have seen a striking increase in the number and variety of research studies that follow participants' everyday experiences and collect real-world psychological time series \citep[e.g.,][]{hamaker2017}. These intensive longitudinal data sets come with new forms of heterogeneity, where researchers have to consider differences across large numbers of participants, time points, and variables \citep[e.g.,][]{cattell1966, wardenaar2013}. Oftentimes, however, researchers are interested in precisely this complexity and wish to understand how people differ in their developments across several variables \citep[e.g.,][]{ernst2021}. Researchers and practitioners are, for example, asking: ``Do the symptoms of different patients develop in contrasting ways?'' \citep{Monden2015} or ``How do migrants differ in the development of their self-reported needs as they arrive in a new country?'' \citep{Kreienkamp2022d}. There is, thus, a clear need for analysis techniques that identify between-subject differences in developmental patterns for psychological data.

Recently, one promising way of identifying between-subject developmental patterns has been \textit{time series clustering} --- the idea of inductively grouping participants based on similarities of their time series \citep{ariens2020}. This analysis type essentially seeks to capture comparable within-person developments --- such as whether a variable remains stable over time, consistently increases, or exhibits cyclical patterns --- and then groups the persons based on these patterns  \citep[][]{liao2005}\footnote{It should be noted in some cases the time series do not need to be summarized and can be compared directly. Such analyses are however only possible with highly regular and controlled data, such as EEG data \citep{huang1985}, or when clustering variables within the person rather than identifying differences between persons \citep{haslbeck2022}. Multivariate data from intensive psychological survey studies, as we discuss here, are seldomly directly comparable across persons \citep[e.g., ][]{faloutsos1994}. We provide a broader embedding of the current methods for psychological time series in the discussion section.}. Time series clustering, thus, crucially depends on identifying meaningful summaries of the time series developments, which can be used to compare participants \citep[][]{Aghabozorgi2015}. 

Fortunately, past conceptual and empirical works in the experience sampling (ESM) literature have collected a number of meaningful aspects of psychological time series\footnote{In psychology, intensive longitudinal data collection methods are often referred to as experience sampling method (ESM), ecological momentary assessment (EMA), or ambulatory assessment (AA) studies. While the terms come from different conceptual backgrounds, they share a focus on collecting data over an extended period of time to capture people's behaviors and experiences as they vary over time and in response to different situations and events. In this article, we will use the experience sampling (ESM) term as it has the strongest footing within the clustering literature.}. Important aspects might include concerns over whether a symptom consistently stays at a certain level without much variability or whether some emotions develop together. For many of the most important developmental aspects, researchers have assembled numeric measures that capture these patterns. These summary statistics are often called ``dynamic measures'', ``principles of change'', or ``dynamic features'' of the psychological time series \citep{dejonckheere2019, kuppens2017, krone2018}. For most conceptualizations, the proposed number of aspects ranges from four to twelve key features \citep[][]{wang2006, dejonckheere2019}. Each of these statistics not only captures a distinct aspect of psychological time series but also holds conceptual value --- inertia, for example, describes a resistance to change that can be indicative of psychological maladjustment \citep{kuppens2010} or a higher within-person variability can signal an erratic state \citep{myin-germeys2018}.

Yet, despite this rich diversity of meaningful time series features in psychology, most clustering of ESM data has only focused on a small and restrictive selection of time series characteristics. Thus far, the most common approach has been to cluster participants based on person-specific model parameters --- notably intercepts and slopes from vector autoregression models \citep[VAR; e.g.,][]{ariens2020, bulteel2016, stefanovic2022}. While such model parameters are familiar to researchers in the field, they are often restricted to autocorrelations and cross-lagged partial correlations \citep[e.g.,][]{bringmann2018c} --- only two of the many potentially important time series features. Additionally, the validity of model parameters is traditionally restricted by the assumptions of the particular statistical model. To take the common VAR model as an example, the model explicitly assumes that the time intervals between measurements are consistent (i.e., equidistant measurement assumption), it does not allow for missing values, and it assumes that means or variances do not change over time \citep[i.e., stationarity assumption;][]{lutkepohl2005}. These assumptions however stand in contrast to types of data researchers commonly collect to address important questions of erratic, and context-specific phenomena \citep[][]{myin-germeys2018, hamaker2017, kivela2022, helmich2020a}. Model parameters might, thus, not always accurately capture the time series and are often restricted in the features they capture.

% There is, thus, a resolute need for more flexible and meaningful time series summaries that can capture diverse measurement and missingness patterns and can capture non-stationary data, including non-linear trends, also across different time scales.

%Recent efforts to address the shortcomings of model parameters for time series clustering have primarily proposed more complicated models aimed at addressing either accuracy or interpretability concerns. Accuracy concerns have particularly been addressed by building more advanced models that relax specific assumptions \citep[e.g.,][]{denteuling2021} or by switching to different types of models with fewer assumptions, including spline models \citep{axen2011}. To expand interpretability, researchers have proposed to model additional types of parameters, including variability or granularity parameters \citep[e.g., see][]{krone2018}. 
%\citet{krone2018} even built a parametric model to tentatively cluster study participants.

Recent efforts to address the shortcomings of model parameters for time series clustering have primarily proposed more complicated models, which either seek to relax specific assumptions \citep[e.g.,][]{denteuling2021} or include additional time series features \citep[e.g., see][]{krone2018, gates2017}. In this manuscript, we instead, propose to directly cluster based on important and relevant features. Adequately named \textit{feature-based time series clustering}, such an approach has been a common procedure in digital phenotyping \citep[][]{loftus2022} and the broader machine learning literature \citep[][]{maharaj2019}. As such, the analysis has been applied to a variety of data, including analyses of astronomical, meteorological, and aviation pathways, biological and medical developments, as well as energy and finance patterns \citep{Aghabozorgi2015}. We argue that for psychological time series data, feature-based clustering offers a flexible variety, fewer strict assumptions, easy and intuitive analysis, as well as meaningful psychological interpretability. 

In the sections below we particularly seek to provide a practical introduction to the method. To do so, we illustrate the utility of the method with real-world ESM data. We use this data to discuss which psychological time series features are well-suited for a clustering approach, introduce the individual analysis steps, and provide practical guidance on common algorithms and analysis code. We first introduce the empirical case study data and then follow the individual analysis steps to discuss key decisions and approaches.

\section{Data Used for Illustration}

To illustrate the functioning and utility of feature-based time series clustering with psychological ESM data, we apply the clustering process to a recent set of studies that collected data on migration experiences. Researchers have recently started using ESM data to follow the daily interactions of migrants with the cultural majority groups. We have seen both new technological advances to capture such interactions \citep[e.g.,][]{Keil2020} as well as a rise in empirical studies that look at aspects such as the well-being of migrants \citep[e.g.,][]{hendriks2016} and the distress of intergroup contact \citep[e.g.,][]{doucerain2023}. 

This type of research comes as a response to a long-standing theoretical tradition highlighting the dynamic and developmental nature of cultural adaptation \citep[e.g.,][]{berry1986}. Importantly, such developmental trajectories are often difficult and stressful \citep{Ward2016} --- so that some may have an adaptive trajectory while others face a more difficult and grievous trajectory \citep{kim2017}. Prominent reviews within the migration literature have, thus, called for more longitudinal \citep[e.g.,][]{Ward2019} and real-world data \citep[e.g.,][]{McKeown2017}. By the same token, there is a crucial need to distinguish (mal-)adaptive clusters within these developmental trajectories and to relate these clusters to individual differences and contextual variables to make them understandable \citep[e.g.,][]{choi2009}. 

At the same time, conceptual works on cultural adaptation have highlighted the multidimensional perspective necessary to understand migration experiences. A recent scoping review has particularly highlighted that cultural adaptation is best understood as a joint process of motivational, affective, cognitive, and behavioral aspects \citep[e.g.,][]{Kreienkamp2022d}. The data on migration experiences are, thus, explicitly multidimensional, mirroring the increase of complex data within the ESM literature \citep[][]{wardenaar2013}.

In short, the data on migration experience we are using for this illustration, address a pressing societal issue of identifying and understanding diverging trajectories. And importantly for our illustration, the migration ESM research, also, exemplifies the real-world data issues that ESM data commonly face, including a multivariate conceptualization with event-specific missingness patterns (also see \appref[]{app:ChallengesAppendix} for an expanded discussion of the current challenges within ESM data that are addressed within the data).

The data set we use consists of three studies that followed migrants who
had recently arrived in the Netherlands in their daily interactions with
the Dutch majority group members
\citep[for the data set see][]{Kreienkamp2022b}. After a general
migration-focused pre-questionnaire, participants were invited twice per
day to report on their (potential) interactions with majority group
members for at least 30 days. The short ESM surveys were sent out at
around lunch (12pm) and dinner time (7pm). After the 30 day study period
participants filled in a post-questionnaire that mirrored the
pre-questionnaire. Participants received either monetary compensation or
partial course credits based on the number of surveys they completed.

The original studies included 207 participants (\(N_{S1}=\) 23,
\(N_{S2}=\) 113, \(N_{S3}=\) 71) with a total of 10,297 ESM
measurements. Each of the studies focused on recently arrived
first-generation migrants and each study included a number of
ideosyncratic variables relevant for the broader research collective.
For our empirical example we focus on the variables that were collected
during the ESM surveys and were available in all three studies. Variable
selection and preparation is described as part of the illustration below
but for additional methodological details about the study setup see
\citet[][]{Kreienkamp2022b}.


\section{Analysis Steps and Application} 

To introduce and illustrate the feature-based clustering analysis, we will follow the conceptual steps of the procedure in sequential order and discuss key issues for each step. To do so we will follow the common separation that has structured feature-based clustering into four main steps \citep{rasanen2009, wang2006}. (1) The selection and preparation of the input variables, (2) the extraction of the features that describe the time series, with an optional feature reduction step if there are too many data points for the clustering algorithms, (3) the actual clustering of the time series features, and (4) the evaluation and interpretation of the clusters. 
While this is a common conceptual separation of procedural elements, it is important to note that these steps are a general outline, and the specific details of the analysis will depend on the nature of the data and the research question being addressed. Nonetheless, the conceptual nature of these steps allows us to introduce the major elements of the analysis. We also provide a conceptual overview that can be used alongside this section in \fgrref{fig:TSCFlow}.

\begin{figure}[!ht] % \begin{figure*}[hbtp] <-- on its own page
  \caption{Flowchart Feature-Based Time Series Clustering in Psychology}
  \label{fig:TSCFlow}
  \centering\includegraphics[width=\textwidth]{figures/TS Cluster Flow/TimeSeriesClusterFlowSelection.pdf}
  \caption*{Note: \\
  Choices selected for illustration in this manuscript are marked in bold.}
\end{figure}

\subsection{Input Variables}
Time series clustering starts with the selection and preparation of the variables of interest. While the selection will necessarily be field- and concept-specific, there are a few conceptual and methodological issues that should be considered. Conceptually, the included variables should adequately capture the concept of interest and should be meaningful to the understanding of the time series. One of the advantages of feature-based clustering is that it is inherently adept to accomodating multi-variate concepts --- a common aim in ESM research. There are, for example, calls that emotion dynamics should be assessed with a repertoire of positive and negative emotions \citep[e.g.,][]{dejonckheere2019}, many health developments are captured within the biopsychosocial domains \citep[e.g.,][]{suls2004}, and migration experiences are fully captured with affect, behavior, cognition, and desire measurements \citep[e.g.,][]{Kreienkamp2022d}. At the same time, however, the added number of variables can become a methodological concern. Not only can redundant and irrelevant variables diminish the quality of the analyses, but with intensive longitudinal data the number of data points compounds across participants, measurement occasions, and variables so that additional variables can make many of the following steps substantially more difficult (also see \fgrref{fig:TSCFlowN}). 

\begin{figure}[!ht] %hbtp
  \caption{Exemplary Flowchart of Data Points in Feature-Based Time Series Clustering}
  \label{fig:TSCFlowN}
  \centering\includegraphics[width=\textwidth]{figures/TS Cluster Flow/tsClustFlowN.pdf}
  \caption*{Note: \\
  The presented number of participants, variables, and measurement occasions are somewhat arbitrary but generally represent common sample sizes found within the literature. Also, the number of extracted clusters is presented for illustrative purposes only.}
\end{figure}

We included 12 variables that were available across all three studies
and captured information about the participant's interactions, as well
as cognitive-, emotional-, and motivational self in relationship with
the majority group. We chose these two aspects in particular because (1)
the interaction-specific information exemplified the structural
missingness issue of modern ESM data and (2) the motivational,
emotional, and cognitive experience offered a diverse conceptualization
of migration experience (beyond behavioral measurements) that is
becoming more common in the literature \citep[][]{Kreienkamp2022d}. Full
methodological details are available in Online Supplemental Material A,
but basic item information, descriptives, and correlations are also
available in \tblref{tab:descrLong}.

\begin{sidewaystable*}[!hbtp]
\centering

\caption{\label{tab:descrLong}Correlation Table and Descriptive Statistics}
\centering
\resizebox{\linewidth}{!}{
\begin{tabular}[t]{lcccccccccccc}
\toprule
  & \makecell{Int: \\ Accidental} & \makecell{Int: \\ Voluntary} & \makecell{Int: \\ Cooperative} & \makecell{Int: \\ Representative} & \makecell{Int: \\ Meaningful} & \makecell{Int: \\ Quality} & \makecell{Int: \\ Need Fulfil.} & \makecell{Int: \\ Need Fulfil. Partner} & \makecell{Attitude \\ Partner} & \makecell{Daytime \\ Core Need} & \makecell{Outgroup \\ Attitude} & Well-being\\
\midrule
\addlinespace[0.3em]
\multicolumn{13}{l}{\textbf{Correlations}}\\
\hspace{1em}Int: Accidental &  & -0.18 & -0.20 & 0.05 & -0.09 & -0.09 & -0.35*** & -0.23* & -0.04 & -0.26* & -0.01 & 0.18\\
\hspace{1em}Int: Voluntary & -0.15*** &  & 0.61*** & -0.06 & 0.09 & 0.40*** & 0.17 & 0.14 & 0.40*** & 0.09 & 0.23* & -0.05\\
\hspace{1em}Int: Cooperative & -0.14*** & 0.28*** &  & 0.17 & 0.37*** & 0.67*** & 0.42*** & 0.52*** & 0.45*** & 0.24* & 0.25** & -0.07\\
\hspace{1em}Int: Representative & 0.00 & 0.07** & 0.12*** &  & 0.09 & 0.13 & -0.08 & -0.06 & 0.14 & -0.02 & 0.41*** & -0.03\\
\hspace{1em}Int: Meaningful & -0.19*** & 0.21*** & 0.28*** & 0.00 &  & 0.65*** & 0.08 & 0.17 & 0.43*** & 0.08 & -0.03 & -0.05\\
\hspace{1em}Int: Quality & -0.09*** & 0.32*** & 0.39*** & 0.06** & 0.44*** &  & 0.41*** & 0.33*** & 0.63*** & 0.23* & 0.23* & 0.10\\
\hspace{1em}Int: Need Fulfillment & -0.08*** & 0.18*** & 0.27*** & 0.10*** & 0.17*** & 0.32*** &  & 0.65*** & 0.10 & 0.64*** & 0.15 & 0.31**\\
\hspace{1em}Int: Need Fulfillment Partner & -0.11*** & 0.20*** & 0.33*** & 0.08*** & 0.20*** & 0.32*** & 0.52*** &  & 0.13 & 0.52*** & 0.13 & 0.08\\
\hspace{1em}Attitude Partner & -0.04 & 0.30*** & 0.30*** & 0.03 & 0.41*** & 0.58*** & 0.23*** & 0.26*** &  & -0.10 & 0.56*** & 0.11\\
\hspace{1em}Daytime Need Fulfillment & -0.06** & 0.11*** & 0.16*** & 0.02 & 0.15*** & 0.16*** & 0.15*** & 0.14*** & 0.09*** &  & 0.07 & 0.15\\
\hspace{1em}Outgroup Attitude & -0.02 & 0.14*** & 0.16*** & 0.14*** & 0.20*** & 0.31*** & 0.19*** & 0.21*** & 0.37*** & 0.09*** &  & 0.25**\\
\hspace{1em}Well-being & -0.05* & 0.16*** & 0.16*** & -0.03 & 0.22*** & 0.32*** & 0.15*** & 0.13*** & 0.26*** & 0.19*** & 0.24*** & \\
\addlinespace[0.3em]
\multicolumn{13}{l}{\textbf{Descriptives}}\\
\hspace{1em}Grand Mean & 39.10 & 80.08 & 79.55 & 64.65 & 61.16 & 79.85 & 85.42 & 78.52 & 80.59 & 76.48 & 66.84 & 74.82\\
\hspace{1em}Between SD & 31.14 & 20.61 & 18.41 & 21.12 & 24.62 & 17.05 & 16.01 & 21.53 & 16.33 & 21.63 & 18.54 & 15.97\\
\hspace{1em}Within SD & 28.72 & 19.27 & 17.43 & 19.92 & 22.32 & 16.37 & 18.63 & 20.02 & 15.81 & 22.26 & 9.45 & 12.86\\
\hspace{1em}ICC(1) & 0.21 & 0.29 & 0.27 & 0.35 & 0.31 & 0.25 & 0.18 & 0.26 & 0.25 & 0.20 & 0.77 & 0.52\\
\hspace{1em}ICC(2) & 0.90 & 0.93 & 0.93 & 0.89 & 0.94 & 0.92 & 0.91 & 0.92 & 0.91 & 0.92 & 0.99 & 0.98\\
\bottomrule
\multicolumn{13}{l}{\rule{0pt}{1em}\textit{Note: }}\\
\multicolumn{13}{l}{\rule{0pt}{1em}"Int." = outgroup interaction, "ICC" = intraclass correlation coefficient, "SD" = standard deviation}\\
\multicolumn{13}{l}{\rule{0pt}{1em}Upper triangle: Between-person correlations;}\\
\multicolumn{13}{l}{\rule{0pt}{1em}Lower triangle: Within-person correlations;}\\
\multicolumn{13}{l}{\rule{0pt}{1em}*** p < .001, ** p < .01,  * p < .05}\\
\end{tabular}}
\end{sidewaystable*}



\subsection{Feature Extraction}
Armed with a relevant selection of key variables, the main aim of the feature extraction is to describe the most important and meaningful aspects of a time series. In its most general approach, feature extraction can include any numeric summary of the time series \citep[e.g.,][]{maharaj2019}. Given this flexibility, a staggering variety of time series features have been proposed across different disciplines. For example, \citet{wang2006} proposed 9 structural characteristics \citep[also see][]{fulcher2013}, \citet{adya2001} collected 28 features relevant for forecasting, and a commonly used software package for feature extraction `tsfresh' allows users to extract a total of 794 features of a time series \citep[][]{christ2018}. 

However, not all time series features might be relevant to psychological time series or any particular research question. For example, a psychologist interested in well-being might not necessarily be interested in the exact time point after which 50\% of the summed well-being values lie (i.e., relative mass quantile index) or how much different sine wave patterns within the well-being data correlate with one another (i.e., cross power spectral density). Instead, we advocate that we look at time series features that have a strong backing within the ESM literature and offer meaningful interpretability. 

Fortunately, past conceptual and empirical efforts offer valuable discussions of common time series features in psychological research. To understand emotion dynamics, \citet{kuppens2017} originally proposed four features: (1) within-person variability, (2) co-variance or ICC, (3) inertia or autocorrelation, and (4) cross-lagged correlations. These features were then extended by \citet{krone2018}, adding (5) innovation variance, and (6) mean intensity. \citet{krone2018} even built a parametric model to tentatively cluster study participants. From a slightly different perspective \citet{dejonckheere2019} later added three additional features for psychological time series: (7) instability (8) interdependence (i.e., network density), and (9) diversity \citep[i.e., Gini coefficient; also see][]{wendt2020}\footnote{It should be noted that also within the psychological literature, alternative summaries have been proposed that, for example, include measurement distribution, nonlinear developments, or categorical states. As an example, \citet{kiwuwa-muyingo2011} proposed to extract clinically meaningful states for medical adherence data and suggests these states as meaningful time series features.}. 

Some of the features found in the psychological literature are not necessarily well-suited to summarize time series for feature-based clustering and some key conceptual features are not well represented in the literature. In particular, covariances and cross-lagged correlations often produce a large number of parameters and might not necessarily summarize the existing data enough \citep{ernst2021}. Others such as network density parameters, used to summarize variable interdependence, might not always be meaningful for psychological data \citep{bringmann2019a}. At the same time, linear and nonlinear trends are not captured by the features commonly proposed for psychological time series, because the features are often developed for stationary models \citep[e.g.,][]{krone2018}. 

Thus, while the final selection of features should always be driven by the research questions and field-specific conventions, for our illustration we chose six features that relate to common psychological research questions and recent works within the field: (1) central tendency, (2) variability, (3) instability, (4) self-similarity, (5) linear trend, and (6) nonlinearity. An overview of the selected features, their substantive interpretations, and mathematical operationalizations is also available in \tblref{tab:esmFeatures}. For each of the six time series features we chose, we selected a mathematical representation that was appropriate for our type of data. We provide a short introduction of each feature below. Beyond the operationalizations we chose for our case study, we provide an R-function that automatically extracts and prepares a large selection of the time series feature operationalizations presented in \tblref{tab:esmFeatures} in our GitHub repository (see the \texttt{featureExtractor} function).

%\begin{table}%[hbt]
\begin{sidewaystable}
    \centering
    \caption{Examples of Features for Psychological Time Series.}
    \label{tab:esmFeatures} 
    \begin{tabular}{
    >{\raggedright\arraybackslash}p{0.15\linewidth} 
    >{\raggedright\arraybackslash}p{0.35\linewidth} 
    >{\raggedright\arraybackslash}p{0.25\linewidth} 
    >{\raggedright\arraybackslash}p{0.20\linewidth}
    }
        \hline 
        Time Series Feature & Substantive Interpretation & Example Operationalizations & Formulas, refs, ...? \\ 
        \hline \\ [-0.5em]
        Diversity \newline \hl{(drop this \& keep in var. selection only?)} & 
        Multivariate measurement of concepts (e.g., affect-behavior-cognition-desire, or bio-psycho-social) \linebreak & 
        -----\linebreak  & 
        {\centering --- ? ---\par} \\
        
        Central Tendency \linebreak & 
        Average level of the experience across the entire measurement period. \linebreak & 
        \vspace{-1em}
        \begin{itemize}[nosep,leftmargin=*,label={--}]
            \item mean
            \item median
            \item mode
        \end{itemize} \linebreak  & 
        {\centering --- ? ---\par} \\ 
        
        Variability & 
        Describes the average deviation from the central tendency across the entire measurement period. \linebreak & 
        \vspace{-1em}
        \begin{itemize}[nosep,leftmargin=*,label={--}]
            \item standard deviation
            \item variation coefficient
            \item median absolute deviation
        \end{itemize} \linebreak & 
        {\centering --- ? ---\par} \\ 
        
        (In)stability & 
        Describes the average change between two consecutive measurements of the experience. \linebreak & 
        \vspace{-1em}
        \begin{itemize}[nosep,leftmargin=*,label={--}]
            \item mean sum squared differences
            \item mean absolute change
            \item Ix instability index
        \end{itemize} \linebreak & 
        {\centering --- ? ---\par} \\ 
        
        Inertia & 
        Describes how much experiences carry over to the future measurements. This includes resistance to change (i.e., carries over to the next measurement) and periodic or seasonal returns (e.g., self-predictive on a daily or weekly basis). \linebreak &
        \vspace{-1em}
        \begin{itemize}[nosep,leftmargin=*,label={--}]
            \item autocorrelation (e.g., lag–1)
            \item fourier coefficients
            \item continuous wavelet transform
        \end{itemize} \linebreak & 
        {\centering --- ? ---\par} \\ 

        Linear Trend & 
        Describes upwards or downwards linear trend of the experience reports. \linebreak & 
        \vspace{-1em}
        \begin{itemize}[nosep,leftmargin=*,label={--}]
            \item OLS regression slope
            \item avg. piecewise linear reg. slope
        \end{itemize} \linebreak & 
        {\centering --- ? ---\par} \\ 
        
        Nonlinearity & 
        Describes the nonlinear structure of the time series. This includes measures that indicate the deviation from the a linear trend as well as nonlinear model parameters. \linebreak & 
        \vspace{-1em}
        \begin{itemize}[nosep,leftmargin=*,label={--}]
            \item GAM spline edf
            \item bicoherence metrics
            \item Langevin polinomial coefficient
        \end{itemize} \linebreak & 
        {\centering --- ? ---\par} \\ 
        
        \hline \\ [-0.75em]
        \multicolumn{4}{p{\linewidth}}{\footnotesize \textit{Note.} The presented features and operationalizations are neither exhaustive nor necessary for feature-based clustering.}
    \end{tabular}
\end{sidewaystable}


\paragraph{Central tendency.}

The central tendency refers to the statistical measures that represent
the ``typical'' or ``average'' of a set of data. The most common
measures of central tendency are the mean, median, and mode
\citep{weisberg1992}. As a familiar statistic from probability theory,
the central tendency sits at the heart of many fundamental questions
about psychological time series. Researchers might, for example, be
interested in whether ``Over a one-month period, are some people happier
than others?''

For the central tendency feature of our illustration, we chose the
median (\(M\)), which effectively addresses potential complications
arising from non-normally distributed responses or outliers within time
series datasets \citep{weisberg1992}. To compute the median, it is
imperative to differentiate between two types of time series
representations for a given variable \(j\) related to participant \(i\):
the chronological series and the ordered series. The chronological time
series, denoted by \(X_{ij}\), encapsulates the sequence of observations
\(\{x_{ij1}, x_{ij2}, ..., x_{ijn}\}\) for variable \(j\) concerning
participant \(i\), organized by their temporal occurrence. Here,
\(x_{ijt}\) signifies a specific observation at time \(t\) within this
sequence. In contrast, the ordered time series, represented as
\(\mathbf{X}_{ij}\), is derived from \(X_{ij}\) by sorting the
observations in ascending order of magnitude. This ordered set is
expressed as
\(\{\mathbf{x}_{ij1}, \mathbf{x}_{ij2}, ..., \mathbf{x}_{ijn}\}\), with
each \(\mathbf{x}_{ijk}\) corresponding to the \(k\)-th element in the
reordered series \(\mathbf{X}_{ij}\).

The median \(M(\mathbf{X}_{ij})\) is then the value located precisely at
the center of the ordered time series \(\mathbf{X}_{ij}\). Depending on
whether the total number of observations (\(n\)) is odd or even, the
median is either the middel \(k\)-th element if \(n\) is odd, or the
average of the two middle values if \(n\) is even:

\begin{equation} \label{eq:median}
  M(\mathbf{X}_{ij}) = 
  \begin{cases}
    \mathbf{x}_{ij\left(\frac{n+1}{2}\right)} & \text{if $n$ is odd} \\
    \frac{\mathbf{x}_{ij\left(\frac{n}{2}\right)} + \mathbf{x}_{ij\left(\frac{n}{2} + 1\right)}}{2} & \text{if $n$ is even}
  \end{cases}
\end{equation}

This approach ensures that the median is a reliable indicator of central
tendency in time series analysis, unaffected by data distribution
asymmetries or the presence of outliers.

\paragraph{Variability.}

Variability captures the degree to which a set of data differs from the
central tendency and is sometimes also referred to as the dispersion or
spread of the data \citep{weisberg1992}. Common measurements of the
variability are the variance or standard deviation as well as their
robust counter parts. In time series analyses, variability is
conceptually important because information about the distribution and
diversity of the data has been found to be indicative of worse
psychological states \citep{myin-germeys2018, helmich2021}. Person-level
differences of ESM measurements have, for example, been associated with
higher levels of psycho-pathological recurrences among depression
patients \citep{timm2017}. As such, psychological researchers and
practitioners are often empirically interested in between-person
differences in variability. Researchers on polarization and
radicalization might for example ask: ``Are people settled in their
attitudes towards migrants or do they vary across the measurement
period?''

For our illustration data, we chose the
\textit{Median Absolute Deviation} (\(MAD\)) to gauge the variability
within our time series data. This choice is motivated by the robustness
of MAD, particularly its resilience to the effects of non-normal
distributions and outliers, which can significantly skew traditional
variability measures like the standard deviation \citep{weisberg1992}.
For a given variable \(j\) and participant \(i\), the MAD is calculated
by first determining the median (\(M\)) of the ordered time series
\(\mathbf{X}_{ij}\) as outlined in \equatref{eq:median}. We then compute
the absolute deviations of each observation in the time series
\(X_{ij}\) from this median value. Specifically, for each time point
\(t\), we calculate the absolute difference between \(x_{ijt}\) and the
series median \(M(\mathbf{X}_{ij})\). The MAD is then the median of
these absolute deviations:

\begin{subequations}\label{eq:mad}
    \begin{align} 
      MAD(X_{ij}) &= M(\left| x_{ijt} - M(\mathbf{X}_{ij}) \right|) \label{eq:mad_general} \\
                  &= M(\{ |x_{ij1} - M(\mathbf{X}_{ij})|, |x_{ij2} - M(\mathbf{X}_{ij})|, \ldots, |x_{ijn} - M(\mathbf{X}_{ij})| \}) \label{eq:mad_detailed}
    \end{align}
\end{subequations}

The calculation of MAD focuses on the magnitudes of deviations, ensuring
it provides a robust measure of dispersion that reflects the inherent
variability in the time series data.

\paragraph{Instability.}

Instability captures the average change between two consecutive
measurements \citep{ebner-priemer2009, jahng2008}. While instability is
conceptually related to the variability feature, variability does not
take into account temporal dependency, whereas instability looks at the
`jumpy-ness' of the data over time. In other words, variability reflects
the range or diversity of values in the un-ordered time series data,
while instability reflects the fluctuation or inconsistency in a time
series data over time \citep{trull2008, houben2015, koval2013}. For
example, if a person has rapid and extreme changes in mood their mood is
highly unstable, while if a person's mood responses span a wide range
over the entire study period, their mood is highly variable
\citep[note that this does not need to be rapidly changing or instable, e.g., when there is linear increase over time; also see][]{jahng2008}.
Within psychological time series, instability measurements have
especially been important in the research of borderline personality
disorder \citep{trull2008} and suicidality \citep{kivela2022}, but also
in understanding early warning signals more generally
\citep{wichers2019}. Conceptually, the instability feature, thus,
relates to a broad range of research questions, including: ``What is the
nature of the identification changes in those who start working in a new
country?'' or ``Do strong daily fluctuations in self-esteem reflect the
process of identity formation in adolescents?''

For our data we chose the \textit{mean absolute change}
\citep[$MAC$; e.g.,][]{ebner-priemer2009, barandas2020}, which looks at
the average absolute difference of two consecutive measurements \(x\) at
time points \(t\) and \(t-1\), for each time series \(X\) of participant
\(i\) and variable \(j\).

\begin{equation} \label{eq:mac}
  MAC(X_{ij}) = \frac{1}{n-1} \sum_{t=2, \ldots, t}\left|x_{t}-x_{t-1}\right|
\end{equation}

Another common measurement of instability is the
\textit{Mean of the Squared Successive Differences} (\(MSSD\)), which is
often preferred where differences in magnitude are more important than
the frequency of those changes, for example, when big shifts in time
series are considered more impactful or when outliers are meaningful and
need to be taken into account \citep{chatfield2003, bos2019}. For
psychological ESM data, some research suggests that amplitude and
frequency might predict different health outcomes and can be
investigated jointly \citep{wang2012a, jahng2008}.

\paragraph{Temporal dependence.}

Temporal dependence in time series data refers to the degree to which a
time series is influenced by its past values, exhibiting patterns of
behavior that may be regular over different time scales
\citep{dmello2021}. In the context of psychological time series, an
important aspect of temporal dependence is \textit{inertia} --- how much
a measurement carries over to its next measurement
\citep{kuppens2010, suls1998}. If inertia is high a development tends to
stay in a certain state. Because high inertia is resistant to change, in
emotion dynamics high inertia of negative affect has been found to be
indicative of under-reactive systems and to be characteristic of
psychological maladjustment \citep{kuppens2010}. In a similar vein, high
inertia in negative affect at baseline was predictive of the initial
onset of depression \citep{kuppens2012}. Conceptually, inertia is more
broadly connected to research questions such as: ``Do patients stay in a
negative mood for several measurements?'' or ``Do migrants stay with
their language practice for several days at a time?''

For our illustration case, we chose the commonly used autocorrelation or
autoregression with a lag-1 to capture the inertia. High autocorrelation
values can indicate high levels of inertia, while low autocorrelation
values may indicate a more unpredictable or volatile time series
\citep{dejonckheere2019}. The lag--1 autocorrelation \(r_{ij,1}\) looks
at the average correlation between a measurement \(x\) and the preceding
measurement \(x_{t-1}\) for the time series \(X\) of participant \(i\)
and variable \(j\) with \(n\) measurements.

\begin{equation} \label{eq:ar}
  r_{ij,1} = \frac{\sum_{t=2}^{n}(x_{ijt}-\overline{x}_{ij})(x_{ij,t-1}-\overline{x}_{ij})}{\sum_{t=1}^{n}(x_{ijt}-\overline{x}_{ij})^2}
\end{equation}

Where \(\overline{x}_{ij}\) is the mean of the time series \(x_{ij}\),
calculated as:

\begin{equation} \label{eq:mean_for_ar1}
  \overline{x}_{ij} = \frac{1}{n} \sum_{t=1}^{n} x_{ijt}
\end{equation}

While inertia captures the simplest case of temporal dynamics, lag-1, we
acknowledge that temporal dependence in psychological time series may
also exhibit more complex relationships, including higher lagged auto
correlations or cyclical relationships (fourier coefficients, or
continuous wavelet transforms are often used to capture such
relationships).

\paragraph{Linear trend.}

In non-stationary time series, a linear trend can be observed when there
is a consistent increase or decrease in the data over time
\citep{nyblom1986}. For psychological time series, researchers have, for
example, pointed out the importance of linear trends in interpersonal
communications \citep{vasileiadou2014}, and emotion dynamics
\citep{oravecz2016}. Theoretically, linear trends are often considered
the simplest way of assessing whether a psychological theory of change
is appropriate \citep{gottman1969}. In empirical practice, linear trends
are, thus, commonly exemplified by research questions such as ``Do
patient symptoms improve consistently?'' or ``Does worker productivity
decline continuously?''

For the variables in our illustration data set, we chose an overall
linear regression slope to capture the linear trend. The regression
slope \(b_{ij}\) provides the average change from one time point \(t\)
to the next across all measurements \(x\) of a time series \(X\) of
participant \(i\) and variable \(j\). The specific form of the OLS slope
formula we provide below calculates \(b_{ij}\) as the sum across all
time points of the product of the deviation of time \(t\) from its mean
\(\overline{t}\) and the deviation of \(x_{ij}\) from its mean
\(\overline{x}_{ij}\) at each time point, divided by the sum across all
time points of the square of the deviation of time from its mean
(\(\sum(t-\overline{t})^2\)). Intuitively, the formula captures the rate
of change of variable \(x_{ij}\) with respect to time. This slope will
indicate how the variable \(x_{ij}\) changes over time, controlling for
its mean value and the mean of time. If the slope is positive,
\(x_{ij}\) increases over time; if it's negative, \(x_{ij}\) decreases
over time.

\begin{equation} \label{eq:lin}
  b_{ij} = \frac{\sum(t-\overline{t})(x_{ijt}-\overline{x}_{ij})}{\sum(t-\overline{t})^2}
\end{equation}

\paragraph{Nonlinearity.}

Changes in psychology are not always linear, instead, nonlinearity is a
common feature of psychological time series \citep{hayes2007}. As an
example, episodic disorders, such as depression, are often best
described as non-linear systems \citep{hosenfeld2015}. Similarly,
patients in recovery from depression showed sudden changes in the
improvement of depression \citep{helmich2020a}. But also substance abuse
\citep{boker1998} or attitude changes rarely develop linearly
\citep{vandermaas2003}. Conceptually, researchers might have research
questions about the type of the development: ``Is the development of
well-being a nonlinear process?'' as well as the shape and structure of
the development: ``How many spikes in well-being did a migrant
experience?''

We summarized the nonlinear trend with the
\textit{estimated degrees of freedom} of an empty GAM spline model. The
\(edf\) summarizes the \textit{wiggliness} of a spline trend line
\citep{wood2017, bringmann2017}. The degrees of freedom of a spline
model are primarily determined by the number of knots and the order of
the spline. For instance, a cubic spline with \(k\) knots has \(k\)+3
degrees of freedom
\citep{faraway2016, haslbeck2021a, castro-alvarez2024}. However, in a
penalized spline framework, which is commonly used for GAMs, the
effective degrees of freedom can be less than \(k\)+3. This is because
the model employs a smoothing parameter to control the trade-off between
the complexity (flexibility) of the model and its fit to the data,
thereby penalizing overly complex models and potentially reducing the
effective degrees of freedom \citep{marx1998}. Intuitively then an edf
of 1 would be equivalent to a linear relationship (i.e., one linear
slope parameter), whereas a higher edf (particularly an edf
\textgreater{} 2) is indicative of a non-linear trend. The estimated
degrees of freedom are commonly based on a concept called `effective
degrees of freedom' and can be represented as the trace \(tr\) (i.e.,
the sum of the diagonal elements) of the smoother matrix \(S\), a
symmetric matrix that maps from the raw data to the smooth estimates
\citep{wood2017}.

\begin{equation} \label{eq:df}
  edf = tr(S)
\end{equation}

\paragraph{Additional considerations.}

Beyond our main features of interest, we also extracted the
participant's number of completed ESM measurements to ensure that the
clusters are comparable in that regard (i.e., to exclude spurious
explanations for the cluster assignments). After the feature extraction,
we found that about 1.40\% of the extracted features are missing across
the 72 features per participant. This might, for example, happen if
participants do not have two subsequent measurements with outgroup
interactions, so that an autocorrelation with lag-1 cannot be calculated
for the contact-specific variables. The small number of missing values
indicates that the feature-based approach indeed largely avoids the
structural missingness issue. However, even the few missing values can
be an issue for some feature reduction or feature clustering algorithms.
We, thus, impute missing feature values via predictive mean matching
(PMM) with the MICE package in R, employing a single imputation and
specifying a maximum of 50 iterations and a fixed seed for convergence
and reproducibility \citep[][]{buuren2011}. We chose PMM for its ability
to preserve the original data distribution without assuming normality
and robustly handling multiple data types \citep{vanbuuren2006}. Note
again that with this procedure we only need to impute an extremely small
number of missing values as most feature calculations can use the
available data instead.


It is important to reiterate that the six selected time-series features are in no way exhaustive or imperative. Both using a more data-driven approach to the selection of features or selecting entirely different aspects to summarize the time series are legitimate options \citep[e.g., see][]{heylen2016}. Our choice seeks to offer a practical toolbox of features that are common and meaningful to psychological research questions and -practice but are also easy to extract and summarize a broad range of developments without asserting strict assumptions.

\subsection{Feature Reduction}
Once a meaningful selection of time series features has been extracted for each variable and participant, the total number of data points sometimes remains too large for the desired clustering algorithm. As an example, a relatively common scenario would include 10 variables of interest, where eight time series features are extracted, resulting in 80 features per participant (with a common sample size of 100 participants that would result in a total of 8,000 data points in this hypothetical example). We offer an illustration of the compounding numbers of data points in \fgrref{fig:TSCFlowN}. The difficulty of finding stable clusters for data with a large number of dimensions is sometimes termed the `dimensionality curse' \citep[e.g.,][]{altman2018}. 

To deal with this dimensionality issue, two main approaches have been proposed --- feature selection and feature projection \citep[e.g.,][]{erdogmus2008}. While feature selection refers to the process of identifying and selecting a subset of relevant features from the original feature set \citep{alelyani2014}, feature projection refers to the process of transforming the original feature set into a new feature set of lower dimensionality \citep[][]{carreira-perpinan1997}. In general, feature selection procedures have the benefit that they retain the interpretable feature labels directly and immediately indicate which features were most informative in the sample. Feature projection methods, on the other hand, have been popular because they are efficient, widely available, and applicable to a wide range of data types. We provide an overview of the common approaches, an intuitive introduction to common methods, and exemplar algorithms in \tblref{tab:featureReduction}. 

%Feature selections seek to create a subset of the most important features. The many available approaches differ in how they seek to determine the importance of the individual features \citep{alelyani2014}. Generally speaking, selection methods can be categorized as `filter', `wrapper', or `hybrid' methods \citep{hoque2014}. The filter methods, broadly speaking determine important features by identifying irrelevant features (e.g., because features do not capture much information), and identifying redundant features \citep[e.g., because features capture the same information; e.g., see][]{yu2004}. Wrapper methods, on the other hand, avoid the feature-based evaluation and focus on the performance of the later model to identify the most important features \citep{kohavi1997}. A wrapper, thus, compares the performance of models with different feature combinations \citep[e.g., see][]{tang2014}. Traditional examples of wrappers are forward selection or backward elimination procedures. Because filter methods might not always perform well and wrapper methods are computationally intensive\footnote{Computationally \textit{k} features could be considered in \(2^k – 1\) possible combinations --- for the example of 80 features that would allow for \(1.20 \times 10^{24}\) (over one septillion) combinations.}, hybrid methods seek to combine the two methods and find a balance between computational effort (i.e., efficiency) and performance \citep[i.e., effectiveness; e.g.,][]{alelyani2014}. Methods might, for example, use a filter step to reduce the size of features considered in a later wrapper step \citep{hsu2011}. Feature selection procedures have the benefit that they retain the interpretable feature labels directly and immediately indicate which features were most informative in the sample.

%The second general approach to the dimensionality curse has been feature projection. Broadly speaking, projection methods seek to transform the many features in such a way that a much smaller number of new variables can accurately capture the variance and structure of the original features \citep[i.e., the data is projected to a lower dimensional space; e.g.,][]{carreira-perpinan1997}. Projection methods are relatively common in psychological research --- including, factor- and principal component analyses \citep{vandermaaten2009}. Commonly, this is achieved using linear transformations \citep[e.g., principal component analysis; see][]{cunningham2015}, or more complex nonlinear transformations \citep[e.g., t-SNE; see]{lee2007}. Feature projection methods have been popular because of their widespread availability as well as their high efficiency.

%\begin{table}%[hbt]
\begin{sidewaystable}
    \centering
    \caption{Examples of Feature Reduction Approaches and Methods.}
    \label{tab:featureReduction} 
    \begin{tabular}{
    >{\raggedright\arraybackslash}p{0.08\linewidth} 
    >{\raggedright\arraybackslash}p{0.08\linewidth} 
    >{\raggedright\arraybackslash}p{0.54\linewidth} 
    >{\raggedright\arraybackslash}p{0.25\linewidth}
    }
        \hline 
        Approach & Method & Intuitive Description & Algorithm Examples \\ 
        \hline \\ [-0.5em]
        
        Selection \linebreak & 
        filter \linebreak & 
        Features are selected individually or jointly based selection criteria. One common selection criteria is the amount of information and (unique) variance a feature captures. Univariate methodologies are able identify irrelevant features (i.e., because features do not capture much information), multivariate methods additionally allow to remove redundant features (i.e., because features capture the same information). \linebreak &
        \vspace{-1em}
        \begin{itemize}[nosep,leftmargin=*,label={--}]
            \item univariate filter (e.g., Laplacian Score)
            \item multivariate filter (e.g., variance-covariance)
        \end{itemize}
         \linebreak \\ 
        
        \linebreak & 
        Wrapper \linebreak & 
        Wrapper methodologies run models with different feature combinations and compare performance. Because the selection process is essentially a search problem this method is computationally intensive. Traditionally wrappers have used forward selections or backwards eliminations but recently alternative approaches have been proposed based such as ant colony and swarm intelligence paradigms. \linebreak &
        \vspace{-1em}
        \begin{itemize}[nosep,leftmargin=*,label={--}]
            \item sequential (e.g., forward selection)
            \item bio-inspired (e.g., ELSA)
            \item iterative (e.g., feature salience)
        \end{itemize}
         \linebreak \\ 
        
        \linebreak & 
        hybrid \linebreak & 
        Hybrid selection methodological combine filter and wrapper methodologies to avoid the shortcomings of the individual methods. An initial filter step reduces computational effort (efficiency) and the wrapper ensures high performance (effectiveness). \linebreak &
        \vspace{-1em}
        \begin{itemize}[nosep,leftmargin=*,label={--}]
            \item filter + wrapper (e.g., Calinski-Harabasz Index)
        \end{itemize}
        \linebreak \\

        Projection \linebreak & 
        linear \linebreak & 
        Linear dimensionality reduction methodologies use linear transformations of the original data to stretch and shift the data in such a way that the data can be `projected' to a lower dimensional space without loosing too much information. These methods are well-established, tend to be fast, and usually do not need much conceptual input from the user. \linebreak &
        \vspace{-1em}
        \begin{itemize}[nosep,leftmargin=*,label={--}]
            \item Principal Component Analysis (PCA)
            \item Factor Analysis (FA)
            \item Linear Discriminant Analysis (LDA)
        \end{itemize}
         \linebreak \\
        
        \linebreak & 
        nonlinear \linebreak & 
        Nonlinear dimensionality reduction methods also seek to map high-dimensional data to a lower dimensional space. However, nonlinear methods have been developed to preserve the local and global structure of more complex multidimensional patterns.
        \linebreak &
        \vspace{-1em}
        \begin{itemize}[nosep,leftmargin=*,label={--}]
            \item t-distributed Stochastic Neighbor Embedding (t-SNE)
            \item Multidimensional Scaling (MDS)
            \item Isometric mapping (Isomap)
        \end{itemize}
        \linebreak \\
        
        \hline \\ [-0.75em]
        \multicolumn{4}{p{\linewidth}}{\footnotesize \textit{Note.} The presented dimensionality reduction methods and -approaches are neither exhaustive nor necessary for feature-based clustering.}
    \end{tabular}
\end{sidewaystable}


For the feature reduction, we chose the common
\textit{principal component analysis} (PCA). Some of the more
tailor-made feature selection algorithms can be more accurate in
reducing the feature dimensionality and might retain feature importance
information more directly, depending on the specific data structure.
However, PCAs have the distinct benefit that they are well-established
within the psychometric literature and can broadly be applied to a wide
variety of studies in an automatize manner. As our aim is to present a
general illustration that can also be adopted for more general data
descriptive uses, we present the workflow using a PCA here but we
encourage users to consider more specialized methods as well.

To use the principal component analysis with our extracted time series
features, we first standardize all features across participants to
ensure that all features are weighted equally. We then enter all 72
features into the analysis. The PCA will use linear transformations in
such a way that the first component captures the most possible variance
of the original data (e.g., by finding a vector that maximizes the sum
of squared distances). The following components will then use the same
method to iteratively explain the most of the remaining variance while
also ensuring that the components are linearly uncorrelated. In
practice, this meant that the PCA decomposed the 72 features into 72
principal components but now (because of the uncorrelated linear
transformations) the first few principal components will capture a
majority of the variance. We can then decide how much information (i.e.,
variance) we are willing to sacrifice for a reduced dimensionality. A
common rule of thumb is to use the principal components that jointly
explain 70--90\% of the original variance (i.e., cumulative percentage
explained variance). For our illustration we select the first 27
principal components that explain 80\% of the variance in the original
72 features (reducing the dimensionality by 62.50\%). For the extracted
principal components we save the 27 PC-scores for each participant
(i.e., the participants' coordinates in the reduced dimensional space).

We would like to comment on two practical matters when using principal
components --- the amount of dimensionality reduction and the
interpretation of the principal components. As for the expected
dimensionality reduction, given its methodology, PCAs tend to `work
better' at reducing dimensions with (highly) correlated variables. Thus,
with a set of very homogeneous variables and features users will need
less principal components to explain a large amount of variance, while a
more diverse set of variables and features will tend to require more
principal components to capture the same amount of variance. Our 27
principal components are still a relatively high number of variables but
this is not surprising as we chose a diverse conceptualization and a
diverse set of time series features. As for the interpretability, PCA
allows users to extract information on the meaning of the principal
components. In particular, because the principal components are linear
combination of original features, users can extract the relative
importance of each feature for the extracted principal components (i.e.,
the eigenvectors). While this can be useful in understanding the
variance in the original data or help with manual feature selection, we
use the PCA purely to reduce the dimensionality for the clustering step.
Instead of relying on the principal components, we use the original
features of interest to interpret the later extracted clusters. We
particularly advocate for such an approach if the original features were
chosen for their psychological meaning in understanding the time series
and broader phenomenon of interest.


\subsection{Feature Clustering}
For the actual clustering of the time-series features, the main aim is to organize participants into groups so that the features of participants within a group are as similar as possible, while the features of people in different groups are as different as possible \citep{liao2005}. The crux of clustering is, thus, to have clearly defined and effective measurements of (dis)similarity. Most of the clustering algorithms used today use some form of distance measurement to optimize group assignment \citep[or similarity measurement for qualitative features; see][]{Aghabozorgi2015}. While others have produced excellent overviews of the many clustering approaches available \citep[e.g.,][]{xu2015}, the more readily available approaches suitable for most time series feature data can, broadly speaking, be categorized as based on (1) centroids, (2) distributions, (3) density, (4) hierarchies, or (5) a combination thereof \citep[see \tblref{tab:clusterApproaches} for an overview; also see][for a broader review]{jain1999}. 

There is, unfortunately, no one-size-fits-all solution to clustering and users will usually have to make an informed decision based on the structure of their data as well as an appropriate weighing of accuracy and efficiency. We provide a short intuitive explanation for common approaches, together with some of their characteristics and example algorithms in \tblref{tab:clusterApproaches}. For our own illustration, we have chosen the centroid-based k-means clustering. While k-means comes at the expense of high accuracy with more complex cluster shapes, we specifically chose k-means because it is an extremely efficient method that works well with large participant- and feature numbers without making too many restrictive assumptions about the shape of the clusters \citep{jain2010}. K-means is also well-established within the research community and has been readily implemented in many statistical software packages \citep{hand2005}. Additionally, many of the feature selection methods have specifically been designed for the well-established k-means algorithm \citep[e.g.,][]{boutsidis2010}. As such, the k-means offers a good starting point for many psychological researchers and the method should be generalizable across a relatively wide variety of projects.

%Each of these approaches can be a valuable clustering approach for time series feature data and users will usually have to make an informed decision based on the structure of their data as well as an appropriate weighing of accuracy and efficiency. As an example, under ideal conditions, most approaches are likely to provide a similar cluster solution \citep[e.g., for well-separated groups with little noise and few outliers; e.g., see][]{peng2022}. However, when the shapes of clusters, for example, become more complex in a multi-dimensional space, density-based or hierarchy-based approaches that allow for more bottom-up clustering are likely to be more accurate \citep[e.g.,][]{langfelder2008}. Yet, with higher numbers of participants and features, both density- and hierarchy-based approaches may perform less well and the more efficient centroid-based methods might be more effective \citep[e.g.,][]{jain2010}. Similarly, several of the proposed time series features are statics that are generated from Gaussian distributions and a distribution-based algorithm might be ideally suited to separate such distributions \citep[e.g.,][]{corduas2008}. Yet, in cases where misspecifications are costly, a method with fewer assumptions might be more advisable \citep[e.g.,][]{ankerst1999}.

%\begin{table}%[hbt]
\begin{sidewaystable}
    \centering
    \caption{Common Clustering Approaches.}
    \label{tab:clusterApproaches} 
    \begin{tabular}{
    >{\raggedright\arraybackslash}p{0.15\linewidth} 
    >{\raggedright\arraybackslash}p{0.40\linewidth} 
    >{\raggedright\arraybackslash}p{0.28\linewidth} 
    >{\raggedright\arraybackslash}p{0.12\linewidth}
    }
        \hline 
        Approach & Description & Characteristics & Examples \\ 
        \hline \\ [-0.5em]
        
        centroid-based \linebreak & 
        Chooses a pre-defined number of potential cluster centers in the feature space and assigns participants to closest center. Then, iteratively, moves the centers until a convergence criterion is met (e.g., all distances to centers minimized).
        \linebreak &
        \vspace{-1em}
        \begin{itemize}[nosep,leftmargin=*,label={--}]
            \item[\scriptsize\faPlusCircle] simple and efficient
            \item[\scriptsize\faPlusCircle] no assumptions
            \item[\scriptsize\faPlusCircle] well implemented
            \item[\scriptsize\faMinusCircle] may struggle with complex shapes
            \item[\scriptsize\faMinusCircle] sensitive to initial values and outliers
            %\item[\scriptsize\faMinusCircle] not suitable for non-convex data, relatively sensitive to the outliers, easily drawn into local optimal, the number of clusters needed to be preset, and the clustering result sensitive to the number of clusters \citep{xu2015}.
        \end{itemize}\linebreak & 
        \vspace{-1em}
        \begin{itemize}[nosep,leftmargin=*,label={--}]
            \item k-means
            %\item fuzzy c-means
            %\item CLARA
            \item PAM
        \end{itemize}\linebreak \\ 
        
        distribution-based \linebreak & 
        Assumes that the data points belong to one of several specific distributions (e.g., Gaussian distributions). Data points that fit to a distribution-based expectation are given higher probability of belonging to that distribution. We can then iteratively check with which model parameters the data points best fit within given number of distributions.
        \linebreak &
        \vspace{-1em}
        \begin{itemize}[nosep,leftmargin=*,label={--}]
            \item[\scriptsize\faPlusCircle] probabilistic
            \item[\scriptsize\faPlusCircle] well supported
            \item[\scriptsize\faMinusCircle] time intensive
            \item[\scriptsize\faMinusCircle] distribution and parameter sensitive
        \end{itemize}\linebreak & 
        \vspace{-1em}
        \begin{itemize}[nosep,leftmargin=*,label={--}]
            \item GMM
            %\item negative binomial model-based
            \item DBCLASD
        \end{itemize}\linebreak\\ 
        
        density-based \linebreak & 
        Regions where several data points are relatively close together (i.e., high density) are assumed to be a cluster. Based on what is considered a dense region (e.g., radius of a region and minimum number of points the radius), points can be either be assigned to one of the clusters or be considered too far away from a dense region. \linebreak &
        \vspace{-1em}
        \begin{itemize}[nosep,leftmargin=*,label={--}]
            \item[\scriptsize\faPlusCircle] efficient
            \item[\scriptsize\faPlusCircle] no shape assumption
            \item[\scriptsize\faPlusCircle] do not assign outliers 
            \item[\scriptsize\faMinusCircle] may struggle with uneven densities
            \item[\scriptsize\faMinusCircle] sensitive to high dimensionality
        \end{itemize}\linebreak & 
        \vspace{-1em}
        \begin{itemize}[nosep,leftmargin=*,label={--}]
            \item DBSCAN
            \item OPTICS
        \end{itemize}\linebreak\\ 

        %graph-based \linebreak & 
        %``realized on the graph where the node is regarded as the data point and the edge is regarded as the relationship among data points'' \citep{xu2015} \linebreak &
        %\vspace{-1em}
        %\begin{itemize}[nosep,leftmargin=*,label={--}]
        %    \item[\scriptsize\faPlusCircle] efficient and accurate
        %    \item[\scriptsize\faPlusCircle] no shape assumption
        %    \item[\scriptsize\faMinusCircle] sensitive to graph complexity
        %\end{itemize}\linebreak & 
        %\vspace{-1em}
        %\begin{itemize}[nosep,leftmargin=*,label={--}]
        %    \item MST-based
        %    \item CLICK
        %\end{itemize}\linebreak\\ 

        hierarchy-based \linebreak & 
        Builds a hierarchy of cluster by step-wise combining the closest two clusters (bottom-up; agglomerative) or top down dividing the data into smaller clusters that maximize distances (top-down; divisive).
        \linebreak &
        \vspace{-1em}
        \begin{itemize}[nosep,leftmargin=*,label={--}]
            \item[\scriptsize\faPlusCircle] flexible number of clusters
            \item[\scriptsize\faPlusCircle] no shape assumption 
            \item[\scriptsize\faMinusCircle] small number of cases
            \item[\scriptsize\faMinusCircle] no reversal of assignments
        \end{itemize}\linebreak & 
        \vspace{-1em}
        \begin{itemize}[nosep,leftmargin=*,label={--}]
            \item Chameleon
            \item CURE
        \end{itemize}\linebreak \\ 
        
        hybrid \linebreak & 
        Usually combines different approaches, which combine the strength of the complementary approaches. Oftentimes the combination also increases efficiency. \linebreak &
        \vspace{-1em}
        \begin{itemize}[nosep,leftmargin=*,label={--}]
            \item[\scriptsize\faPlusCircle] avoids individual shortcomings
            \item[\scriptsize\faMinusCircle] less readily available
        \end{itemize}\linebreak & 
        \vspace{-1em}
        \begin{itemize}[nosep,leftmargin=*,label={--}]
            \item DD-means
            \item hybridHclust
        \end{itemize}\linebreak\\ 
        
        \hline \\ [-0.75em]
        \multicolumn{4}{p{\linewidth}}{\footnotesize \textit{Note.} The presented clustering approaches and algorithms are neither exhaustive nor necessary for feature-based clustering. Notably, recently innovations have been made based on graph-, fractal-, swarm, and quantum theory \citep[for a more in-depth review see][]{xu2015}.}
    \end{tabular}
\end{sidewaystable}


During the k-means clustering itself, the analysis seeks to minimize the
total within-cluster variation. The analysis is designed to optimize the
clustering of the feature data into \(k\) groups, where \(k\) is a
pre-defined number of clusters. We used the Hartigan and Wong algorithm,
which is a widely used algorithm in k-means clustering
\citep{hartigan1979}. The algorithm starts by randomly separating the
data points into k clusters and then iteratively updates the assignment
of each point to the nearest cluster center until convergence. To do so,
the Hartigan and Wong algorithm specifically calculates the
within-cluster variation (\(W\)) of cluster \(C_i\) as the summed
squared Euclidean distances of the feature \(x\) to the closest cluster
centroid \(\mu_i\):

\begin{equation} \label{eq:kWCi}
  W(C_i) = \sum_{x \in C_i}(x-\mu_i)^2
\end{equation}

By summing the within-cluster sum of squares from all \(k\) clusters, we
can then derive the total within-cluster sum of square \(WCSS\):

\begin{equation} \label{eq:kWCSS}
  WCSS = \sum_{i=1}^k W(C_i) = \sum_{i=1}^k \sum_{x \in C_i} (x - \mu_i)^2
\end{equation}

It is this \(WCSS\) that becomes the objective function to be minimized,
by iteratively moving features from one cluster to another
\citep{hartigan1979}. In particular, the algorithm (1) calculates the
cluster centroids of the initial partitioning, (2) checks whether any
feature has a centroid that is closer than that of the currently
assigned cluster (3) updates the centroids based on any reassigned
features, and then iterates between steps two and three until \(WCSS\)
is minimized (i.e., locally optimal convergence) or a maximum number of
iterations is reached. Given the iterative nature of the algorithm, the
initial partitioning is often important because the algorithm might
arrive at a suboptimal clustering where the \(WCSS\) cannot be further
reduced by moving any feature to another cluster, despite a better
solution existing (i.e., a local minimum). It is, therefore often
recommended to run the k-means clustering with several different
starting positions.

In our case we used the Elbow and the Silhouette method to determine the
recommended number of \(k\) clusters, which for both methods was two
clusters (see Supplemental Material A). We then entered the
participants' PC-scores from the feature reduction step into the k-means
algorithm. To avoid local minima we used 100 random initial centroid
positions. In the final solution the k-means analysis assigned 80
participants to cluster 1 and 76 participants to cluster 2.


\subsection{Cluster Evaluation}
Now that the participants have been assigned to their respective clusters based on the similarity of their time series features, the final evaluation step includes two main elements, (1) evaluating the performance of the clustering analyses to choose an optimal solution and (2) interpreting the extracted clusters conceptually. 

\subsubsection{Performance}
Performance evaluation often means assessing the accuracy, stability, and separation or purity of the clustering \citep{keogh2003}. Importantly, any evaluation of the results depends on the research questions, the data, and the methods used. However, broadly speaking, evaluation methods can be categorized based on whether the true cluster labels are known or not \citep{saxena2017}. If true class labels are known, the cluster assignments can be compared to the true class labels --- using measures such as the F-measure, adjusted Rand index, mutual information, and normalized mutual information \citep[i.e., external evaluation; e.g.,][]{liao2005}. However, if the true cluster assignments are unknown, as with our psychological time series, the quality of the clusters is assessed based on the characteristics of the data itself, such as separation and homogeneity of the clusters, or goodness of fit indices \citep[i.e., internal evaluation; e.g.,][]{Aghabozorgi2015}. 

In our own illustration example, we used the \texttt{cluster.stats()}
function from the \texttt{fpc} \textsf{R} package, which calculates a
wide variety of internal cluster validity statistics for each of the
extracted clustering solutions. With real-world data, no single
evaluation measure is likely to be perfect. Different measures may yield
varying results based on the data characteristics and the research
question at hand \citep{kittler1998}. It is therefore important to
consider a variety of evaluation measures and to carefully interpret the
results in the context of the specific analysis \citep{vinh2009}. We
found that across most indices, the analysis with \(k=2\) clusters
performed the best. Three commonly reported indices we would like to
highlight are the comparison of within clusters sum of squares, the
average silhouette score, and the Calinski-Harabasz index. The first
statistic we looked at was the total within-cluster sum of square
\(WCSS\) (see also \equatref{eq:kWCSS}). While the within-cluster
variation will naturally decrease with (more) smaller clusters, we
observed that the decrease in \(WCSS\) was largest until \(k=2\), after
which the decrease was much smaller. This method is also sometimes
referred to as the `elbow method' \citep{syakur2018}. We then looked at
a second, commonly used measure, the average silhouette score. This
statistic measures the degree to which each time feature data point is
similar to other points within the same cluster, compared to points in
other clusters \citep{rousseeuw1987}. In our case, the \(k=2\) solution
maximized the silhouette coefficient (\(s_2=\) 0.09). Finally, the
Calinski-Harabasz index assesses the compactness and separation of the
clusters by assessing the ratio of the sum of between-clusters
dispersion and of inter-cluster dispersion for all clusters --- thus,
the higher the score the better the performances \citep{calinski1974}.
In our case, the \(k=2\) solution also showed the highest
Calinski-Harabasz index (\(CH_2=\) 16.38; a full table of all extracted
validity statistics is available in Supplemental Material
A)\footnote{It is important to note that another commonly assessed aspect of the evaluation is determining the stability and robustness of the clusters \citep{berkhin2006}. This can be assessed by evaluating the sensitivity of the clusters to different feature sets or clustering algorithms, or by using techniques such as bootstrapping to assess the uncertainty of the clusters \citep{vinh2009}. Especially when comparing different clustering algorithms one common index is the Bayesian information criterion (BIC), where a lower BIC indicates that a model is more representative of the data \citep{vandeschoot2017}.}.
In the final \(k=2\) solution the k-means analysis also assigned a
relatively even number of participants to cluster 1 (\(n_{C_1}=\) 76)
and cluster 2 (\(n_{C_1}=\) 80).

To ensure that the clustering is necessary in the first place, we also
compare the performance to a single cluster solution (i.e., a single
centroid). The comparison to this \(k=1\) solution is slightly different
because metrics like the between-cluster separation are not available.
Nonetheless, comparing the within-cluster sums of squares (SS) and the
explained variance, we find that two clusters indeed outperform a single
cluster solution. Specifically, the total within-cluster SS decreased
from 8940.21 for one cluster to 8080.67 for two clusters. Additionally,
the variance explained increased from \textless.001 to 0.096 when the
cluster count was increased to two
\citep[e.g., ][; also see \situtorial\ for full results]{beijers2022}.


\subsubsection{Interpretation}
The interpretation of feature-based time series clustering in psychology involves understanding the meaning and implications of the obtained clusters. In order to make sense of the clustering results, we here focus on three general aspects of the results \citep{kaufman1990}. (1) Assessing differences between the clusters in the original time series features, (2) comparing the clusters based on prototype developments, (3) comparing the clusters based on between-person differences that were not included in the initial clustering.

In short, we find that the feature-based clustering discerned two
meaningfully different groups of participants. We find an adaptive group
(cluster 1) that reports higher well-being (\textit{median}:
\textit{difference} = -0.52, \(t\)(153.87) = -3.34, \(p\) = 0.001,
\textit{95\%CI} {[}-0.82, -0.21{]}; also see
\fgrref[A]{fig:cluster_comparison_features}) and more positive outgroup
interactions (\textit{median}: \textit{difference} = -1.38,
\(t\)(152.31) = -11.94, \(p\) \textless{} .001, \textit{95\%CI}
{[}-1.61, -1.15{]}), which are also stable over time (\textit{MAC}:
\textit{difference} = 0.54, \(t\)(153.98) = 3.49, \(p\) \textless{}
.001, \textit{95\%CI} {[}0.23, 0.84{]}) and tend to increase more over
the 30 day test period (\textit{linear trend}: \textit{difference} =
-0.55, \(t\)(149.90) = -3.55, \(p\) \textless{} .001, \textit{95\%CI}
{[}-0.85, -0.24{]}; also see
\fgrref[C]{fig:cluster_comparison_features}). This group also reported
consistently more meaningful (\textit{median}: \textit{difference} =
-1.00, \(t\)(136.40) = -7.16, \(p\) \textless{} .001, \textit{95\%CI}
{[}-1.28, -0.73{]}), need-fulfilling (\textit{median}:
\textit{difference} = -0.99, \(t\)(135.30) = -7.17, \(p\) \textless{}
.001, \textit{95\%CI} {[}-1.26, -0.72{]}), and cooperative outgroup
interactions (\textit{median}: \textit{difference} = -1.33,
\(t\)(120.36) = -11.28, \(p\) \textless{} .001, \textit{95\%CI}
{[}-1.56, -1.10{]}). This group with overwhelmingly positive experiences
stands in contrast with a more detrimental group (cluster 2). This
cluster, on average, reported much less positive, less meaningful, and
less fulfilling interactions and interaction patterns (\textit{median}).
This group also reported less positive outgroup attitudes, lower
well-being, and more discrimination experiences (\textit{median}). At
the same time, for members of this detrimental cluster (cluster 2)
conditions seemed to deteriorate over time (\textit{linear trend}), and
there was generally less consistency in the experiences they were able
to have (\textit{MAC}, \textit{MAD}, \textit{edf}; also see
\fgrref[]{fig:cluster_comparison_features}; for a full and interactive
comparison of all features see \situtorial).

To identify these patterns, we first inspect the clusters based on the
average values of meaningful features (see
\fgrref[A]{fig:clusterFeatVar}; \citealp{Kennedy2021}). We see that for
some variables the features are generally stronger in separating the
clusters. We, for example, see that the item on
`\textit{how cooperative the interaction was}' distinguishes the two
clusters across almost all seven features (except for the
\textit{auto-correlation}, see \fgrref[A]{fig:clusterFeatVar}). Compare
this to the `\textit{outgroup attitudes}' item where the differences
between the clusters are much smaller for almost all features. We then
inspect the clusters with a focus on the features (see
\fgrref[B]{fig:clusterFeatVar}). While this is the same data as for the
variable focus, we can see more clearly that some features are better at
distinguishing the clusters across variables. For example, \textit{MAD}
and \textit{median} distinguish the two clusters across almost all
variables (except for the item of whether the interaction was
representative of the outgroup). These two features stand in stark
contrast to other features, such as the \textit{lag-1 auto correlations}
or the \textit{GAM edf}, which showed much smaller differences between
the two clusters (see \fgrref[B]{fig:clusterFeatVar} and
\fgrref[]{fig:cluster_comparison_features}; please note that we offer
readers an interactive tool to assess the cluster differences for all
features in \situtorial). This offers some information on which features
were most important in differentiating the two extracted groups but also
shows that with real-world data, not all features will have enough range
to distinguish people on all variables (e.g., see the non-linearity
patterns in \fgrref[]{fig:cluster_comparison_features}; for a more
direct illustration of GAM edf differences see \citealp{bringmann2017}).
Taking these two perspectives together, we can also focus on individual
features or variables in particular. We, for example, see a strong
difference in the average well-being, where participants in cluster 2
showed a much lower median well-being over the time series
(\textit{difference} = -0.52, \(t\)(153.87) = -3.34, \(p\) = 0.001,
\textit{95\%CI} {[}-0.82, -0.21{]}). At the same time, in terms of
well-being stability, both groups have virtually identical average
\textit{MAC} statistics for well-being (\textit{difference} = -0.01,
\(t\)(153.96) = -0.04, \(p\) = 0.968, \textit{95\%CI} {[}-0.32, 0.31{]};
also see \fgrref[A]{fig:clusterFeatVar}). There are, thus, variables and
features that distinguish the clusters better than others and a
combination of variables and features lets us explore meaningful group
differences in more detail. In our case, we see that the central
tendency, variability, and linear trend are best at distinguishing a
group with mainly positive experiences (cluster 1) from a group with a
more negative experience (cluster 2). We also see that our clusters line
up with the past literature on the importance of focusing on simpler and
more meaningful statistics
\citep{bringmann2018c, eronen2021a, dejonckheere2019}.

\begin{figure}[!ht] %hbtp
  \caption{Cluster Group Comparisons based on Features and Variables}
  \label{fig:clusterFeatVar}
  \centering\includegraphics[width=\textwidth]{figures/clusterFeatVarComb_tutorial.pdf}
  \caption*{Note: \\
  "Int." = outgroup interaction, "mad" = median absolute deviation, "mac" = mean absolute change, "lin" = linear slope, "edf" = estimated degrees of freedom of an empty GAM spline model, "ar01" = lag-1 autocorrelation, "OCC"/"occ" = out-of-cluster comparison\\
  Within the "(B) Feature Focus" subplot, the 'n (within ooc)' is an out-of-cluster comparison of the within-person available measurements for each variable; the 'between ooc (mean)' are also out-of-cluster comparisons but on a between person level. 'Measurements removed' is the person-specific count of measurement occasions removed during the missingness handling and 'Discrimination' is the scale mean of daily discimination experiences (measured during the final survey).}
\end{figure}

In the second step, we look at the prototypical trajectories of the
clusters. For k-means clustering it is often recommended to use the
average over time of the responses within the cluster
\citep[see \fgrref{fig:clusterTs};][]{niennattrakul2007}\footnote{It is important to note, however, that direct comparability can be a concern, and often times some subset selection or nonlinear alignment is necessary \citep[e.g.,][]{gupta1996}.}.
Immediately striking are the mean differences, where participants in
cluster 1 had more meaningful and fulfilling outgroup interactions and
also consistently reported more voluntary and cooperative interactions
but fewer accidental and involuntary interactions. The same cluster
(cluster 1) also reported an increase in need-fulfilling interactions
over the 30-day period and an increase in interactions that were
representative of the outgroup. Whereas the other cluster (cluster 2)
showed a decrease in voluntary, cooperative, and positive interactions
over the 30 days. This `deterioration' cluster (cluster 2) also saw a
decrease in general need fulfillment but not experienced well-being over
the 30 days (see \fgrref[C]{fig:cluster_comparison_features}). We also
see that while interaction representativeness, outgroup attitudes, and
well-being are relatively stable for both clusters, the deteriorating
cluster (cluster 2) also showed substantially higher variability and
instability on most of the other variables (although these effects are
much smaller; see \fgrref[A]{fig:clusterTs}).

Finally, we can also assess the clusters across other individual
difference variables \citep[e.g.,][]{monden2022}. This out-of-feature
comparison allows us to check for data artifacts, as well as check
whether the developmental clusters are associated with important social
markers and individual differences. To illustrate artifact checks, we
added the number of ESM measurements into the comparison and find that
the participants in the deterioration cluster (cluster 2) on average
completed slightly more ESM surveys in general and reported on more
intergroup interactions in particular (see \(n\) in
\fgrref[B]{fig:clusterFeatVar}). In our data exclusion procedures, we
ensured that the general time frame and completion rates are similar for
all participants and indeed the numbers in ESM measurements generally
are largely similar (e.g., see \(n\) for well-being and outgroup
attitudes). However, the difference in the reported number of
interactions might indicate either a clustering artifact or a meaningful
difference. The higher average number of interactions in cluster 2
could, for example, indicate a clustering artifact if variances are
substantially larger due to the larger samples
\citep[e.g., restriction of range in the smaller sample][]{kogan2006}.
In our case, this seems less likely because one out of four variables
did not differ in terms of the MAD (i.e., our selected measurement of
the time series variance; see \fgrref{fig:cluster_comparison_features}
for an illustration). At the same time, however, the difference in the
number of experienced interactions might also indicate a meaningful
difference, where the deteriorating cluster (cluster 2) on average
reported more outgroup interactions (\textit{difference} = 1.03,
\(t\)(150.83) = 7.50, \(p\) \textless{} .001, \textit{95\%CI} {[}0.76,
1.30{]}), but these interactions were less voluntary
(\textit{difference} = -1.04, \(t\)(108.89) = -7.71, \(p\) \textless{}
.001, \textit{95\%CI} {[}-1.31, -0.77{]}), less meaningful
(\textit{difference} = -1.00, \(t\)(136.40) = -7.16, \(p\) \textless{}
.001, \textit{95\%CI} {[}-1.28, -0.73{]}), and less positive
(\textit{difference} = -1.38, \(t\)(152.31) = -11.94, \(p\) \textless{}
.001, \textit{95\%CI} {[}-1.61, -1.15{]}). Thus, while more research is
needed for a conclusive test, our data seems to suggest that the
differences in reported interactions are a meaningful difference between
the clusters. Such a finding would also be in line with past research
highlighting the role of negative intergroup interactions in explaining
intergroup relations \citep[e.g.,][]{Barlow2012, Prati2021, Graf2014}. A
related validity check was the inclusion of the missingness handling,
where we compared the two clusters on the average number of measurements
removed as part of the missingness handling. We find that the clusters
did not significantly differ on this metric suggesting that the
missingness handling did not affect the cluster separation (also see
\appref[]{app:ValidationAnalyses} and \situtorial).

\begin{figure}[!ht] %hbtp
  \caption{Comparison Cluster Differences by Features and Variables.}
  \label{fig:cluster_comparison_features}
  \centering\includegraphics[width=\textwidth]{figures/feature_comparison_combined.pdf}
  \caption*{Note:\\
  The figure shows the differences between the clusters in the standardized features that were entered into the dimensionality reduction (for each input variable). We display the median (panel A), the median absolute deviation (MAD, panel B), the univariate linear slope (panel C), as well as the estimated degrees of freedom of the generalized additive model splines (GAM edf, panel D). Please also note that as part of \situtorial, we provide readers with an interactive selection tool to compare cluster differences on all variables and features.}
\end{figure}

To further illustrate the utility of assessing out-of-feature individual
differences, we also compare the two samples in terms of the
participants' self-reported discrimination experiences in the
Netherlands (measured during the post-measurement). When looking at the
group comparison, we find that participants in the deteriorating cluster
(cluster 2) reported substantially higher levels of everyday
discrimination (\textit{difference} = 0.40, \(t\)(151.71) = 2.56, \(p\)
= 0.011, \textit{95\%CI} {[}0.09, 0.71{]};
\fgrref[B]{fig:clusterFeatVar}). Thus, both intensive longitudinal
(e.g., the sum of specific ESM measurements) and cross-sectional
variables (e.g., general discrimination differences) that were not
included in the original clustering step can be used to explore and
understand the cluster differences in more detail.

This cluster separation, then, has a number of empirical and practical
applications. Firstly, the clusters are descriptive. With tens of
variables, hundreds of participants, and thousands of measurements,
singular descriptive statistics are often not able to capture the
complex patterns that describe the data set. The feature-based
clustering offers some direct insight into the complexity within the
data set. In our empirical example, we, for example, see that
participants are meaningfully distinguished by a combination of high
(vs.~low) central tendency, variability, and linear trend. Secondly, the
clusters identify important groups. The adaptive and deteriorating
groups offer starting points for empirical exploration as well as
practical interventions. Researchers can start probing what exactly
distinguishes the two groups further and generate new bottom-up
hypotheses. Practitioners in the resettlement field can use the group
separation to identify individuals in need of assistance and can explore
contextual factors that might contribute to the difficulties some might
face. In our illustration we, for example, found that participants in
the deteriorating cluster (cluster 2) reported less need fulfilling
interactions over time. Thirdly, the feature-based approach is flexible
and meaningful. We were able to use a wide range of time series features
that have been central in the ESM literature and were able to use them
directly to identify meaningful groups. For our empirical illustration
we, among others for example, chose to focus on whether participants
differed in their average well-being (i.e., \textit{median}), how much
their well-being would vary over time (i.e, \textit{MAD}), and whether
their well-being would on average increase or decrease over time (i.e.,
\textit{linear trend}). Alternatively, for others cyclical patterns
might be more important --- for example, whether well-being was higher
on weekends. Importantly, in any case, we did not need to translate
these dynamic features into probabilistic inference models (e.g., VAR
models) to cluster the participants.

\begin{figure}[!ht] %hbtp
  \caption{Cluster Group Comparisons over time}
  \label{fig:clusterTs}
  \centering\includegraphics[width=\textwidth]{figures/clusterTsComb.pdf}
  \caption*{Note: \\
  Subplot (A) displays the variable cluster means at every measurement occasion. The thinner lines present all individual time series. Subplot (B) shows the GAM spline for each cluster across the measurement occasions. The thinner lines present all individual GAM Splines.}
\end{figure}


\newpage
\section{Discussion}
% aims re-iterated
The purpose of this article was to introduce feature-based time series clustering as an amenable and transparent approach to understanding between-person differences in developmental patterns of psychological time series data. Rather than relying on person-specific model parameters, which can be restrictive and assumption-bound, we argue for the more flexible and theoretically grounded approach of directly clustering on important and relevant features of the time series data. By leveraging the rich array of dynamic measures, our approach offers the advantages of flexibility, fewer strict assumptions, and improved interpretability, thus potentially enriching our understanding of heterogeneous psychological processes in intensive longitudinal studies.

% summary
To illustrate the practical utility of the approach, we applied the method to real-world empirical data that highlight common ESM issues of multivariate conceptualizations, structural missingness, and nonlinear trends \citep[e.g.,][]{ariens2020}. With the real-world data, we followed a stepwise approach to discuss key issues during input selection, feature extraction, feature reduction, feature clustering, and cluster evaluation. Within this step-wise approach, our article shows that feature-based clustering offers an excellent fit for psychological research practice, as both the features and the analysis steps are well established within the field, and statistical packages are readily available. Time series features (such as means or linear trends) are not only easy to extract, but also hold conceptual meaning for psychological data and can be chosen to address specific research questions (also see \tblref{tab:esmFeatures}).

Importantly, we show that feature-based clustering is not only approachable but provides interpretable and transparent insights about the grouped patterns. For our example of migration experiences, the method was useful to discern adaptive from more stressful experiences and helped to contextualize divergent experiences. We found that some variables, such as interaction quality perceptions or need fulfillment were particularly important in distinguishing the groups (see \fgrref[A]{fig:clusterFeatVar}). Similarly, we found that the central tendency (\textit{median}), variability (\textit{MAD}), and linear trend (\textit{slope}) were the most impactful dynamic features in discerning the trajectory clusters (see \fgrref[B]{fig:clusterFeatVar}). Jointly these two approaches allowed us to identify a cluster that had generally positive and improving experiences while the other cluster had more negative and deteriorating experiences. We were even able to further contextualize the results with out-of-feature comparisons, where we found that the group with the more difficult experiences also reported substantially more discrimination experiences during the post-test (see, e.g., \fgrref[B]{fig:clusterFeatVar}). In short, the feature-based approach allowed us to identify directly interpretable and meaningful groups, where we transparently know what data input the clusters are based on.

\subsection{Limitations}
While feature-based time series clustering offers a promising approach to understanding psychological time series data, it is not without limitations. In particular, feature-based clustering has both usability- and robustness limitations across its multiple steps. 

In terms of convenience, each of the steps requires users to make an informed decision about the choice of method and algorithm. These additional steps of decision-making and transparency increase the initial barrier to entry. We hope that our empirical illustration, the sample code, and the custom functions, offer a relatively generalizable procedure that showcases the ease of use, but clustering, unfortunately, does not offer a universal one-size-fits-all solution. 

In terms of methodological robustness, the variety of methods in each of the steps also brings with it the potential inconsistent results between methods \citep[e.g.,][]{bastiaansen2019}. A different set of variables, features, or a different clustering algorithm, might have resulted in substantially different cluster assignments. While the variety and diversity of methods are helpful in finding options even for more complex types of data, different algorithms often offer different results \citep[e.g.,][]{keogh2005}. And even when patterns produce robust clustering solutions across algorithms, individual methods might still have their idiosyncratic shortcomings \citep{xu2015}. 

As an illustration, the choice of features to extract from the time series data is a critical step that can significantly influence the results of the clustering process. In the current example, we chose to extract features such as means, autocorrelations, and linear trends, which are psychologically and conceptually meaningful in interpreting our time series clusters. However, this selection is not exhaustive and may not capture all relevant aspects of the time series data. For example, we did not consider attributes like periodicity or spectral density, which could shed light on the data's cyclical patterns. The choice of features largely hinges on the researcher's specific research question and assumptions about the data, thereby injecting a level of subjectivity into the process. Similar challenges arise with the choice of the clustering algorithm or the cluster illustration. These challenges are not unique to feature-based clustering, rather they are common to all clustering approaches \citep{liao2005, horne2020}. However, it is important to remember that multi-stepped data-driven approaches are particularly vulnerable to the impact of the researchers' degrees of freedom.

One potential remedy to many of the limitations of feature-based clustering lies in transparently and reproducibly reporting the decisions of the user for each of the analysis steps. In our own description of the method, we have provided a range of options and have motivated our own choices to facilitate the transparency of the individual steps and decision moments. Beyond the structures proposed here, \citet{vandeschoot2017} have proposed an extensive checklist for latent trajectory studies. Most of their recommendations and reporting guidelines also apply to feature-based clustering and might even offer a template for researchers who hope to preregister their analysis procedures \citep[also see][]{kirtley2021}.

\subsection{Implications}
Notwithstanding the limitations, we believe that feature-based clustering offers exciting new potential for researchers and practitioners assessing psychological time series. 

For researchers, the feature-based time series clustering approach offers a number of compelling implications. The flexibility and interpretability mean that feature-based time series clustering can be applied to a wide range of data types and research questions. 
The method can be used to contextualize preexisting groups by extracting their time series features and comparing a data-driven approach with existing group labels. Furthermore, the feature-based approach can also be used as an exploratory, descriptive, or predictive approach to intensive longitudinal data. By reducing the complexities of ESM data to important and meaningful patterns, a bottom-up approach can aid in the creation of more embedded theories and interventions, or simply in describing the often complex and heterogeneous data researchers collect during ESM studies.

Looking ahead, the feature-based time series clustering approach opens up new avenues for future research. While the approach has shown promise in dealing with the challenges of dimensionality, missingness, and time scales, there is potential for further refinement and expansion. To showcase the exciting potential for future methodological integrations, we will briefly consider the broad range of alternative approaches to time series clustering (see \fgrref{fig:tsClustTax}).

For instance, given that the approach does not assume the stationarity restrictions of many model-based approaches, future research can now more easily integrate many of the (non-)linear trend features. Research on capturing nonlinear trends has been growing over the past years and there are exciting possibilities to bring these developments to ESM data \citep{bringmann2023}. For example, bicoherence metrics, polynomial-, and differential equation parameters may be used to capture the type and structure of nonlinear developments \citep[e.g. shape-based approaches in \fgrref{fig:tsClustTax}; see also][]{caro-martin2018, mayor2022}. New features capturing nonlinear structures would add to the under-studied (non-)linear trend features of ESM data.

% Additionally, feature-based clustering algorithms traditionally do not provide a cluster-level prototype trajectory. Instead, users have to extract, often suboptimal, approximations of the group-level patterns (e.g., see the section on \textit{Cluster Evaluation}). Iterative or embedded clustering approaches, such as mixture models or group-based trajectory models, often estimate a group-based model and variance around that group-based model \citep[see \fgrref{fig:tsClustTax} and \tblref{tab:clusterApproaches}; e.g.,][]{denteuling2021, lane2019}. To provide similar functionality for feature-based models, researchers may evaluate the utility of using recurrent neural networks, such as long short-term memory (LSTM) models to generate cluster-level trajectories. One could, for example, train an LSTM based on the feature matrix and the original time series as the target \citep[e.g., see][]{fraley2002, nagin1999}. Alternatively, rule-based classifiers could extract rule sets that narrow down the conditions under which a given feature set is assigned to a specific cluster \citep[][]{benard2019}. This could potentially allow us to generate cluster-level trajectories, giving us an estimate of the typical trend for each group.

\begin{figure}[!hbtp] %hbtp
  \caption{Time Series Clustering Taxonomy}
  \label{fig:tsClustTax}
  \centering\includegraphics[width=\textwidth]{figures/TS Cluster Flow/tsClustTax.pdf}
  \caption*{Note: \\
  The taxonomy only exemplifies some of the basic differences between a number of common time series clustering approaches. As such, the taxonomy and the notes are neither exhaustive nor complete in distinguishing different approaches. Additionally, terms and labels are used inconsistently across different types of literature and are chosen to avoid overlapping labels.}
\end{figure}

Beyond the direct academic use, the feature-based time series clustering approach also addresses practical and applied uses. For practitioners, the approach offers a practical and grounded method for dealing with the challenges of complex and messy data from multiple patients, customers, or users. Not only does the approach directly deal with dimensionality, missingness, and time scales in the time series, but the interpretability and transparency aspects offer particular utility in applied settings, where the costs of misspecification are high. Additionally, the approach is also more readily accessible to practitioners who may not have extensive training in complex data analysis techniques. We provide practical algorithm overviews and readily available code for data preparation, analysis, and interpretation. The ability to identify and interpret meaningful patterns in time series data can have significant implications for practice, particularly in fields such as clinical, organizational, or social psychology, where understanding individual differences and developmental patterns can inform interventions and decision-processes.

% conclusion
In conclusion, we show that feature-based time series clustering can effectively reduce the complexities of psychological time series data to important and meaningful patterns. It does so with more flexibility, versatility, and less strict assumptions than many of the commonly used approaches to date. As such the feature-based time series clustering approach addresses key challenges in the field and aids researchers and practitioners in describing and exploring patterns across participants. We hope that the method adds to the methodological toolkit of ESM researchers and promotes the creation of more embedded methods, theories, and interventions.


% Tables
% Example
%\input{Tables/descrFullWide}


% Figures


\printbibliography

\appendix

\section{ESM Data Challenges and Promises}
\label{app:ChallengesAppendix}

\subsection{Promises}

Time series clustering has a number of conceptual use cases with psychological data. Prime among them is the ability to reduce the time, variable, and person complexity by extracting and organizing participant-level structures. These reduction and structuring qualities can be essential in detecting phenomena and extracting more abstract functional principles \citep[][]{eronen2021a}. These phenomena and principles can be meaningful differences that distinguish participants in different clusters, as well as important patterns, trends, and relationships that participants share within a cluster \citep[e.g.,][]{schrodt2000}. Once distinct groups and patterns have been identified, researchers can examine the extent to which these within-group and between-group structures are associated with other variables of interest, such as personality traits, demographic characteristics, or other psychological constructs \citep[e.g.,][]{monden2022}. By detecting meaningful and robust structures and patterns, time series clustering can, thus, be used to inform the development of robust theories as well as targeted interventions and therapies for individuals, for example, with mood disorders and other psychological conditions \citep[e.g.,][]{borsboom2021, eronen2020}.

However, while clustering can be incredibly useful, arriving at these clusters critically depends on two core challenges. First, time series need to be made comparable in order to identify key (dis)similarities and second, comparable (dis)similarities need to be accurately distinguishing into different groups \citep[e.g.,][]{Aghabozorgi2015}. In practice, most psychological time series cannot be compared based on the raw data itself. This is the case because in most cases the raw time series include too many data points --- sometimes referred to as the dimensionality curse \citep[e.g.,][]{altman2018} --- and, more importantly, individual time points are oftentimes not directly comparable between participants in psychological data and would lead to misspecifications \citep[e.g., ][]{faloutsos1994}. While such issues can be avoided with transformations for highly regular, controlled, and comparable time series such as EEG data \citep[e.g.,][]{huang1985}, most ESM researchers are usually not interested in directly comparing individual timepoints between participants but are interested in developmental patterns and relationships. 

As a result, most psychological time series are summarized via a numerical representation and these numerical summaries are then comparable and used to cluster participants (e.g., \citealp[]{timmerman2013}; see \fgrref{fig:tsClustTax}). Ideally, the representations that summarize the original time series data should (1) capture the original data accurately without loosing too much information, and (2) should be conceptually meaningful \citep[][]{vandermaaten2009}. Extracting accurate and meaningful representations of the time series can be essential for understanding what goes into the clustering algorithm (i.e., assists with explainability) and can be crucial in making sense of the final cluster output \citep[i.e., assists with interpretability; e.g.,][]{Kennedy2021}. 

\subsection{Challenges}

We will briefly consider which challenges modern ESM data introduce and what qualities are called for in an extension of the clustering repertoire. We particularly highlight issues of dimensionality, non-equidistant or missing measurements, an interest in non-stationary trends, as well as inconsistent/diverse time scales. 

Concerning dimensionality issues, especially more abstract psychological experiences often need a wider variety of measurements to be captured adequately. Today, few clinical conditions are captured with a single symptom measure \citep[e.g.,][]{cramer2016}, emotions are rarely assessed in isolation \citep[e.g.,][]{reitsema2022}, and socio-cultural experiences are now widely considered to be multimodal \citep[e.g.,][]{Kreienkamp2022d}. This also means that modern analysis techniques increasingly need be able to accommodate an increased focus on multivariate developments. At the same time, however, an increase in the number of considered variables tends to come at the expense of computational load for model estimations, and clustering models may not converge \citep[the aforementioned dimensionality curse;][]{altman2018}. A modern time series clustering technique should consequently be able to summarize and structure multivariate phenomena without running into computational load issues.

Another common type of data are measurement regiments that collect data in irregular time intervals (i.e., non-equidistant measurements). Common are, for example, procedures where participants are asked to respond at random times throughout the day (i.e., signal-contingent) or following specific natural events of interest \citep[i.e., event-contingent; see][]{shiffman2008, myin-germeys2018}. Under such conditions data tends to violate the equidistance assumption that is expected by many time series models \citep[][]{hamaker2017}. Smaller issues of non-equidistant data can be avoided with transformations \citep[e.g., dynamic time warping,][]{berndt1994} or newer modeling procedures \citep[e.g., continuous-time models;][]{dehaan-rietdijk2017} but for many analyses, including some cluster approaches, non-equidistant measurements remain a prevalent issue. 

Structural missingness remains an even more strenuous challenge. Structural missingness occurs when data is missing because it logically cannot be collected \citep[as opposed to probabilistically missing data;][]{little2020, mclean2017}. Often, however, researchers might want to include variables in their models that are not available under all conditions. Follow-up and event-contingent questions are a common example in ESM studies. Researchers, for example, ask about the frequency, intensity, or duration of symptoms --- but only if a symptom was present \citep{kivela2022}. Such approaches become specifically critical in cases of sensitive questions such as questions about suicidal ideation or other potentially trauma-inducing questions \citep[e.g.,][]{glenn2022}. The most common practice for structurally missing data is to either exclude the variable or any measurement that has no structurally missing data \citep[e.g.,][]{lavori2008}\footnote{This is the case because the most commonly used models require complete data \citep{schafer2002} and structurally missing data cannot be imputed as it logically does not exist \citep[e.g.,][]{lavori2008}.} --- neither option suits a research question that wishes to include variables with common structural missingness, such as event-specific or follow-up questions. In short, new clustering approaches should be able to deal with structurally missing data in order to address modern ESM data.

When it comes to studying developmental trajectories, psychological researchers are often also interested in nonstationary processes because they are more representative of the complex, dynamic patterns of the human mind. In psychology, nonstationary processes are typically used to study phenomena such as cognitive development \citep[][]{quartz1997}, decision-making \citep[][]{ratcliff2016}, and emotion dynamics \citep[][]{bringmann2018b}. These processes are often characterized by changes in the underlying statistical properties of the data over time, such as changes in the mean or variance \citep[][]{molenaar2009}. Especially when considering changes in mean levels, researchers are often interested in nonlinear changes because they describe human functioning better. For example, in decision making people might switch between choices \citep[][]{ratcliff2016}, or patients reducing medication might experience mood swings \citep[][]{helmich2020a}. Similarly, psychologists are often also interested in how variances change over time. This is especially the case because several changes in an individual's variance have been found to be indicative of critical changes, including depression relapses and symptom shifts more generally \citep[e.g.,][]{schreuder2020, wichers2020}. There is, thus, also a need for time series clustering algorithms that capture nonstationary processes, including nonlinear trends.

Psychological time series often exhibit complex patterns and relationships that can change over different time scales. For example, a time series of daily mood ratings may show a weekly pattern, with higher ratings on the weekends and lower ratings during the week. At the same time, the series may also exhibit a longer-term trend, with overall mood levels increasing or decreasing over the course of several months or years \citep[e.g.,][]{Ram2014}. These different time scales can be studied separately or in combination, using different statistical techniques and modeling approaches \citep[][]{bertenthal2007, jeronimus2019a}. Different time scales can become an even more difficult issue when different variables in a model develop on different time scales \citep{bringmann2022b}. Different time scales are thus also a concern clustering approaches should be able to address.

It is this background of the common challenges of current ESM data, upon which we propose to consider feature-based clustering. The flexibility of using a wide variety of features that represent the important developmental patterns allows users to circumvent many of the issues with multi-dimensionality, non-equidistant or missing measurements, non-stationary trends, as well as diverse time scales.

%These shortcomings, however, stand in sharp contrast with the types of data researchers commonly collect and do not align with common research interests in the field. Psychological researchers might, for example, collect data based on the occurrence of natural events, which tends to result in non-equidistant measurements \citep[e.g.,][]{myin-germeys2018, hamaker2017} and context-specific missingness that cannot be imputed, for instance, interaction quality perceptions \citep[e.g.,][]{kivela2022, lavori2008}. At the same time, researchers are often specifically interested in non-stationary developments, looking, for example, at how symptom levels and -variations change over time --- often abruptly and non-linearly \citep[e.g.,][]{bringmann2018b, helmich2020a}.


\end{document}
