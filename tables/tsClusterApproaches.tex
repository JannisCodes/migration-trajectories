%\begin{table}%[hbt]
\begin{sidewaystable}
    \centering
    \caption{Common Clustering Approaches.}
    \label{tab:clusterApproaches} 
    \begin{tabular}{
    >{\raggedright\arraybackslash}p{0.15\linewidth} 
    >{\raggedright\arraybackslash}p{0.30\linewidth} 
    >{\raggedright\arraybackslash}p{0.25\linewidth} 
    >{\raggedright\arraybackslash}p{0.25\linewidth}
    }
        \hline 
        Approach & Description & Characteristics & Examples \\ 
        \hline \\ [-0.5em]
        
        centroid-based \linebreak & 
        efficient, effective, and simple; sensitive to initial conditions and outliers  \linebreak &
        \vspace{-1em}
        \begin{itemize}[nosep,leftmargin=*,label={--}]
            \item[\scriptsize\faPlusCircle] simple and efficient
            \item[\scriptsize\faPlusCircle] no assumptions
            \item[\scriptsize\faMinusCircle] sensitive to initial conditions and outliers
        \end{itemize}\linebreak & 
        \vspace{-1em}
        \begin{itemize}[nosep,leftmargin=*,label={--}]
            \item k-means
            \item fuzzy c-means
        \end{itemize}\linebreak \\ 
        
        distribution-based \linebreak & 
        description 2 \linebreak &
        \vspace{-1em}
        \begin{itemize}[nosep,leftmargin=*,label={--}]
            \item[\scriptsize\faPlusCircle] probabilistic
            \item[\scriptsize\faMinusCircle] distributions sensitive
        \end{itemize}\linebreak & 
        \vspace{-1em}
        \begin{itemize}[nosep,leftmargin=*,label={--}]
            \item Gaussian mixture models
            \item negative binomial model-based
        \end{itemize}\linebreak\\ 
        
        density-based \linebreak & 
        description 3 \linebreak &
        \vspace{-1em}
        \begin{itemize}[nosep,leftmargin=*,label={--}]
            \item[\scriptsize\faPlusCircle] no shape assumption
            \item[\scriptsize\faPlusCircle] do not assign outliers 
            \item[\scriptsize\faMinusCircle] sensitive to high dimensionality
        \end{itemize}\linebreak & 
        \vspace{-1em}
        \begin{itemize}[nosep,leftmargin=*,label={--}]
            \item DBSCAN
            \item OPTICS
        \end{itemize}\linebreak\\ 

        connectivity-based \linebreak & 
        Often also called hierarchical clustering. \linebreak &
        \vspace{-1em}
        \begin{itemize}[nosep,leftmargin=*,label={--}]
            \item[\scriptsize\faPlusCircle] flexible number of clusters
            \item[\scriptsize\faPlusCircle] visual inspection (dendrogram) 
            \item[\scriptsize\faMinusCircle] small number of cases
            \item[\scriptsize\faMinusCircle] no reversal of assignments
        \end{itemize}\linebreak & 
        \vspace{-1em}
        \begin{itemize}[nosep,leftmargin=*,label={--}]
            \item Ward hierarchical
            \item agnes clustering
        \end{itemize}\linebreak \\ 
        
        hybrid \linebreak & 
        description 5 \linebreak &
        \vspace{-1em}
        \begin{itemize}[nosep,leftmargin=*,label={--}]
            \item[\scriptsize\faPlusCircle] avoid individual shortcomings
            \item[\scriptsize\faMinusCircle] less readily available
        \end{itemize}\linebreak & 
        \vspace{-1em}
        \begin{itemize}[nosep,leftmargin=*,label={--}]
            \item GMM + k-means centroids as initial values
            \item hybridHclust
        \end{itemize}\linebreak\\ 
        
        \hline \\ [-0.75em]
        \multicolumn{4}{p{\linewidth}}{\footnotesize \textit{Note.} The presented dimensionality reduction methods and -approaches are neither exhaustive nor necessary for feature-based clustering. Notable additional approaches are `embedded selection methods' that filter as part of the model estimation procedure (e.g., mixture models) and `network-based projection methods' that use neural networks to reduce dimensions (e.g., autoencoders).}
    \end{tabular}
\end{sidewaystable}
