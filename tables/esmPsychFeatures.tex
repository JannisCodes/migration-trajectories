%\begin{table}%[hbt]
\begin{sidewaystable}
    \centering
    \caption{Examples of Features for Psychological Time Series.}
    \label{tab:esmFeatures} 
    \begin{tabular}{
    >{\raggedright\arraybackslash}p{0.15\linewidth} 
    >{\raggedright\arraybackslash}p{0.35\linewidth} 
    >{\raggedright\arraybackslash}p{0.25\linewidth} 
    >{\raggedright\arraybackslash}p{0.20\linewidth}
    }
        \hline 
        Time Series Feature & Substantive Interpretation & Example Operationalizations & Formulas, refs, ...? \\ 
        \hline \\ [-0.5em]
        Diversity \newline \hl{(drop this \& keep in var. selection only?)} & 
        Multivariate measurement of concepts (e.g., affect-behavior-cognition-desire, or bio-psycho-social) \linebreak & 
        -----\linebreak  & 
        {\centering --- ? ---\par} \\
        
        Central Tendency \linebreak & 
        Average level of the experience across the entire measurement period. \linebreak & 
        \vspace{-1em}
        \begin{itemize}[nosep,leftmargin=*,label={--}]
            \item mean
            \item median
            \item mode
        \end{itemize} \linebreak  & 
        {\centering --- ? ---\par} \\ 
        
        Variability & 
        Describes the average deviation from the central tendency across the entire measurement period. \linebreak & 
        \vspace{-1em}
        \begin{itemize}[nosep,leftmargin=*,label={--}]
            \item standard deviation
            \item variation coefficient
            \item median absolute deviation
        \end{itemize} \linebreak & 
        {\centering --- ? ---\par} \\ 
        
        (In)stability & 
        Describes the average change between two consecutive measurements of the experience. \linebreak & 
        \vspace{-1em}
        \begin{itemize}[nosep,leftmargin=*,label={--}]
            \item mean sum squared differences
            \item mean absolute change
            \item Ix instability index
        \end{itemize} \linebreak & 
        {\centering --- ? ---\par} \\ 
        
        Inertia & 
        Describes how much experiences carry over to the future measurements. This includes resistance to change (i.e., carries over to the next measurement) and periodic or seasonal returns (e.g., self-predictive on a daily or weekly basis). \linebreak &
        \vspace{-1em}
        \begin{itemize}[nosep,leftmargin=*,label={--}]
            \item autocorrelation (e.g., lag–1)
            \item fourier coefficients
            \item continuous wavelet transform
        \end{itemize} \linebreak & 
        {\centering --- ? ---\par} \\ 

        Linear Trend & 
        Describes upwards or downwards linear trend of the experience reports. \linebreak & 
        \vspace{-1em}
        \begin{itemize}[nosep,leftmargin=*,label={--}]
            \item OLS regression slope
            \item avg. piecewise linear reg. slope
        \end{itemize} \linebreak & 
        {\centering --- ? ---\par} \\ 
        
        Nonlinearity & 
        Describes the nonlinear structure of the time series. This includes measures that indicate the deviation from the a linear trend as well as nonlinear model parameters. \linebreak & 
        \vspace{-1em}
        \begin{itemize}[nosep,leftmargin=*,label={--}]
            \item GAM spline edf
            \item bicoherence metrics
            \item Langevin polinomial coefficient
        \end{itemize} \linebreak & 
        {\centering --- ? ---\par} \\ 
        
        \hline \\ [-0.75em]
        \multicolumn{4}{p{\linewidth}}{\footnotesize \textit{Note.} The presented features and operationalizations are neither exhaustive nor necessary for feature-based clustering.}
    \end{tabular}
\end{sidewaystable}
