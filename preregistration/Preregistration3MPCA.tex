\documentclass[]{article}
    \usepackage{lmodern}
    \usepackage{amssymb,amsmath}
\usepackage{ifxetex,ifluatex}
\usepackage{fixltx2e} % provides \textsubscript
\ifnum 0\ifxetex 1\fi\ifluatex 1\fi=0 % if pdftex
\usepackage[T1]{fontenc}
\usepackage[utf8]{inputenc}
  \else % if luatex or xelatex
\ifxetex
\usepackage{mathspec}
\usepackage{xltxtra,xunicode}
\else
  \usepackage{fontspec}
\fi
\defaultfontfeatures{Mapping=tex-text,Scale=MatchLowercase}
\newcommand{\euro}{€}
        \fi
% use upquote if available, for straight quotes in verbatim environments
\IfFileExists{upquote.sty}{\usepackage{upquote}}{}
% use microtype if available
\IfFileExists{microtype.sty}{%
  \usepackage{microtype}
  \UseMicrotypeSet[protrusion]{basicmath} % disable protrusion for tt fonts
}{}
  \usepackage[left=2.5in,bottom=1.25in,top=1.25in,right=1in]{geometry}
  \ifxetex
\usepackage[setpagesize=false, % page size defined by xetex
            unicode=false, % unicode breaks when used with xetex
            xetex]{hyperref}
\else
  \usepackage[unicode=true]{hyperref}
\fi
\hypersetup{breaklinks=true,
bookmarks=true,
pdfauthor={},
pdftitle={Migration Experience Trajectories: A Three-Mode Principal Component Analysis},
colorlinks=true,
citecolor=BlueViolet,
urlcolor=BlueViolet,
linkcolor=BlueViolet,
pdfborder={0 0 0}}
\urlstyle{same}  % don't use monospace font for urls
\usepackage{natbib}
\bibliographystyle{apalike}
\setlength{\parindent}{0pt}
\setlength{\parskip}{6pt plus 2pt minus 1pt}
\setlength{\emergencystretch}{3em}  % prevent overfull lines
\providecommand{\tightlist}{%
\setlength{\itemsep}{0pt}\setlength{\parskip}{0pt}}
\setcounter{secnumdepth}{0}

% Customization for cos_prereg
\usepackage{longtable,booktabs,threeparttable,tabularx}
\linespread{1.5}
\newcounter{question}
\setcounter{question}{0}

%%% Use protect on footnotes to avoid problems with footnotes in titles
\let\rmarkdownfootnote\footnote%
\def\footnote{\protect\rmarkdownfootnote}

%%% Change title format to be more compact
\usepackage{titling}

\def\changemargin#1#2{\list{}{\rightmargin#2\leftmargin#1}\item[]}
\let\endchangemargin=\endlist

% Create subtitle command for use in maketitle
\newcommand{\subtitle}[1]{
\posttitle{
\begin{center}\large#1\end{center}
}
}

\setlength{\droptitle}{-2em}
\title{Migration Experience Trajectories: A Three-Mode Principal
Component Analysis}
\pretitle{\begin{changemargin}{-8pc}{0pc} \centering\large Preregistration\\ \Huge}
\posttitle{\end{changemargin}}
  \author{
          Jannis Kreienkamp\textsuperscript{1},
          Kai Epstude\textsuperscript{1},
          Rei Monden\textsuperscript{2},
          Maximilian Agostini\textsuperscript{1},
          Peter de Jonge\textsuperscript{1},
          Laura F.
Bringmann\textsuperscript{1}          \\ \vspace{0.5cm}
              \textsuperscript{1} University of Groningen, Department of
Psychology\\
              \textsuperscript{2} Hiroshima University, Graduate School
of Advanced Science and Engineering      }

  \def\affdep{{"", "", "", "", "", ""}}%
  \def\affcity{{"", "", "", "", "", ""}}%
  \preauthor{\begin{changemargin}{-8pc}{0pc} \centering\large}
  \postauthor{\end{changemargin}}
\date{01. June 2022}
\predate{\begin{changemargin}{-8pc}{0pc} \centering\large\emph}
\postdate{\end{changemargin}}
\usepackage{fancyhdr}
\pagestyle{fancy}
\setlength{\headheight}{14.0pt}
\renewcommand{\headrulewidth}{0pt}
\lhead{}
\rhead{\large\textsc{\MakeLowercase{Preregistration: Migration
Experience Trajectories}}}

\usepackage{amsmath}
\usepackage{nccmath}
\usepackage{enumerate}
\usepackage{LatexPackage/trackchanges}
\usepackage[usenames,dvipsnames]{xcolor}
\addeditor{Jannis}


% Title settings
\usepackage{titlesec}
\titleformat{\section}[display]{\bfseries\Large}{\thesection}{}{}[]
\titlespacing{\section}{0pc}{*3}{*1.5}
\titleformat{\subsection}[leftmargin]{\titlerule\bfseries\filleft}{\thesubsection}{.5em}{}
\titlespacing{\subsection}{8pc}{5ex plus .1ex minus .2ex}{1.5pc}


% Redefines (sub)paragraphs to behave more like sections
\ifx\paragraph\undefined\else
\let\oldparagraph\paragraph
\renewcommand{\paragraph}[1]{\oldparagraph{#1}\mbox{}}
\fi
\ifx\subparagraph\undefined\else
\let\oldsubparagraph\subparagraph
\renewcommand{\subparagraph}[1]{\oldsubparagraph{#1}\mbox{}}
\fi


\begin{document}
\maketitle
\vspace{2pc}


\newcommand\Question[2]{%
   \leavevmode\par
   \stepcounter{question}
   \noindent
   \textbf{\thequestion. #1}. #2\par}

\newcommand\Answer[1]{%
    \noindent
    \textit{Registered response}: #1\par}

\newlength{\mylength}
\setlength{\fboxsep}{15pt}
\setlength{\mylength}{\linewidth}
\addtolength{\mylength}{-2\fboxsep}
\addtolength{\mylength}{-2\fboxrule}
\definecolor{editPurple}{RGB}{84, 28, 117}

\hypertarget{study-information}{%
\section{Study Information}\label{study-information}}

\hypertarget{title}{%
\subsection{Title}\label{title}}

Migration Experience Trajectories: A Three-Mode Principal Component
Analysis

\hypertarget{description}{%
\subsection{Description}\label{description}}

\textbf{Shortened Description:}

Recent reviews have called for more comprehensive assessment of human
experiences and for more longitudinal real-life data, within the
psychological sciences more broadly and in migration research in
particular
\citep[e.g.,][]{Kreienkamp2022d, MacInnis2015, McKeown2017, Pettigrew2011, Ward2019}.
However, while generally speaking analytical methods for such, more
complex, data have become more readily available
\citep[e.g.,][]{ODonnell2021}, it remains unclear how we should identify
key developmental patterns --- especially across multiple variables at
the same time.

In this manuscript, we aim to assess the utility of a promising analysis
technique for describing, summarizing, and understanding the initial
developmental data patterns ---
\textit{three-mode principal component analysis} (3MPCA). We use 3MPCA
specifically because recent studies have laid out the potential
effectiveness of dimension reduction procedures, which can address the
new forms of person-, variable-, and time point heterogeneity jointly
\citep[e.g.,][]{Monden2015}. We analyse data from three experience
sampling studies, which followed the migration experiences of recent
migrants to the Netherlands. All three studies focus on the
psychological adaptation of migrants (i.e., psychological
acculturation). However, given that past investigations of psychological
acculturation have underexplored the crucial aspect of motivational
experiences \citep{Kreienkamp2022d}, the three studies have placed a
particular emphasis on the needs, goals, and motives of young migrants.
To make full use of all three data sets and to guide future use of
dimension reduction procedures with extensive psychological data, we
will analyze the data in a descriptive and step wise manner, where we
first consider the data jointly and will then assess potential
differences between datasets, variables, and time scales.

\begin{center}\rule{0.5\linewidth}{0.5pt}\end{center}

\textbf{Full Description:}

Recent reviews have called for more comprehensive assessment of human
experiences and for more longitudinal real-life data, within the
psychological sciences more broadly and in migration research in
particular
\citep[e.g.,][]{Kreienkamp2022d, MacInnis2015, McKeown2017, Pettigrew2011, Ward2019}.
However, while generally speaking analytical methods for such, more
complex, data have become more readily available
\citep[e.g.,][]{ODonnell2021}, it remains unclear how we should identify
key developmental patterns --- especially across multiple variables at
the same time.

In essence, the novel extensive longitudinal datasets come with new
forms of heterogeneity, where we have to consider differences between
people, over time, and across variables. Yet, past analytical advances
have almost exclusively pushed for inferential modeling
procedures\footnote{For example, stationary lagged regression models
  that assume stable means and variances over time (incl., vector
  autoregressive models, dynamic structural equation models,
  autoregressive integrated moving average models, and cross-lagged
  panel analyses) or basic trajectory models (e.g., mixed effects
  models, spline regression models, and latent growth curve modeling).}.
And while inferential model testing is certainly important, we still
miss discussions of methods for the more fundamental task of describing,
summarizing, and understanding the initial developmental data patterns.

As an example, a recent review of migration experiences has pointed out
that despite many complex and dynamic theories, investigations of
migrant adaptation have undervalued developmental data --- especially
when it comes to the more internal experiences of motivations and
emotions \citep{Kreienkamp2022d}. Yet, understanding how people differ
in their migration trajectories, can be crucial in understanding
adaptive and maladaptive patterns. To identify which variables are most
important in the adaptation of migrants over time, we need methods to
break down the data heterogeneity into its core components (in terms of
important variables and developments) and we need ways to identify how
these core components relate to key adaptation markers (including,
well-being, intergroup anxiety, outgroup trust, or societal
participation). There is, thus, a clear need to assess the utility of
analysis procedures focused on the description and understanding of
complex dynamical data.

In this manuscript, we aim to assess the utility of one such promising
analysis technique --- \textit{three-mode principal component analysis}
(3MPCA). We use 3MPCA specifically because recent studies have laid out
the potential effectiveness of dimension reduction procedures, which can
address the new forms of person-, variable-, and time point
heterogeneity jointly \citep[e.g.,][]{Monden2015}. We are among the
first to apply this analysis to experience sampling data and, to the
best of our knowledge, we are the first to decompose social
psychological experiences. This stands in stark contrast, to a renewed
recognition that social psychological phenomena unfold over time, a
rapid increase of extensive longitudinal data collections, and a growing
interest in understanding the co-development of multiple experience
aspects
\citep[e.g.,][]{Kreienkamp2022d, MacInnis2015, McKeown2017, Pettigrew2011, Ward2019}.

To untangle the heterogeneity in real-life migrant adaptations, we will
analyse data from three experience sampling studies, which followed the
migration experiences of recent migrants to the Netherlands. All three
studies focus on the psychological adaptation of migrants (i.e.,
psychological acculturation). However, given that past investigations of
psychological acculturation have underexplored the crucial aspect of
motivational experiences \citep{Kreienkamp2022d}, the three studies have
placed a particular emphasis on the needs, goals, and motives of young
migrants. To make full use of all three data sets and to guide future
use of dimension reduction procedures with extensive psychological data,
we will analyze the data in a descriptive and step wise manner, where we
first consider the data jointly and will then assess potential
differences between datasets, variables, and time scales.

\hypertarget{hypotheses}{%
\subsection{Hypotheses}\label{hypotheses}}

We do not have hypotheses in the traditional sense. Our analysis plan is
based on the aim of describing, summarizing, and understanding a complex
set of extensive longitudinal data. We propose that the dimension
reduction procedure we employ will identify meaningful patterns and
developments within the data, which are useful to migration researchers
and practitioners. An additional aim is to test the feasibility and
utility of simultaneous dimension reduction procedures in experience
sampling data.

\hypertarget{design-plan}{%
\section{Design Plan}\label{design-plan}}

\hypertarget{study-type}{%
\subsection{Study type}\label{study-type}}

\textbf{Observational Study}. Data is collected from study subjects that
are not randomly assigned to a treatment. This includes surveys, natural
experiments, and regression discontinuity designs.

\hypertarget{blinding}{%
\subsection{Blinding}\label{blinding}}

No participant blinding is involved in this study.

\hypertarget{study-design}{%
\subsection{Study design}\label{study-design}}

All three studies used an extensive longitudinal design. Using a daily
diary format, for at least 30 days participants received a short survey
twice per day (at around 12pm and 7pm). We additionally included a
longer pre- and post measurement survey the days before and after the
extensive longitudinal data collection.

\hypertarget{randomization}{%
\subsection{Randomization}\label{randomization}}

No randomization is involved in this study.

\hypertarget{sampling-plan}{%
\section{Sampling Plan}\label{sampling-plan}}

\hypertarget{existing-data}{%
\subsection{Existing data}\label{existing-data}}

\textbf{Registration following analysis of the data}: As of the date of
submission, you have accessed and analyzed some of the data relevant to
the research plan. This includes preliminary analysis of variables,
calculation of descriptive statistics, and observation of data
distributions. Please see cos.io/prereg for more information.

\hypertarget{explanation-of-existing-data}{%
\subsection{Explanation of existing
data}\label{explanation-of-existing-data}}

The data was collected as part of a larger collaboration on daily
intergroup relations. A sub-sample of variables has recently been
accessed by the research team for an unrelated analysis
\citep{Kreienkamp2022b}. Additionally, several of the variables were
prepared for a graphical presentation of the dataset. Thus far, none of
the proposed analyses have been conducted and none of the previous
analyses have been related to dimension reductions or time-series
modeling.

\hypertarget{data-collection-procedures}{%
\subsection{Data collection
procedures}\label{data-collection-procedures}}

We collected three experience sampling studies, following the daily
migration experiences of young migrants, who had recently arrived in the
Netherlands (median time in the Netherlands = 3 Months). For all three
studies, the data was collected in a three-step procedure:

\begin{enumerate}
\def\labelenumi{\arabic{enumi}.}
\tightlist
\item
  Entry Survey: A pre-measurement questionnaire (appr. 25 minutes)
  including demographic information, and relations to the Dutch majority
  (payment: 2 Euros).
\item
  Experience Recaps: At least 30 days of short reflection surveys (appr.
  3---5 minutes) on intergroup interactions twice a day (payment: 1 Euro
  per Recap; up to 2 Euros per day).
\item
  Conclusion Survey: On the last day, we conclude with a
  post-measurement questionnaire (appr. 25 minutes) with some questions
  on habits and reflections on the study (payment: 2 Euros).
\end{enumerate}

For the third study, participants had the option of continuing the study
for an unspecified amount of time. After the initial 30 day duration,
participants were offered the possibility to continue participating in
the study either with payment if daily diary measures were missed during
the initial study phase or without payment after a total of 60 daily
diary measurements were completed. After the initial 30-day period,
participants receive automated feedback visualizing the development of
their own well-being, attitudes, and motive responses as an additional
initiative and to give participants access to their own data and to
compensate study participation.

\hypertarget{sample-size}{%
\subsection{Sample size}\label{sample-size}}

For Study 1 (initial preliminary study) our target sample size is a sum
of 1,000 daily diary measurements. As we expected a completion rate of
around 80\% we aimed to recruite 20 participants (20 participants X 50
measurements). We further over-sampled slightly to compensate for
potential drop outs given the length and intensity of the study. The
total collected sample size of Study 1 was thus 1,225 survey responses
from 23 participants.

For Studies 2 and 3, our target sample size was a sum of 4,000 daily
diary measurements each. With 100\% completion rate that would be
archived with 67 participants (60 daily diary responses each). Given
that we expect some incomplete daily diary measurements, we again
over-sampled in both studies. In Study 2, we were ultimately able to
collect 4,965 survey responses from 113 participants. In Study 3 we
collected 4,107 survey responses from 71 participants.

\hypertarget{sample-size-rationale}{%
\subsection{Sample size rationale}\label{sample-size-rationale}}

The targeted sample size depended on a combination of different factors.
Different analyses were planned as part of the collaboration and
budgeting was a practical constraint. Some analyses were planned based
on (1) the pre- to post measurements, (2) the dynamic developments over
the daily diary measurements, or (3) the contemporary effects within the
daily diary measurements.

Power considerations of mixed effects model such as with extensive
longitudinal data are difficult to estimate because of the complex
covariance structures. Simulation studies based on the first sample
indicated that with well-distributed scales, and small to medium effect
sizes, 70-80 participants with at least seven daily diary measurements
and a simple pre--post survey were sufficiently powerful (power = .8,
alpha = .05) to answer most of the key research questions of the
collaboration.

The ultimate sampling procedure decision was made as a practical
balancing of the number of participants and the number of measurements
provided by each participants.

\hypertarget{stopping-rule}{%
\subsection{Stopping rule}\label{stopping-rule}}

Participants were recruited until the targeted number of participant
finished the pre-measurement. Invitations to complete additional daily
diary measurements (in Study 3) were extended until participants chose
to leave the study or at the most until two months (i.e., 64 days) after
the initial entry survey (i.e., from the pre measurement survey).

\hypertarget{variables}{%
\section{Variables}\label{variables}}

\hypertarget{manipulated-variables}{%
\subsection{Manipulated variables}\label{manipulated-variables}}

Not applicable given that the study design is observational.

\hypertarget{measured-variables}{%
\subsection{Measured variables}\label{measured-variables}}

Given the circumstance that we utilize three independent experience
sampling studies for our analyses, the variables are at times slightly
different between studies. Additionally, given that the analysis we aim
to undertake (i.e., 3MPCA) uses a large number of variables for the
multiple imputations, dimension reduction, and correlational analyses,
we will not list all items individually. Instead, we provide full
variable information in the following data sheets in
`VariableSelectionAnalyses.xlsx'.

\textbf{Key variables}\\
While most of the variables are identical across all studies, some
differ slightly in content or wording. However, for each study we aimed
to collect the following types of variables.

\begin{enumerate}
\def\labelenumi{\arabic{enumi}.}
\tightlist
\item
  Motivational variables

  \begin{enumerate}
  \def\labelenumii{\alph{enumii}.}
  \tightlist
  \item
    core motive fulfillment (i.e., ``\emph{During your interaction with
    -X- (/this morning) your goal (-TEXT-) was fulfilled.}'')
  \item
    goal importance ratings of 10 individual goals (e.g., ``\emph{career
    goals}'', ``\emph{health / fitness goals}'')
  \item
    self determination theory needs (i.e., autonomy, relatedness,
    competence)
  \end{enumerate}
\item
  Affective states

  \begin{enumerate}
  \def\labelenumii{\alph{enumii}.}
  \tightlist
  \item
    experienced well-being (happiness, energy)
  \item
    general emotional state scale (e.g., ``\emph{How do you feel right
    now? not angry at all to very angry}'')
  \end{enumerate}
\item
  Behavioral self-reports (Studies 2 and 3)

  \begin{enumerate}
  \def\labelenumii{\alph{enumii}.}
  \tightlist
  \item
    pro-social behavior (e.g., ``\emph{Made demeaning, rude or
    derogatory remarks about someone.}'')
  \item
    anti-social behavior (e.g., ``\emph{I was there to listen to
    someone's problems.}'')
  \end{enumerate}
\item
  Cognitive measures

  \begin{enumerate}
  \def\labelenumii{\alph{enumii}.}
  \tightlist
  \item
    outgroup attitude (``\emph{After the interaction, how favorably do
    you feel towards \ldots{} / At the moment, how favorably do you feel
    towards \ldots{} the Dutch.}'')
  \item
    Allport's contact condition perceptions (i.e., equal status, shared
    goal, cooperative, voluntary)
  \item
    interaction quality ratings (e.g., ``\emph{Overall, the interaction
    was\ldots{} Unpleasant to Pleasant}'')
  \end{enumerate}
\end{enumerate}

It should be noted, that we collected substantially more motivational
and affecive variables, as they have been understudied in the past.

For the detailed variable selection of each analysis see
`VariableSelectionAnalyses.xlsx'.

\textbf{Interpretation correlation variables}\\
To interpret the person-mode components identified by the 3MPCA, we
would like to correlate the component scores with a range of person- or
time point specific variables. While the full list of variables is
available in the accompanying data sheets (see
`VariableSelectionAnalyses.xlsx'), we provide a few examples of the
targeted categories below.

\begin{enumerate}
\def\labelenumi{\arabic{enumi}.}
\tightlist
\item
  demographic variables

  \begin{enumerate}
  \def\labelenumii{\alph{enumii}.}
  \tightlist
  \item
    age
  \item
    gender
  \item
    \ldots{}
  \end{enumerate}
\item
  societal participation

  \begin{enumerate}
  \def\labelenumii{\alph{enumii}.}
  \tightlist
  \item
    language comprehension and -use
  \item
    work inclusion
  \item
    \ldots{}
  \end{enumerate}
\item
  adaptation indicators

  \begin{enumerate}
  \def\labelenumii{\alph{enumii}.}
  \tightlist
  \item
    social identification
  \item
    intergroup anxiety
  \item
    \ldots{}
  \end{enumerate}
\item
  everyday life obstacles

  \begin{enumerate}
  \def\labelenumii{\alph{enumii}.}
  \tightlist
  \item
    discrimination experiences
  \item
    negative life events
  \item
    \ldots{}
  \end{enumerate}
\item
  individual differences

  \begin{enumerate}
  \def\labelenumii{\alph{enumii}.}
  \tightlist
  \item
    big five personality scores
  \item
    attachment style
  \item
    \ldots{}
  \end{enumerate}
\end{enumerate}

\textbf{Auxiliary variables for multiple imputation}\\
For the missing data imputation we follow the procedure reported by
\citet{Monden2015}, and include any pre-, post-, or daily dairy
variables that are significantly correlated with either the key
variables or missingness on the key variables (at \(p\) \textless{}
.01). In case we encounter a convergence problem during the multiple
imputation, we only include variable with a correlation larger than .3.
For an overview of the variables that qualify as auxiliary variables see
the enclosed data sheets (see `VariableSelectionAnalyses.xlsx').
Additional information about the multiple imputation procedure is also
available at \protect\hyperlink{statistical-models}{Statistical models}
2.

\emph{We also provide an exemplary codebook (using Study 3 as an
example; see `Codebook\_AOT-M\_ItemsPerSection.xlsx'). The full survey
files will be available as part of the main OSF repository connected to
this preregistration.}

\hypertarget{indices}{%
\subsection{Indices}\label{indices}}

\begin{enumerate}
\def\labelenumi{\arabic{enumi}.}
\tightlist
\item
  Mean Allport's conditions. We create a mean-averaged index of
  Allport's conditions in response to past findings indicating that the
  conditions are best conceptualized jointly and as functioning together
  rather than as fully independent factors
  \citep[p.~766]{Pettigrew2006}. Similar to past studies we thus hope to
  build a global indicator \citep[e.g., see][]{Pettigrew2006}. As with
  other indices we will ensure that the individual items indeed relate
  to a common latent construct and are meaningfully combined in an
  index. If this is not possible we will create sub-indices and/or
  assess the impact of the conditions separately.

  \begin{enumerate}
  \def\labelenumii{\alph{enumii}.}
  \tightlist
  \item
    The interaction with {[}name interaction partner{]} was on equal
    footing (same status)
  \item
    {[}name interaction partner{]} shared your goal ({[}free-text entry
    interaction key need{]})
  \item
    The interaction with {[}name interaction partner{]} was cooperative
  \item
    The interaction with {[}name interaction partner{]} was voluntary
  \end{enumerate}
\item
  Mean belongingness during intergroup contact

  \begin{enumerate}
  \def\labelenumii{\alph{enumii}.}
  \tightlist
  \item
    I shared information about myself.
  \item
    {[}name interaction partner{]} shared information about themselves.
  \end{enumerate}
\item
  Mean alternative interaction quality definition (``Overall, the
  interaction with {[}name interaction partner{]} was \ldots{}'')

  \begin{enumerate}
  \def\labelenumii{\alph{enumii}.}
  \tightlist
  \item
    Unpleasant to Pleasant
  \item
    Superficial to Meaningful
  \item
    Ineffective to Effective
  \item
    Unimportant to Important
  \end{enumerate}
\item
  Mood Subscales (MDMQ)

  \begin{enumerate}
  \def\labelenumii{\alph{enumii}.}
  \tightlist
  \item
    alertness
  \item
    calmness
  \item
    valence
  \end{enumerate}
\item
  Mean anti-social behaviors

  \begin{enumerate}
  \def\labelenumii{\alph{enumii}.}
  \tightlist
  \item
    Put someone down
  \item
    Show little attention in someones opinion
  \item
    Demeaning remarks
  \item
    Inappropriately addressing someone
  \item
    Ignored or excluded someone
  \item
    Doubt someones judgement
  \item
    Unwanted attempts of personal matters
  \end{enumerate}
\item
  Mean pro-social behaviors

  \begin{enumerate}
  \def\labelenumii{\alph{enumii}.}
  \tightlist
  \item
    Listen to someones problems
  \item
    Cheer someone up
  \item
    Help someone get things done
  \item
    Help someone with responsibilities
  \end{enumerate}
\end{enumerate}

Additionally, in the pre- and post-measurement surveys are several
validated scales that we will combine into their sub-scales. Indices
will be created in line with their respective validation literature. For
a full list of all variables and indices see
`VariableSelectionAnalyses.xlsx'.

\hypertarget{analysis-plan}{%
\section{Analysis Plan}\label{analysis-plan}}

\hypertarget{statistical-models}{%
\subsection{Statistical models}\label{statistical-models}}

For our analysis, we adapt the procedures outlined by
\citet{Monden2015}:

\begin{enumerate}
\def\labelenumi{\arabic{enumi}.}
\tightlist
\item
  \textbf{Sample Selection.} For our sample selection we address both
  the selection of time points and participants. Given that multiple
  imputation procedures even work well with large proportions of missing
  data {[}assuming missingness at random; \citet{Madley-Dowd2019}{]}, we
  decided on a general criterion of less than 45\% missingness to
  balance sample size retention and bias in the multiple imputation
  model. Thus, we then select the time points for which we have less
  than 45\% missingness and select participants who have less than 45\%
  missingness across the selected time range.
\item
  \textbf{Time point aggregation.} We aggregate the key variables over
  time to archive a reasonably interpretable number of time points and
  remove a first proportion of missing data. Given that little data is
  available on the meaningful time scales of the selected psychological
  variables, we chose to determine the appropriate time scales using
  variance decomposition \citep[e.g., see][]{Ram2014}. This is to say
  that we create multi-level unconditional means models (without
  predictors) that include possible nested time scales as levels. We
  chose to select time scales that align with common human cycles. We
  thus compare the variances of bi-daily, daily, and weekly
  aggregations. Additional aggregations of two weeks or the full four
  weeks might be possible but would most likely reduce the variance too
  much for any meaningul further reduction during the 3MPCA. We then
  chose the time scales that have the most variance.
\item
  \textbf{Multiple Imputation.} We then create 20 imputed datasets to
  perform the analyses on. As outlined by \citet{Monden2015}, we will
  use the key variables themself as well as auxiliary variables to
  impute missing values. Auxiliary variables are any pre-, post-, or
  daily dairy variables that are significantly (\(p\) \textless{} .01)
  and meaningfully (\(r\) \textgreater{} .3) correlated with (1) the key
  variables or (2) missingness on the key variables.
\item
  \textbf{Three-way ANOVA*.} We assess the percentages of explained
  variance for the person-, variable-, and time aspects --- which offers
  an indication of whether a 3MPCA is useful for the dataset. Based on
  the grand mean centered variables we conduct a fixed-effects three-way
  ANOVA of the person-, variable-, and time modes as well as their
  various interactions. We are particularly interested in the highest
  order interaction term Person * Variable * Time + error, as a large
  amount of variance in this effect would speak towards the possible
  interdependence of the three modes.
\item
  \textbf{Data preprocessing*.} For the 3MPCA, we center (across
  participants but within time point) and normalize (within variable but
  across time points) all key variables. The between-person centering,
  ensures that all variations are around the mean trend (which is
  removed by the centering). The normalization within variable are
  important for ensuring equal variances across variables, which ensures
  equal weighting of the variables in the 3MPCA. For methodological
  detail see \protect\hyperlink{transformations}{Transformations}.
\item
  \textbf{Selection of 3MPCA model complexity*.} We then run the 3MPCA.
  We use a generalized scree plot to select the appropriate number of
  components for each of the three modes (i.e., person, variable, time).
  To test the stability of the solution results are compared across the
  20 imputed datasets and split-half stability is assessed.
\item
  \textbf{3MPCA model fitting*.} For the selected complexities we then
  extract simple component structures (i.e., for each mode individually)
  as well as the core array (the full 3MPCA array, which specifies all
  combinations of the three mode components). A Joint Orthomax
  orthogonal rotation and standard weights are used to extract human
  interpretable component scores. This procedure is done on all 20
  imputed data sets.
\item
  \textbf{Generalized Procrustes rotation*.} In order to combine the
  three individual component structure from the 20 data sets, we use a
  generalized Procrustes rotation --- calculating the average of each
  compontent and core array.
\item
  \textbf{Explained variance*.} As fit indices we then calculate (1) fit
  percentage of the estimated array for each imputed data set (i.e.,
  explained variance around the removed general trend) as well as (2) an
  \emph{overall} fit percentage of the estimated array (i.e., explained
  variance including the general trend). Both these fit metrices should
  give an indication of how much of the original variance the reduced
  component structure was able to capture.
\item
  \textbf{Component interpretation*.} To embed and interpret the
  resulting components (especially the person components), we calculate
  correlations of the component scores with key adaptation markers
  (including, well-being, anxiety, trust, or societal inclusion). If
  components remain unclear still we can additionally perform k-means
  clustering on the lower dimensional space to identify clearer groups,
  which can be compared on the adaptation markers if necessary.
\end{enumerate}

\textbf{*Step-wise Analysis Approach:}

\begin{enumerate}[I.]
  \item \textbf{Studies (Study 1, Study 2, Study 3)}
    \begin{itemize}
      \item \textbf{Main Analysis:} All studies combined to have the most power.
      \item \textbf{Follow-up Analysis \#01:} To make sure there is no Simpson's paradox, we also separate the three data sets in separate follow-up analyses.
      \item \textbf{Follow-up Analysis\#02:} We add idiosyncratic variables (i.e., variables only available in a subset of studies) within the separated follow-up analyses. With these additional variables we explore whether our main analysis is robust to other psychological experiences or whether the additional variables might provide a qualitatively different understanding of migration processes.
    \end{itemize}
  \item \textbf{Response types (interaction, no interaction)}\footnote{Some questions were only available or had a slightly different focus depending on whether participants had an interaction or not.}
    \begin{itemize}
      \item \textbf{Main Analysis:} All available questions combined to have the most power.
      \item \textbf{Follow-up Analysis\#03:} To make sure there is no Simpson's paradox, we separate the data into two subgroup follow-up analyses (i.e., responses following an interaction and responses without interactions). This analysis will again test the robustness of the main analysis and will explore whether the core dimensions are substantially different for interaction specific phenomena.
    \end{itemize}
  \item \textbf{Time scales (half-day, day, week)}
    \begin{itemize}
      \item \textbf{Main Analysis:} For the main analysis, we will use the time scale that shows the most variance across the included items (e.g., daily; see \textit{Time point aggregation} above).
      \item \textbf{Follow-up Analysis\#04:} To explore the impact of time scales in (social) psychological data, we will also run the 3MPCA on the other time scale options (e.g., bi-daily, weekly). 
    \end{itemize}
\end{enumerate}

\hypertarget{transformations}{%
\subsection{Transformations}\label{transformations}}

\setlength{\abovedisplayskip}{0pt}
\setlength{\belowdisplayskip}{0pt}
\setlength{\abovedisplayshortskip}{0pt}
\setlength{\belowdisplayshortskip}{0pt}

For three-mode PCA it is custom to center (across participants but
within time point) and to normalize (within variable but across time
points). This is done to create a meaningful zero value for each time
point (to compare participants) and a variance of 1 across all time
points of a variable (ensuring equal weightin of variables in the
3MPCA).

We provide the main equations of the procedure below, where individual
\(i\) (\(i = 1,\ ...,\ I\)) reported variable \(x\) and its naturalized
form \(z\) (where \([k = 1,\ ...,\ K]\) indexes all available variables)
at time point \(t\) (\(t = 1,\ ...,\ T\)), so that \(I\)=number of
persons, \(K\)=number of items, \(T\)=number of time points within the
dataset. For further information as well as an example illustration see
Supplemental Material 3 in \citet{Monden2015}.

\begin{equation} \label{eq:Mean}
  \overline{x}_{kt} =  \sum_{i=1}^I x_{ikt}
\end{equation}

\begin{equation} \label{eq:NormalizationSd}
  \sigma_{k} =  \sqrt{\sum_{i=1}^I \sum_{k=1}^K \frac{(x_{ikt}-\overline{x}_{kt})^2}{I*T}}
\end{equation}

\begin{equation} \label{eq:NormalizedScore}
  z_{ikt} =  \frac{x_{ikt}-\overline{x}_{kt}}{\sigma_{k}}
\end{equation}

\hypertarget{inference-criteria}{%
\subsection{Inference criteria}\label{inference-criteria}}

Many of the proposed procedures are descriptive rather than inferential.
Importantly, we will use amounts of variances explained as our main fit
indices. If we perform inferential analyses, we will use the standard
p\textless.05 criteria for determining whether the correlation and
regression coefficients are statistically significant. We will use
appropriate multiple-comparison corrections whenever multiple tests are
performed (such as Bonferroni correction). We will place particular
emphasis on effect sizes (e.g., correlations) in our interpretations of
the results wherever possible.

\hypertarget{data-exclusion}{%
\subsection{Data exclusion}\label{data-exclusion}}

No checks will be performed to determine eligibility for inclusion
besides verification that each subject answered each of the variables of
interest for a given analysis. Outliers will generally be included in
analyses. However, we will use sensitivity analyses to assess the
robustness of the results to outliers.

\hypertarget{missing-data}{%
\subsection{Missing data}\label{missing-data}}

The sample selection based on proportions of missing data and the
multiple imputation procedures are described in
\protect\hyperlink{statistical-models}{Statistical models}.

\hypertarget{exploratory-analyses-optional}{%
\subsection{Exploratory analyses
(optional)}\label{exploratory-analyses-optional}}

A recent alternative for trajectory dimension reduction has been the use
of LSTM auto encoders. The encoder decoder classification procedure has
the advantage of being extremely flexible in the use of the data set and
could be interpreted in a similar way as the 3MPCA (see
\protect\hyperlink{statistical-models}{Statistical models} 9).

\hypertarget{other}{%
\section{Other}\label{other}}

\hypertarget{other-optional}{%
\subsection{Other (Optional)}\label{other-optional}}

Not applicable.

\hypertarget{section}{%
\subsection{}\label{section}}

\vspace{-2pc}
\setlength{\parindent}{-0.5in}
\setlength{\leftskip}{-1in}
\setlength{\parskip}{8pt}

\noindent

\bibliography{../references.bib}

\end{document}
