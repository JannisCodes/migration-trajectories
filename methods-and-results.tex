Something about the migration experience literature should go here as
part of the illustration intro \ldots{}

\subsection{Data}

We used a data set following migration experiences collected by
\citet[][]{Kreienkamp2022b}. The data set consisted of three individual
studies that followed migrants who had recently arrived in the
Netherlands in their daily interactions with the Dutch majority group.
After a general migration-focused pre-questionnaire, participants were
invited twice per day to report on their (potential) interactions with
majority group members for at least 30 days. The short ESM surveys were
sent out at around lunch (12pm) and dinner time (7pm). After the 30 day
study period participants filled in a post-questionnaire that mirrored
the pre-questionnaire. Participants received either monetary
compensation or partial course credits based on the number of surveys
they completed. For our empirical example we focused on the variables
that were collected during the ESM surveys and were available for all
three studies. Full methodological details are available in the article
by \citet[][]{Kreienkamp2022b}.

\subsubsection{Sample and time frame (?)}

filter criteria
\citep[proportion of missing data and imputations:][]{Madley-Dowd2019}

Total N t. Total N ppt.

Study specific exclusions see Online Supplementary Material A.

\subsubsection{Variables}

We focus on \ldots{}

full methodological details are available in Online Supplemental
Material A, but basic item information, descriptives, and correlations
are available in \tblref{tab:??}

\subsection{Analysis and Results}

\subsubsection{Feature extraction}

\subsubsection{Feature reduction}

\subsubsection{Feature clustering}

feature engineering: feature extraction + feature selection

use domain knowledge to extract new variables from raw data (summarize)

options: tsfresh \citep[][]{christ2018}

\textit{k} features means \(2^k – 1\) possible models

maximize relevance and minimize redundancy
