\subsection{Data}

We used a data set following migration experiences collected by
\citet[][]{Kreienkamp2022b}. The data set consists of three studies that
followed migrants who had recently arrived in the Netherlands in their
daily interactions with the Dutch majority group members. After a
general migration-focused pre-questionnaire, participants were invited
twice per day to report on their (potential) interactions with majority
group members for at least 30 days. The short ESM surveys were sent out
at around lunch (12pm) and dinner time (7pm). After the 30 day study
period participants filled in a post-questionnaire that mirrored the
pre-questionnaire. Participants received either monetary compensation or
partial course credits based on the number of surveys they completed.
For our empirical example we focus on the variables that were collected
during the ESM surveys and were available in all three studies. Full
methodological details are available in the empirical article by
\citet[][]{Kreienkamp2022b}.

\subsubsection{Sample}

The original studies included 207 participants (\(N_{S1}=\) 23,
\(N_{S2}=\) 113, \(N_{S3}=\) 71) with a total of 10,297 measurements.
The studies differed substantially in the maximum length of
participation (\(\text{max}(t_{S1})=\) 63, \(\text{max}(t_{S2})=\) 69,
\(\text{max}(t_{S3})=\) 155). This was likely due to the option to
continue participation without compensation in the later two studies. To
make the three studies comparable in participation and time frames, we
iteratively removed all measurement occasions and participants that had
more than 45\% missingness
\citep[which was in line with the general rcecommendation for data that might still need to rely on imputations for later model testing][]{Madley-Dowd2019}.
This procedure let to a final sample of 157 participants, who jointly
produced 8,132 measurements. Importantly, both the participant
repsonse-patterns and the time frame were now a lot more comparable
(\(\text{max}(t_{S1})=\) 61, \(\text{max}(t_{S2})=\) 60,
\(\text{max}(t_{S3})=\) 67). For a full overview of the data reduction
procedure and study-specific exclusions see Online Supplemental Material
A.

\subsubsection{Variables}

We focus on \ldots{}

full methodological details are available in Online Supplemental
Material A, but basic item information, descriptives, and correlations
are available in \tblref{tab:??}

\subsection{Analysis and Results}

\subsubsection{Feature extraction}

\subsubsection{Feature reduction}

\subsubsection{Feature clustering}

feature engineering: feature extraction + feature selection

use domain knowledge to extract new variables from raw data (summarize)

options: tsfresh \citep[][]{christ2018}

\textit{k} features means \(2^k – 1\) possible models

maximize relevance and minimize redundancy
