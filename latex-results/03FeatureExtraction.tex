\paragraph{Central tendency.}

The central tendency refers to the statistical measures that represent
the ``typical'' or ``average'' of a set of data. The most common
measures of central tendency are the mean, median, and mode
\citep{weisberg1992}. As a familiar statistic from probability theory,
the central tendency sits at the heart of many fundamental questions
about psychological time series. Researchers might, for example, be
interested in whether ``Over a one-month period, are some people happier
than others?''

For the central tendency feature of our illustration we chose the robust
\textit{median}, which can avoid potential issues with non-normally
distributed time series responses or outliers \citep{weisberg1992}. To
calculate the \textit{median} (\(M\)), we let \(X_{ij}\) be the ordered
list of values from the time series of variable \(j\) for participant
\(i\). The calculation depends on whether the number of measurements in
a time series \(n\) is odd or even.

\begin{equation} \label{eq:median}
  M(X_{ij}) = 
    \begin{cases}
      X \left[ \frac{n+1}{2} \right] & \text{if $n$ is odd} \\
      \frac{X \left[ \frac{n}{2} \right] + X \left[ \frac{n}{2} +1 \right]}{2} & \text{if $n$ is even}
    \end{cases}
\end{equation}

\paragraph{Variability.}

Variability captures the degree to which a set of data differs from the
central tendency and is sometimes also referred to as the dispersion or
spread of the data \citep{weisberg1992}. In time series analyses,
variability is conceptually important because information about the
distribution and diversity of the data has been found to be indicative
of worse psychological states \citep{myin-germeys2018, helmich2021}.
Person-level differences of ESM measurements have, for example, been
associated with higher levels of psycho-pathological recurrences among
depression patients \citep{timm2017}. As such, psychological researchers
and practitioners are often empirically interested in between-person
differences in variability. Researchers on polarization and
radicalization might for example ask: ``Are people settled in their
attitudes towards migrants or do they vary across the measurement
period?''

For our illustration data, we chose to capture the time series
variability with the \textit{Median Absolute Deviation} (\(MAD\)), where
we calculate the \textit{median} (\(M\); calculated as in
\equatref{eq:median}) for the absolute deviations of measurement \(x\)
at time point \(t\) for participant \(i\) and variable \(j\) from the
median of that time series \(X\). We again chose the robust statistic
because the Median-based measure is less affected by non-normal
distributions and extreme values or outliers compared to other measures
of variability like the standard deviation \citep{weisberg1992}

\begin{equation} \label{eq:mad}
  MAD(X_{ij}) = M(\left| x_{ijt} - M(X_{ij}) \right|)
\end{equation}

\paragraph{Instability.}

Instability captures the average change between two consecutive
measurements \citep{ebner-priemer2009}. While instability is
conceptually related to the variability feature, variability does not
take into account temporal dependency, whereas instability looks at the
`jumpy-ness' of the data over time. In other words, variability reflects
the range or diversity of values in a time series data, while
instability reflects the fluctuation or inconsistency in a time series
data over time \citep{trull2008}. For example, if a person has rapid and
extreme changes in mood their mood is highly unstable, while if a
person's mood responses span a wide range over the entire study period,
their mood is highly variable \citep{jahng2008}. Within psychological
time series, instability measurements have especially been important in
the research of borderline personality disorder \citep{trull2008} and
suicidality \citep{kivela2022}, but also in understanding early warning
signals more generally \citep{wichers2019}. Conceptually, the
instability feature, thus, relates to a broad range of research
questions, including: ``What is the nature of the identification changes
in those who start working in a new country?'' or ``Do strong daily
fluctuations in self-esteem reflect the process of identity formation in
adolescents?''

For our data we chose the \textit{mean absolute change}
\citep[$MAC$; e.g.,][]{ebner-priemer2009, barandas2020}, which looks at
the average absolute difference of two consecutive measurements \(x\) at
time points \(t\) and \(t-1\), for each time series \(X\) of participant
\(i\) and variable \(j\).

\begin{equation} \label{eq:mac}
  MAC(X_{ij}) = \frac{1}{n-1} \sum_{t=2, \ldots, t}\left|x_{t}-x_{t-1}\right|
\end{equation}

Another common measurement of instability is the
\textit{Mean of the Squared Successive Differences} (\(MSSD\)), which is
often preferred where differences in magnitude are more important than
the frequency of those changes, for example, when big shifts in time
series are considered more impactful or when outliers are meaningful and
need to be taken into account \citep{chatfield2003}.

\paragraph{Self-similarity.}

Self-similarity in time series data refers to the property of a time
series to exhibit similar patterns of behavior over different time
scales \citep{dmello2021}. That is, self-similarity describes how much a
measurement carries over to future measurements. One important
self-similarity in psychological time series is \texit{inertia} --- how
much a measurement carries over to its next measurement
\citep{kuppens2010, suls1998}. If inertia is high a development tends to
stay in a certain state. Because high inertia is resistant to change, in
emotion dynamics high inertia of negative affect has been found to be
indicative of under-reactive systems and to be characteristic of
psychological maladjustment \citep{kuppens2010}. In a similar vein, high
inertia in negative affect at baseline was even predictive of the
initial onset of depression \citep{kuppens2012}. Conceptually, inertia
is more broadly connected to research questions such as:
\texttt{Do\ patients\ stay\ in\ a\ negative\ mood\ for\ several\ measurements?\textquotesingle{}\textquotesingle{}\ or}Do
migrants stay with their language practice for several days at a
time?'\,'

For our illustration case, we chose the commonly used autocorrelation or
autoregression with a lag-1 to capture the inertia. High autocorrelation
values can indicate high levels of inertia, while low autocorrelation
values may indicate a more unpredictable or volatile time series
\citep{dejonckheere2019}. The lag--1 autocorrelation \(r_{ij,1}\) looks
at the average correlation between a measurement \(x\) and the preceding
measurement \(x_{t-1}\) for the time series \(X\) of participant \(i\)
and variable \(j\) with \(n\) measurements.

\begin{equation} \label{eq:ar}
  r_{ij,1} = \frac{\sum_{t=2}^{n}(x_{ijt}-\overline{x}_{ij})(x_{ij,t-1}-\overline{x}_{ij})}{\sum_{t=1}^{n}(x_{ijt}-\overline{x}_{ij})^2}
\end{equation}

Where \(\overline{x}_{ij}\) is the mean of the time series \(x_{ij}\),
calculated as:

\begin{equation} \label{eq:mean_for_ar1}
  \overline{x}_{ij} = \frac{1}{n} \sum_{t=1}^{n} x_{ijt}
\end{equation}

\paragraph{Linear trend.}

In non-stationary time series, a linear trend can be observed when there
is a consistent increase or decrease in the data over time
\citep{nyblom1986}. For psychological time series, researchers have, for
example, pointed out the importance of linear trends in interpersonal
communications \citep{vasileiadou2014}, and emotion dynamics
\citep{oravecz2016}. Theoretically, linear trends are often considered
the simplest way of assessing whether a psychological theory of change
is appropriate \citep{gottman1969}. In empirical practice, linear trends
are, thus, commonly exemplified by research questions such as ``Do
patient symptoms improve consistently?'' or ``Does worker productivity
decline continuously?''

For the variables in our illustration data set, we chose an overall
linear regression slope to capture the linear trend. The regression
slope \(b_{ij}\) provides the average change from one time point \(t\)
to the next across all measurements \(x\) of a time series \(X\) of
participant \(i\) and variable \(j\). The specific form of the OLS slope
formula we provide below calculates \(b_{ij}\) as the sum across all
time points of the product of the deviation of time \(t\) from its mean
\(\overline{t}\) and the deviation of \(x_{ij}\) from its mean
\(\overline{x}_{ij}\) at each time point, divided by the sum across all
time points of the square of the deviation of time from its mean
(\(\sum(t-\overline{t})^2\)). Intuitively, the formula captures the rate
of change of variable \(x_{ij}\) with respect to time. This slope will
indicate how the variable \(x_{ij}\) changes over time, controlling for
its mean value and the mean of time. If the slope is positive,
\(x_{ij}\) increases over time; if it's negative, \(x_{ij}\) decreases
over time.

\begin{equation} \label{eq:lin}
  b_{ij} = \frac{\sum(t-\overline{t})(x_{ijt}-\overline{x}_{ij})}{\sum(t-\overline{t})^2}
\end{equation}

\paragraph{Nonlinearity.}

Changes in psychology are not always linear, instead, nonlinearity is a
common feature of psychological time series \citep{hayes2007}. As an
example, episodic disorders, such as depression, are most likely best
described as non-linear systems \citep{hosenfeld2015}. Similarly,
patients in recovery from depression showed sudden changes in the
improvement of depression \citep{helmich2020a}. But also substance abuse
\citep{boker1998} or attitude changes rarely develop linearly
\citep{vandermaas2003}. Conceptually, researchers might have research
questions about the type of the development: ``Is the development of
well-being a nonlinear process?'' as well as the shape and structure of
the development: ``How many spikes in well-being did a migrant
experience?''

We summarized the nonlinear trend with the
\textit{estimated degrees of freedom} of an empty GAM spline model. The
\(edf\) summarizes the \textit{wiggliness} of a spline trend line
\citep{wood2017}. The degrees of freedom of a spline model are primarily
determined by the number of knots and the order of the spline. For
instance, a cubic spline with \(k\) knots has \(k\)+3 degrees of freedom
\citep{faraway2016}. However, in a penalized spline framework, which is
commonly used for GAMs, the effective degrees of freedom can be less
than \(k\)+3. This is because the model employs a smoothing parameter to
control the trade-off between the complexity (flexibility) of the model
and its fit to the data, thereby penalizing overly complex models and
potentially reducing the effective degrees of freedom \citep{marx1998}.
Intuitively then an edf of 1 would be equivalent to a linear
relationship (i.e., one linear slope parameter), whereas a higher edf
(particularly an edf \textgreater{} 2) is indicative of a non-linear
trend. The estimated degrees of freedom are commonly based on a concept
called `effective degrees of freedom' and can be represented as the
trace \(tr\) (i.e., the sum of the diagonal elements) of the smoother
matrix \(S\), a symmetric matrix that maps from the raw data to the
smooth estimates \citep{wood2017}.

\begin{equation} \label{eq:df}
  edf = tr(S)
\end{equation}

Beyond our main features of interest, we also extracted the
participant's number of completed ESM measurements to ensure that the
clusters are comparable in that regard (i.e., to exclude spurious
explanations for the cluster assignments). After the feature extraction,
we found that about 1.40\% of the extracted features are missing across
the 72 features per participant. This might, for example, happen if
participants do not have two subsequent measurements with outgroup
interactions, so that an autocorrelation with lag-1 cannot be calculated
for the contact-specific variables. The small number of missing values
indicates that the feature-based approach indeed largely avoids the
structural missingness issue. However, even the few missing values can
be an issue for some feature reduction or feature clustering algorithms.
We, thus, impute the missing feature values with a single predictive
mean matching imputation using the MICE library \citep[][]{buuren2011}.
Note again that with this procedure we only need to impute an extremely
small number of missing values as most feature calculations can use the
available data instead.
