For the features, we selected and extracted one of each of the features
proposed in \tblref{tab:esmFeatures}. We particularly chose the more
robust median and median absolute deviation (MAD) for central tendency
and variability. Within the variability structure, we chose the slightly
more intuitive mean absolute change (MAC) for stability and stayed with
the common lag--1 autocorrelation for inertia. For the trend summaries,
we chose the overall linear regression slope to describe the linear
trend and we summarized the nonlinear trend with the estimated degrees
of freedom of an empty GAM spline model (edf) --- the edf summarizes the
\textit{wiggliness} of the spline trend line. We chose these features in
particular because we consider them to be the most broadly applicable
features. We also extracted the participant's number of completed ESM
measurements to ensure that the clusters are comparable in that regard.
We provide an R-function that automatically extracts and prepares a
large selection of the time series feature operationalizations presented
in \tblref{tab:esmFeatures} in our GitHub repository (see the
\texttt{featureExtractor} function).

For the central tendency: \textit{median} (\(M\)) where \(X_{ij}\) is
the an ordered list of values from the time series of variable \(j\) for
participant \(i\). The calculation depends on whether the number of
measurements in a time series \(n\) is odd or even.

\begin{equation} \label{eq:median}
  M(X_{ij}) = 
    \begin{cases}
      X \left[ \frac{n+1}{2} \right] & \text{if $n$ is odd} \\
      \frac{X \left[ \frac{n}{2} \right] + X \left[ \frac{n}{2} +1 \right]}{2} & \text{if $n$ is even}
    \end{cases}
\end{equation}

For distribution we chose the \textit{Median Absolute Deviation}
(\(MAD\)), where we calculate the \textit{median} (\(M\); calculated as
in \eqref{eq:median}) for the absolute deviations of measurement \(x\)
at time point \(t\) for participant \(i\) and variable \(j\) from the
median of that time series \(X\).

\begin{equation} \label{eq:mad}
  MAD(X_{ij}) = M(\left| x_{ijt} - M(X_{ij})} \right|)
\end{equation}

mean absolute change (MAC) for stability

\begin{equation} \label{eq:mac}
  MAC(X_{ij}) = \frac{1}{n-1} \sum_{t=1, \ldots, t-1}\left|x_{t+1}-x_t\right|
\end{equation}

lag--1 autocorrelation for inertia: lag \(l=1\)

\begin{equation} \label{eq:ar}
  r_{ij,1} = \frac{\sum_{t=1}^{n-l}(x_{ijt}-\overline{x}_{ij})(x_{ijt-l}-\overline{x}_{ij})}{\sum_{t=1}^{n}(x_{ijt}-\overline{x}_{ij})^2}
\end{equation}

overall linear regression slope to describe the linear trend

\begin{equation} \label{eq:lin}
  b_{ij} = \frac{\sum(t-\overline{t})(x_{ijt}-\overline{x}_{ij})}{\sum(t-\overline{t})^2}
\end{equation}

summarized the nonlinear trend with the estimated degrees of freedom of
an empty GAM spline model (edf) --- the edf summarizes the
\textit{wiggliness} of the spline trend line. We estimate the number of
parametric and non-parametric (or smooth) terms \(p\) in the model.

\begin{equation} \label{eq:edf}
  edf \approx \sum p
\end{equation}

\textit{Notes}: In a Generalized Additive Model (GAM), the estimated
degrees of freedom refers to the number of parameters used to estimate
the model. This includes the number of parameters used to estimate each
smooth function in the model, as well as any additional parameters used
for the overall model. On the other hand, the effective degrees of
freedom is a measure of the amount of information in the data that is
used to estimate the model. It takes into account the smoothness of the
estimated functions in the model, and is typically smaller than the
estimated degrees of freedom. Effective degrees of freedom is used to
estimate the uncertainty of the model estimates and to calculate the
model's goodness of fit. In general, the effective degrees of freedom is
a more appropriate measure of the complexity of the model than the
estimated degrees of freedom, as it accounts for the smoothness of the
estimated functions, which can help to prevent overfitting. This is
because if the estimated function is smooth, it means that the model is
not fitting the noise and has a better generalization.

After the feature extraction, we found that about 1.40\% of the
extracted features are missing across the 72 features per participant.
This might, for example, happen if participants do not have two
subsequent measurements with outgroup interactions, so that an
autocorrelation with lag-1 cannot be calculated for the contact-specific
variables. The small number of missing values indicates that the
feature-based approach indeed largely avoids the structural missingness
issue. The few missing values can, however, be an issue for some feature
reduction or feature clustering algorithms. We, thus, impute the missing
feature values with a single predictive mean matching imputation using
the MICE library.
