During the k-means clustering itself, the analysis seeks to minimize the
total within-cluster variation. The analysis is designed to optimize the
clustering of the feature data into \(k\) groups, where \(k\) is a
pre-defined number of clusters. We used the Hartigan and Wong algorithm,
which is a widely used algorithm in k-means clustering
\citep{hartigan1979}. The algorithm starts by randomly separating the
data points into k clusters and then iteratively updates the assignment
of each point to the nearest cluster center until convergence. To do so,
the Hartigan and Wong algorithm specifically calculates the
within-cluster variation (\(W\)) of cluster \(C_i\) as the summed
squared Euclidean distances of the feature \(x\) to the closest cluster
centroid \(\mu_i\):

\begin{equation} \label{eq:kWCi}
  W(C_i) = \sum_{x \in C_i}(x-\mu_i)^2
\end{equation}

By summing the within-cluster sum of squares from all \(k\) clusters, we
can then derive the total within-cluster sum of square \(WCSS\):

\begin{equation} \label{eq:kWCSS}
  WCSS = \sum_{i=1}^k W(C_i) = \sum_{i=1}^k \sum_{x \in C_i} (x - \mu_i)^2
\end{equation}

It is this \(WCSS\) that becomes the objective function to be minimized,
by iteratively moving features from one cluster to another
\citep{hartigan1979}. In particular, the algorithm (1) calculates the
cluster centroids of the initial partitioning, (2) checks whether any
feature has a centroid that is closer than that of the currently
assigned cluster (3) updates the centroids based on any reassigned
features, and then iterates between steps two and three until \(WCSS\)
is minimized (i.e., locally optimal convergence) or a maximum number of
iterations is reached \citep{jain2010}. Given the iterative nature of
the algorithm, the initial partitioning is often important because the
algorithm might arrive at a suboptimal clustering where the \(WCSS\)
cannot be further reduced by moving any feature to another cluster,
despite a better solution existing
\citep[i.e., a local minimum;][]{timmerman2013}. It is, therefore often
recommended to run the k-means clustering with several different
starting positions.

In our case we entered the participants' PC-scores from the feature
reduction step into the k-means algorithm. Because we did know the
underlying number of clusters within our sample, we calculated the
cluster solutions for \(k=\{2, \dots , 10\}\) . To avoid local minima we
used 100 random initial centroid positions for each run. Each of the 9
cluster solutions converged within the iteration limit. In the next step
we will then evaluate which of the extracted cluster solutions offers
the best fit with the data.
