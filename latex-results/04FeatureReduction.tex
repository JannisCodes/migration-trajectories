For our own illustration data we chose a feature projection method to
reduce the dimensionality of our extracted features. In particular, we
selected the commonly used \textit{principal component analysis} (PCA).
Some of the more tailor-made feature selection algorithms can be more
accurate in reducing the feature dimensionality and might retain feature
importance information more directly, depending on the specific data
structure. However, PCAs have the distinct benefit that they are
well-established within the psychometric literature and can broadly be
applied to a wide variety of studies in an automatize manner. As our aim
is to present a general illustration that can also be adopted for more
general data descriptive uses, we present the workflow using a PCA here
but we encourage users to consider more specialized methods as well.

To use the principal component analysis with our extracted time series
features, we first standardize all features across participants to
ensure that all features are weighted equally. We then enter all 72
features into the analysis. The PCA uses linear transformations in such
a way that the first component captures the most possible variance of
the original data (e.g., by finding a vector that maximizes the sum of
squared distances). The following components will then use the same
method to iteratively explain the most of the remaining variance while
also ensuring that the components are linearly uncorrelated. In
practice, this meant that the PCA decomposed the 72 features into 72
principal components but now (because of the uncorrelated linear
transformations) the first few principal components will capture a
majority of the variance. We can then decide how much information (i.e.,
variance) we are willing to sacrifice for a reduced dimensionality. A
common rule of thumb is to use the principal components that jointly
explain 70--90\% of the original variance (i.e., cumulative percentage
explained variance). For our illustration we select the first 27
principal components that explain 80\% of the variance in the original
72 features (reducing the dimensionality by 62.50\%). For the extracted
principal components we save the 27 PC-scores for each participant
(i.e., the participants' coordinates in the reduced dimensional space).

We would like to comment on two practical matters when using principal
components --- the amount of dimensionality reduction and the
interpretation of the principal components. As for the expected
dimensionality reduction, given its methodology, PCAs tend to `work
better' at reducing dimensions with (highly) correlated variables. Thus,
with a set of very homogeneous variables and features users will need
less principal components to explain a large amount of variance, while a
more diverse set of variables and features will tend to require more
principal components to capture the same amount of variance. Our 27
principal components are still a relatively high number of variables but
this is not surprising as we chose a diverse conceptualization and a
diverse set of time series features. As for the interpretability, PCA
allows users to extract information on the meaning of the principal
components. In particular, because the principal components are linear
combination of original features, users can extract the relative
importance of each feature for the extracted principal components (i.e.,
the eigenvectors). While this can be useful in understanding the
variance in the original data or help with manual feature selection, we
use the PCA purely to reduce the dimensionality for the clustering step.
Instead of relying on the principal components, we use the original
features of interest to interpret the later extracted clusters. We
particularly advocate for such an approach if all original features are
considered psychologically meaningful in understanding the time series
and users would like to retain the features for inpterpretation
(irrespective of the features importance).
