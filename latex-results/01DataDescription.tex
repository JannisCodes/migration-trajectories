The data set we use consists of three studies that followed migrants who
had recently arrived in the Netherlands in their daily interactions with
the Dutch majority group members \citet[][]{Kreienkamp2022b}. After a
general migration-focused pre-questionnaire, participants were invited
twice per day to report on their (potential) interactions with majority
group members for at least 30 days. The short ESM surveys were sent out
at around lunch (12pm) and dinner time (7pm). After the 30 day study
period participants filled in a post-questionnaire that mirrored the
pre-questionnaire. Participants received either monetary compensation or
partial course credits based on the number of surveys they completed.

The original studies included 207 participants (\(N_{S1}=\) 23,
\(N_{S2}=\) 113, \(N_{S3}=\) 71) with a total of 10,297 ESM
measurements. Each of the studies focused on recently arrived
first-generation migrants and each study included a number of
ideosyncratic variables relevant for the broader research collective.
For our empirical example we focus on the variables that were collected
during the ESM surveys and were available in all three studies. Variable
selection and preparation is described as part of the illustration below
but additional methodological details about the study setup are
available in \citet[][]{Kreienkamp2022b}.
