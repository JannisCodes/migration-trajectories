In order to interpret the two clusters, we compare the two clusters in
their features as well as in their raw time series. We first inspect the
clusters with a focus on the included variables (see
\fgrref[A]{fig:clusterFeatVar}). We (1) see that for some variables the
features are generally stronger in separating the clusters (e.g., `how
cooperative an interaction was' compared to `attitudes towards the
Dutch'). Additionally, we see that (2) within the variables some
features are better at distinguishing clusters (e.g., median of
well-being vs.~MAC of well-being).We then inspect the clusters with a
focus on the features (see \fgrref[B]{fig:clusterFeatVar}). While this
is the same data as for the variable focus, we can see more clearly that
some features are better at distinguishing the clusters across variables
(e.g., mean and median compared to auto correlations). This offers some
information on which features were are most important in understanding
the two extracted groups.

\begin{figure}[!ht] %hbtp
  \caption{Cluster Group Comparisons based on Features and Variables}
  \label{fig:clusterFeatVar}
  \centering\includegraphics[width=\textwidth]{figures/clusterFeatVarComb.pdf}
  \caption*{Note: \\
  Within the "(B) Feature Focus" subplot, the 'n' and 'Discrimination' comparison variables were not part of the original time series clustering.}
\end{figure}

We can then combine the two focus approaches to assess the developments
between the two groups more holistically (see \fgrref{fig:clusterTs}).
Immediately striking are the mean differences, where participants in the
second cluster had more meaningful and fulfilling outgroup interactions
also consistently reported more voluntary and cooperative interactions
but less accidental and involuntary interactions. The same cluster also
reported an increase in need fulfilling interaction over the 30 day
period and an increase in interactions that were representative of the
outgroup. Whereas the other cluster showed a decrease in voluntary,
cooperative, and positive interactions over the 30 days. This
`deterioration' cluster also saw a decrease in general need fulfillment
and experienced well-being over the 30 days (see
\fgrref[B]{fig:clusterTs}). We also see that while interaction
representativeness, outgroup attitudes, well-being are relatively stable
for both clusters, the deteriorating cluster also showed substantially
higher variablity and instability over time (also see
\fgrref[A]{fig:clusterTs}).

We can also assess the clusters across any other person-level variable.
This out-of-feature comparison allows us to check for data artifacts, as
well as check whether the developmental clusters are associated with
important social markers and individual differences. To illustrate
artifact checks, we added the number of measurements into the comparison
and find that the participants in the detereoration cluster on average
completed more ESM surveys and reported on more intergroup interactions
than the cluster with the more positive interactions (see
\fgrref[B]{fig:clusterTs}). While this difference could indicate that
the clusters might not entirely be comparable in the response patterns,
we can find some relief in our data exclusion procedures during which we
ensured that the general time frame and completion rates were not too
dissimilar. To illustrate the utility of individual differences, we
compare the two samples in terms of the participants' self-reported
discrimination experiences in the Netherlands (measured during the
post-measurement). \fgrref[B]{fig:clusterTs} illustrates that
participants in the deteriorating cluster reported substantially higher
levels of everyday discrimination. Thus, both intensive longitudinal and
cross-sectional variables that were not included in the original
clustering step can be used to explore and understand the cluster
differences in more detail.

\begin{figure}[!ht] %hbtp
  \caption{Cluster Group Comparisons over time}
  \label{fig:clusterTs}
  \centering\includegraphics[width=\textwidth]{figures/clusterTsComb.pdf}
  \caption*{Note: \\
  Subplot (A) displays the variable cluster means at every measurement occasion. Subplot (B) shows the GAM spline for each cluster across the measurement occasions.}
\end{figure}
