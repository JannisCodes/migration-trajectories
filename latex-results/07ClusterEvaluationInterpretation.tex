We first inspect the clusters based on the average values of meaningful
features (see \fgrref[A]{fig:clusterFeatVar}; \citealp{Kennedy2021}). We
see that for some variables the features are generally stronger in
separating the clusters. We, for example see that the item on `\textit{how
cooperative the interaction was}' distinguishes the two clusters across
almost all seven features (except for the auto-correlation, see
\fgrref[A]{fig:clusterFeatVar}). Compare this to the `\textit{outgroup
attitudes}' item where the differences between the clusters are much
smaller for almost all features. We then inspect the clusters with a focus
on the features (see \fgrref[B]{fig:clusterFeatVar}). While this is the
same data as for the variable focus, we can see more clearly that some
features are better at distinguishing the clusters across variables. For
example, \textit{MAD} and \textit{median} distinguish the two clusters across almost all
variables (except for the item of whether the interaction was
representative of the outgroup). These two features stand in stark
contrast to other features, such as the \textit{lag-1 auto correlations}, which
showed much smaller differences between the two clusters (see
\fgrref[B]{fig:clusterFeatVar}). This offers some information on which
features were most important in understanding the two extracted
groups. Taken these two perspectives together, we can also focus on
individual features or variables in particular. We, for example, see a
strong difference in the average well-being, where participants in
cluster 2 showed a much lower median well-being over the time series. At
the same time, in terms of stability, both groups have virtually
identical averages \textit{MAC} statistics for well-being (see
\fgrref[A]{fig:clusterFeatVar}). There are, thus, variables and features
that distinguish the clusters better than others and a combination of
variables and features lets us explore meaningful group differences in
more detail. In our case, we see that the central tendency, variability, and linear trend are best at distinguishing a group with mainly positive experiences (cluster 1) from a group with a more negative experience (cluster 2). We also see that our clusters line up with the past literature on the importance of focusing on simpler and more meaningful statistics \citep{bringmann2018c, eronen2021a}.

\begin{figure}[!ht] %hbtp
  \caption{Cluster Group Comparisons based on Features and Variables}
  \label{fig:clusterFeatVar}
  \centering\includegraphics[width=\textwidth]{figures/clusterFeatVarComb.pdf}
  \caption*{Note: \\
  Within the "(B) Feature Focus" subplot, the '\textit{n}' and '\textit{Discrimination}' comparison variables were not part of the original time series clustering.}
\end{figure}

In the second step, we look at prototypical trajectories of the
clusters. For k-means clustering it is often recommended to use the
average over time of the responses within the cluster
\citep[see \fgrref{fig:clusterTs};][]{niennattrakul2007}\footnote{It is important to note, however, that direct comparability can be a concern, and often times some subset selection or nonlinear alignment is necessary \citep[e.g.,][]{gupta1996}. Additionally, finding cluster prototypes is often substantially easier with embedded clustering methods because in many cases a cluster-level model is estimated as part of the expectation–maximization procedure \citep[e.g.,][]{denteuling2021} or \textit{S-GIMME} \citep[e.g.][]{lane2019}. For medoid-based clustering algorithms, a common approach is simply using cluster medoid as the prototype \citep{kaufman1990}.}.
Immediately striking are the mean differences, where participants in the
second cluster had more meaningful and fulfilling outgroup interactions
also consistently reported more voluntary and cooperative interactions
but less accidental and involuntary interactions. The same cluster also
reported an increase in need-fulfilling interaction over the 30-day
period and an increase in interactions that were representative of the
outgroup. Whereas the other cluster showed a decrease in voluntary,
cooperative, and positive interactions over the 30 days. This
`deterioration' cluster also saw a decrease in general need fulfillment
and experienced well-being over the 30 days (see
\fgrref[B]{fig:clusterTs}). We also see that while interaction
representativeness, outgroup attitudes, and well-being are relatively stable
for both clusters, the deteriorating cluster also showed substantially
higher variablity and instability on most of the other variables (also see
\fgrref[A]{fig:clusterTs}).

Finally, we can also assess the clusters across other person-level
variable \citep[e.g.,][]{monden2022}. This out-of-feature comparison
allows us to check for data artifacts, as well as check whether the
developmental clusters are associated with important social markers and
individual differences. To illustrate artifact checks, we added the
number of measurements into the comparison and find that the
participants in the deterioration cluster on average completed more ESM
surveys and reported on more intergroup interactions than the cluster
with the more positive interactions (see \(n\) in
\fgrref[B]{fig:clusterFeatVar}). While this difference could indicate
that the clusters might not entirely be comparable in the response
patterns, we can find some relief in our data exclusion procedures
during which we ensured that the general time frame and completion rates
are similar for all participants. To illustrate the utility of
individual differences, we compare the two samples in terms of the
participants' self-reported discrimination experiences in the
Netherlands (measured during the post-measurement). When looking at the
group comparison, find that participants in the deteriorating cluster
reported substantially higher levels of everyday discrimination
(\fgrref[B]{fig:clusterFeatVar}). Thus, both intensive longitudinal and
cross-sectional variables that were not included in the original
clustering step can be used to explore and understand the cluster
differences in more detail.

In short, we find that that the feature-based clustering discerned two
meaningfully different groups of participants. We find an adaptive group
(cluster 2) that reports higher well-being and more positive outgroup
attiudes (median) that are also stable over time (MAD, MAC) and tend to
increase over the 30 day test period (linear trend). This group also
reported consistently more meaningful, need-fulfilling, and cooperative
outgroup interactions (median). This group with overwhelmingly positive
experiences stands in contrast with a more detrimental group (cluster
1). This cluster, on average, reported much less positive, less
meaningful, and less fulfilling interactions and interaction patterns
(median). This group also reported less positive outgroup attitudes,
lower well-being, and more discrimination experiences (median). At the
same time, for members of this detrimental cluster conditions seemed to
deteriorate over time (linear trend), and there was generally less
consistency in the experiences they were able to have (MAC, MAD, edf).

This cluster separation, then, has a number of empirical and practical
applications. Firstly, the clusters are descriptive. With tens of
variables, hundreds of participants, and thousands of measurements,
singular descriptive statistics are often not able to capture the
complex patterns that describe the data set. The feature-based
clustering offers some direct insight into the complexity within the
data set. Secondly, the clusters identify important groups. The adaptive
and deteriorating groups offer starting points for empirical exploration
as well as practical interventions. Researchers can start probing what
exactly distinguishes the two groups further and generate new bottom-up
hypotheses. Practitioners, can use the group separation to identify
individuals in need of assistance and can explore contextual factors
that might contribute to the difficulties some might face. Thirdly, the
feature-based approach is flexible and meaningful. We were able to use a
wide range of time series features that have been central in the ESM
literature and were able to use them directly to identify meaningful
groups.

\begin{figure}[!ht] %hbtp
  \caption{Cluster Group Comparisons over time}
  \label{fig:clusterTs}
  \centering\includegraphics[width=\textwidth]{figures/clusterTsComb.pdf}
  \caption*{Note: \\
  Subplot (A) displays the variable cluster means at every measurement occasion. Subplot (B) shows the GAM spline for each cluster across the measurement occasions.}
\end{figure}
