For our illustration, we include 12 variables that were measured as part
of the ESM surveys in all three studies and captured information about
the participant's interactions, as well as cognitive-, emotional-, and
motivational self in relationship with the majority group (see
\tblref{tab:var_selection} for an overview). We chose these aspects in
particular because (1) the interaction-specific information exemplified
the structural missingness issue of modern ESM data and (2) the
motivational, emotional, and cognitive experience offered a diverse
conceptualization of migration experience (beyond behavioral
measurements) that is becoming more common in the literature
\citep[][]{Kreienkamp2022d}. The breadth of the included variables also
showcases the utility of the method for a growing body of literature
that considers heterogeneous and complex concepts. As a result, the
number of included variables is also on the higher end for psychological
concepts and additionally allows us to showcase the efficiency benefits
of the method and offers a reasonable use case for the optional feature
reduction step.

\begin{sidewaystable*}[!hbtp]
\centering

\caption{\label{tab:var_selection}Variable Selection}
\centering
\resizebox{\linewidth}{!}{
\begin{tabular}[t]{lllccc}
\toprule
\multicolumn{4}{c}{ } & \multicolumn{2}{c}{Contact Specific} \\
\cmidrule(l{3pt}r{3pt}){5-6}
Variable & Question & Aspect & ESM & unspecific & Interaction Only\\
\midrule
AttitudesPartner & At the moment, how favorably do you feel towards -NAME- & cognition & \Checkmark &  & \Checkmark\\
InteractionContextAccidental & The interaction with -NAME- was accidental. & cognition & \Checkmark &  & \Checkmark\\
InteractionContextCooperative & The interaction with -NAME- was cooperative. & cognition & \Checkmark &  & \Checkmark\\
InteractionContextRepresentativeNL & The interaction with -NAME- was representative of the Dutch. & cognition & \Checkmark &  & \Checkmark\\
InteractionContextvoluntary & The interaction with -NAME- was voluntary. & cognition & \Checkmark &  & \Checkmark\\
KeyNeedDueToPartner & -NAME- helped fulfill your goal (-GOAL-) & needs & \Checkmark &  & \Checkmark\\
qualityMeaning & Overall, the interaction with -NAME- was: Superficial --- Meaningful & cognition & \Checkmark &  & \Checkmark\\
qualityOverall & Overall, the interaction with -NAME- was: Unplesant --- Plesant & cognition & \Checkmark &  & \Checkmark\\
AttitudesDutch & At the moment, how favorably do you feel towards the Dutch. & cognition & \Checkmark & \Checkmark & \\
DaytimeNeedFulfillment & During this -morning/afternoon- your goal (-GOAL-) was fulfilled. & needs & \Checkmark & \Checkmark & \\
KeyNeedFulfillment & During your interaction with -NAME- your goal (-GOAL-) was fulfilled. & needs & \Checkmark & \Checkmark & \\
exWB & How do you feel right now? very sad --- very happy & emotion & \Checkmark & \Checkmark & \\
\bottomrule
\end{tabular}}
\end{sidewaystable*}


Once the important variables have been selected, the data needs to be
prepared for the analysis steps. Importantly, this not only means
validating and cleaning the data (e.g., re-coding, removing duplicate or
unwanted measurements) but also making the time-series comparable. Two
important steps are making the time-frames and response scales
comparable across participants --- for example, by choosing a time frame
that is common to most participants and standardizing the participants'
responses
\citep['data exclusion' and 'data transformation' in \fgrref{fig:TSCFlow}; also see][]{liao2005}.

In our illustration data set, the studies differed substantially in the
maximum length of participation (\(\text{max}(t_{S1})=\) 63,
\(\text{max}(t_{S2})=\) 69, \(\text{max}(t_{S3})=\) 155). This was
likely due to the option to continue participation without compensation
in the latter study. To make the three studies comparable in
participation and time frames, we iteratively removed all measurement
occasions and participants that had more than 45\% missingness
\citep[which was in line with the general recommendation for data that might still need to rely on imputations for later model testing][]{Madley-Dowd2019}.
This procedure led to a final sample of 157 participants, who jointly
produced 8,132 measurements. Importantly, both the participant
response-patterns and the time frame were now substantially more
comparable (\(\text{max}(t_{S1})=\) 61, \(\text{max}(t_{S2})=\) 60,
\(\text{max}(t_{S3})=\) 67). Full methodological details are available
in Online Supplemental Material A, but basic item information,
descriptives, and correlations are also available in
\tblref{tab:descrLong}.

\begin{sidewaystable*}[!hbtp]
\centering

\caption{\label{tab:descrLong}Correlation Table and Descriptive Statistics}
\centering
\resizebox{\linewidth}{!}{
\begin{tabular}[t]{lcccccccccccc}
\toprule
  & \makecell{Int: \\ Accidental} & \makecell{Int: \\ Voluntary} & \makecell{Int: \\ Cooperative} & \makecell{Int: \\ Representative} & \makecell{Int: \\ Meaningful} & \makecell{Int: \\ Quality} & Core Need & \makecell{Core Need \\ Due to Partner} & \makecell{Attitude \\ Partner} & \makecell{Daytime \\ Core Need} & \makecell{Outgroup \\ Attitude} & Well-being\\
\midrule
\addlinespace[0.3em]
\multicolumn{13}{l}{\textbf{Correlations}}\\
\hspace{1em}Int: Accidental &  & -0.20 & -0.21* & 0.07 & -0.10 & -0.13 & -0.39*** & -0.26* & -0.05 & -0.28** & -0.03 & 0.12\\
\hspace{1em}Int: Voluntary & -0.14*** &  & 0.63*** & -0.04 & 0.12 & 0.44*** & 0.19 & 0.16 & 0.42*** & 0.10 & 0.25** & -0.05\\
\hspace{1em}Int: Cooperative & -0.14*** & 0.32*** &  & 0.19 & 0.38*** & 0.68*** & 0.42*** & 0.53*** & 0.47*** & 0.24* & 0.28** & -0.08\\
\hspace{1em}Int: Representative & 0.28*** & 0.39*** & -0.11*** &  & 0.10 & 0.13 & -0.09 & -0.03 & 0.14 & -0.02 & 0.43*** & -0.07\\
\hspace{1em}Int: Meaningful & 0.01 & 0.06** & 0.21*** & 0.30*** &  & 0.66*** & 0.11 & 0.20 & 0.44*** & 0.10 & 0.01 & -0.03\\
\hspace{1em}Int: Quality & 0.07** & 0.44*** & 0.33*** & 0.04 & 0.16*** &  & 0.42*** & 0.34*** & 0.65*** & 0.23* & 0.25* & 0.07\\
\hspace{1em}Core Need & 0.12*** & -0.07** & 0.08*** & 0.41*** & 0.02 & -0.02 &  & 0.65*** & 0.11 & 0.64*** & 0.15 & 0.30**\\
\hspace{1em}Core Need Due to Partner & -0.18*** & 0.18*** & 0.20*** & 0.58*** & 0.15*** & 0.15*** & 0.19*** &  & 0.14 & 0.53*** & 0.13 & 0.07\\
\hspace{1em}Attitude Partner & 0.21*** & 0.27*** & 0.32*** & 0.24*** & 0.17*** & 0.17*** & 0.22*** & 0.16*** &  & -0.09 & 0.57*** & 0.08\\
\hspace{1em}Daytime Core Need & 0.29*** & 0.10*** & 0.53*** & 0.26*** & 0.15*** & 0.14*** & 0.37*** & 0.17*** & 0.33*** &  & 0.07 & 0.17\\
\hspace{1em}Outgroup Attitude & 0.01 & 0.18*** & -0.04 & -0.06* & 0.14*** & 0.20*** & 0.09*** & -0.03 & 0.16*** & 0.26*** &  & 0.21*\\
\hspace{1em}Well-being & -0.09*** & 0.33*** & 0.30*** & 0.11*** & 0.09*** & 0.31*** & -0.06** & 0.23*** & 0.14*** & 0.20*** & 0.24*** & \\
\addlinespace[0.3em]
\multicolumn{13}{l}{\textbf{Descriptives}}\\
\hspace{1em}Grand Mean & 39.10 & 80.08 & 79.55 & 64.65 & 61.16 & 79.85 & 85.42 & 78.52 & 80.59 & 76.48 & 66.84 & 49.64\\
\hspace{1em}Between SD & 31.14 & 20.61 & 18.41 & 21.12 & 24.62 & 17.05 & 16.01 & 21.53 & 16.33 & 21.63 & 18.54 & 31.95\\
\hspace{1em}Within SD & 28.72 & 19.27 & 17.43 & 19.92 & 22.32 & 16.37 & 18.63 & 20.02 & 15.81 & 22.26 & 9.45 & 25.72\\
\hspace{1em}ICC(1) & 0.21 & 0.29 & 0.27 & 0.35 & 0.31 & 0.25 & 0.18 & 0.26 & 0.25 & 0.20 & 0.77 & 0.52\\
\hspace{1em}ICC(2) & 0.90 & 0.93 & 0.93 & 0.89 & 0.94 & 0.92 & 0.91 & 0.92 & 0.91 & 0.92 & 0.99 & 0.98\\
\bottomrule
\multicolumn{13}{l}{\rule{0pt}{1em}\textit{Note: }}\\
\multicolumn{13}{l}{\rule{0pt}{1em}Upper triangle: Between-person correlations;}\\
\multicolumn{13}{l}{\rule{0pt}{1em}Lower triangle: Within-person correlations;}\\
\multicolumn{13}{l}{\rule{0pt}{1em}*** p < .001, ** p < .01,  * p < .05}\\
\end{tabular}}
\end{sidewaystable*}

