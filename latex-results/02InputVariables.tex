For our illustration, we include 12 variables that were measured as part
of the ESM surveys in all three studies and captured information about
the participant's interactions, as well as the cognitive-, emotional-,
and motivational self in relationship with the majority group (see
\tblref{tab:var_selection} for an overview). We chose these aspects in
particular because (1) the interaction-specific information exemplified
the structural missingness issue of modern ESM data and (2) the
motivational, emotional, and cognitive experience offered a diverse
conceptualization of migration experience (beyond behavioral
measurements) that is becoming more common in the literature
\citep[][]{Kreienkamp2022d}. The breadth of the included variables also
showcases the utility of the method for a growing body of literature
that considers heterogeneous and complex concepts. As a result, the
number of included variables is also on the higher end for psychological
concepts and additionally allows us to showcase the efficiency benefits
of the method and offers a reasonable use case for the feature reduction
step.

\begin{sidewaystable*}[!hbtp]
\centering

\caption{\label{tab:var_selection}Variable Selection}
\centering
\resizebox{\linewidth}{!}{
\begin{tabular}[t]{lllccc}
\toprule
\multicolumn{4}{c}{ } & \multicolumn{2}{c}{Contact Specific} \\
\cmidrule(l{3pt}r{3pt}){5-6}
Variable & Question & Aspect & ESM & unspecific & Interaction Only\\
\midrule
AttitudesPartner & At the moment, how favorably do you feel towards -NAME- & cognition & \Checkmark &  & \Checkmark\\
InteractionContextAccidental & The interaction with -NAME- was accidental. & cognition & \Checkmark &  & \Checkmark\\
InteractionContextCooperative & The interaction with -NAME- was cooperative. & cognition & \Checkmark &  & \Checkmark\\
InteractionContextRepresentativeNL & The interaction with -NAME- was representative of the Dutch. & cognition & \Checkmark &  & \Checkmark\\
InteractionContextvoluntary & The interaction with -NAME- was voluntary. & cognition & \Checkmark &  & \Checkmark\\
KeyNeedDueToPartner & -NAME- helped fulfill your goal (-GOAL-) & needs & \Checkmark &  & \Checkmark\\
qualityMeaning & Overall, the interaction with -NAME- was: Superficial --- Meaningful & cognition & \Checkmark &  & \Checkmark\\
qualityOverall & Overall, the interaction with -NAME- was: Unplesant --- Plesant & cognition & \Checkmark &  & \Checkmark\\
AttitudesDutch & At the moment, how favorably do you feel towards the Dutch. & cognition & \Checkmark & \Checkmark & \\
DaytimeNeedFulfillment & During this -morning/afternoon- your goal (-GOAL-) was fulfilled. & needs & \Checkmark & \Checkmark & \\
KeyNeedFulfillment & During your interaction with -NAME- your goal (-GOAL-) was fulfilled. & needs & \Checkmark & \Checkmark & \\
exWB & How do you feel right now? very sad --- very happy & emotion & \Checkmark & \Checkmark & \\
\bottomrule
\multicolumn{6}{l}{\rule{0pt}{1em}\textit{Note: }}\\
\multicolumn{6}{l}{\rule{0pt}{1em}All items used a continuous slider and were rescaled to a range of 0--100.}\\
\end{tabular}}
\end{sidewaystable*}


Once the important variables have been selected, the data needs to be
prepared for the analysis steps. Importantly, this not only means
validating and cleaning the data (e.g., re-coding, combining scale
items) but also making the time-series comparable. Making the
time-frames and response scales comparable across participants, for
example, includes choosing a time frame that is common to most
participants
\citep['data preparation' and 'data cleaning' in \fgrref{fig:TSCFlow}; also see][]{liao2005}.

In our illustration data set, the studies differed substantially in the
maximum length of participation (\(\text{max}(t_{S1})=\) 63,
\(\text{max}(t_{S2})=\) 69, \(\text{max}(t_{S3})=\) 155). This was
likely due to the option to continue participation without compensation
in the latter study. To make the three studies comparable in
participation and time frames, we iteratively removed all measurement
occasions and participants that had more than 45\% missingness
\citep[which was in line with the general recommendation for data that might still need to rely on imputations for later model testing; see][]{Madley-Dowd2019}\footnote{Please note that for cases where the clustering is the main analysis, this high missingness threshold may be too conservative. As part of our validation analyses in \appref[]{app:ValidationAnalyses} we compare the model presented here with varying levels of missing data allowed.}.
This procedure led to a final sample of 157 participants, who jointly
produced 8,132 survey responses. Importantly, both the participant
response patterns and the time frame were now substantially more
comparable (number of measurement occasions per person: \(t_{S1}\): min
= 40, max = 61, mean = 57.33, sd = 4.69; \(t_{S2}\): min = 33, max = 60,
mean = 49.05, sd = 6.73; \(t_{S3}\): min = 36, max = 65, mean = 54.20,
sd = 7.04). It is important to consider that some time series features
may be less reliable when the number of measurement occasions per person
is low (e.g., below 30 measurements per person), and this should be
taken into account when conducting similar analyses. Full methodological
details are available in Online Supplemental Material A, but basic item
information, descriptives, and correlations are also available in
\tblref{tab:descrLong}.

\begin{sidewaystable*}[!hbtp]
\centering

\caption{\label{tab:descrLong}Correlation Table and Descriptive Statistics}
\centering
\resizebox{\linewidth}{!}{
\begin{tabular}[t]{lcccccccccccc}
\toprule
  & \makecell{Int: \\ Accidental} & \makecell{Int: \\ Voluntary} & \makecell{Int: \\ Cooperative} & \makecell{Int: \\ Representative} & \makecell{Int: \\ Meaningful} & \makecell{Int: \\ Quality} & \makecell{Int: \\ Need Fulfil.} & \makecell{Int: \\ Need Fulfil. Partner} & \makecell{Attitude \\ Partner} & \makecell{Daytime \\ Core Need} & \makecell{Outgroup \\ Attitude} & Well-being\\
\midrule
\addlinespace[0.3em]
\multicolumn{13}{l}{\textbf{Correlations}}\\
\hspace{1em}Int: Accidental &  & -0.18 & -0.20 & 0.05 & -0.09 & -0.09 & -0.35*** & -0.23* & -0.04 & -0.26* & -0.01 & 0.18\\
\hspace{1em}Int: Voluntary & -0.15*** &  & 0.61*** & -0.06 & 0.09 & 0.40*** & 0.17 & 0.14 & 0.40*** & 0.09 & 0.23* & -0.05\\
\hspace{1em}Int: Cooperative & -0.14*** & 0.28*** &  & 0.17 & 0.37*** & 0.67*** & 0.42*** & 0.52*** & 0.45*** & 0.24* & 0.25** & -0.07\\
\hspace{1em}Int: Representative & 0.00 & 0.07** & 0.12*** &  & 0.09 & 0.13 & -0.08 & -0.06 & 0.14 & -0.02 & 0.41*** & -0.03\\
\hspace{1em}Int: Meaningful & -0.19*** & 0.21*** & 0.28*** & 0.00 &  & 0.65*** & 0.08 & 0.17 & 0.43*** & 0.08 & -0.03 & -0.05\\
\hspace{1em}Int: Quality & -0.09*** & 0.32*** & 0.39*** & 0.06** & 0.44*** &  & 0.41*** & 0.33*** & 0.63*** & 0.23* & 0.23* & 0.10\\
\hspace{1em}Int: Need Fulfillment & -0.08*** & 0.18*** & 0.27*** & 0.10*** & 0.17*** & 0.32*** &  & 0.65*** & 0.10 & 0.64*** & 0.15 & 0.31**\\
\hspace{1em}Int: Need Fulfillment Partner & -0.11*** & 0.20*** & 0.33*** & 0.08*** & 0.20*** & 0.32*** & 0.52*** &  & 0.13 & 0.52*** & 0.13 & 0.08\\
\hspace{1em}Attitude Partner & -0.04 & 0.30*** & 0.30*** & 0.03 & 0.41*** & 0.58*** & 0.23*** & 0.26*** &  & -0.10 & 0.56*** & 0.11\\
\hspace{1em}Daytime Need Fulfillment & -0.06** & 0.11*** & 0.16*** & 0.02 & 0.15*** & 0.16*** & 0.15*** & 0.14*** & 0.09*** &  & 0.07 & 0.15\\
\hspace{1em}Outgroup Attitude & -0.02 & 0.14*** & 0.16*** & 0.14*** & 0.20*** & 0.31*** & 0.19*** & 0.21*** & 0.37*** & 0.09*** &  & 0.25**\\
\hspace{1em}Well-being & -0.05* & 0.16*** & 0.16*** & -0.03 & 0.22*** & 0.32*** & 0.15*** & 0.13*** & 0.26*** & 0.19*** & 0.24*** & \\
\addlinespace[0.3em]
\multicolumn{13}{l}{\textbf{Descriptives}}\\
\hspace{1em}Grand Mean & 39.10 & 80.08 & 79.55 & 64.65 & 61.16 & 79.85 & 85.42 & 78.52 & 80.59 & 76.48 & 66.84 & 74.82\\
\hspace{1em}Between SD & 31.14 & 20.61 & 18.41 & 21.12 & 24.62 & 17.05 & 16.01 & 21.53 & 16.33 & 21.63 & 18.54 & 15.97\\
\hspace{1em}Within SD & 28.72 & 19.27 & 17.43 & 19.92 & 22.32 & 16.37 & 18.63 & 20.02 & 15.81 & 22.26 & 9.45 & 12.86\\
\hspace{1em}ICC(1) & 0.21 & 0.29 & 0.27 & 0.35 & 0.31 & 0.25 & 0.18 & 0.26 & 0.25 & 0.20 & 0.77 & 0.52\\
\hspace{1em}ICC(2) & 0.90 & 0.93 & 0.93 & 0.89 & 0.94 & 0.92 & 0.91 & 0.92 & 0.91 & 0.92 & 0.99 & 0.98\\
\bottomrule
\multicolumn{13}{l}{\rule{0pt}{1em}\textit{Note: }}\\
\multicolumn{13}{l}{\rule{0pt}{1em}"Int." = outgroup interaction, "ICC" = intraclass correlation coefficient, "SD" = standard deviation}\\
\multicolumn{13}{l}{\rule{0pt}{1em}Upper triangle: Between-person correlations;}\\
\multicolumn{13}{l}{\rule{0pt}{1em}Lower triangle: Within-person correlations;}\\
\multicolumn{13}{l}{\rule{0pt}{1em}*** p < .001, ** p < .01,  * p < .05}\\
\end{tabular}}
\end{sidewaystable*}

